\documentclass{article}
\input{draft_preamble.tex}
\usepackage{amsmath}



\begin{document}


\begin{LARGE}
	\begin{center}
		Preliminary Results – OECD Climate surveys	
	\end{center}
	
\end{LARGE}
	\tableofcontents
	\listoftables

\clearpage

\section{Pre-treatment}

\subsection{Energie Characteristics}

\begin{table}[h!]
	\caption{Main way of heating} \label{table heating}
	\begin{center}
		\scalebox{0.7}{
\begin{tabular}{@{\extracolsep{5pt}}lcccc} 
\\[-1.8ex]\hline 
\hline \\[-1.8ex] 
 & \multicolumn{4}{c}{Main way of heat at home} \\ 
\cline{2-5} 
\\[-1.8ex] & Electricity & Gas & Heating oil & Renewable \\ 
\\[-1.8ex] & (1) & (2) & (3) & (4)\\ 
\hline \\[-1.8ex] 
 White only & 0.172$^{*}$ & $-$0.133 & 0.033 & $-$0.002 \\ 
  & (0.089) & (0.089) & (0.054) & (0.043) \\ 
  & & & & \\ 
 Male & 0.005 & $-$0.142$^{*}$ & 0.044 & 0.056 \\ 
  & (0.078) & (0.077) & (0.047) & (0.038) \\ 
  & & & & \\ 
 Children & 0.012 & $-$0.019 & 0.048 & 0.007 \\ 
  & (0.079) & (0.078) & (0.048) & (0.038) \\ 
  & & & & \\ 
 No college & 0.243$^{***}$ & $-$0.249$^{***}$ & 0.003 & $-$0.039 \\ 
  & (0.088) & (0.088) & (0.054) & (0.043) \\ 
  & & & & \\ 
 Retired & 0.075 & $-$0.045 & $-$0.061 & 0.062 \\ 
  & (0.140) & (0.138) & (0.085) & (0.067) \\ 
  & & & & \\ 
 Student & $-$0.601$^{*}$ & 0.179 & 0.535$^{***}$ & 0.030 \\ 
  & (0.333) & (0.330) & (0.202) & (0.161) \\ 
  & & & & \\ 
 Working & 0.097 & 0.064 & $-$0.080 & 0.008 \\ 
  & (0.138) & (0.137) & (0.084) & (0.067) \\ 
  & & & & \\ 
 Income Q2 & $-$0.054 & 0.013 & 0.104 & $-$0.027 \\ 
  & (0.118) & (0.117) & (0.071) & (0.057) \\ 
  & & & & \\ 
 Income Q3 & 0.020 & 0.028 & $-$0.049 & 0.024 \\ 
  & (0.112) & (0.111) & (0.068) & (0.054) \\ 
  & & & & \\ 
 Income Q4 & $-$0.128 & 0.144 & $-$0.006 & $-$0.012 \\ 
  & (0.119) & (0.118) & (0.072) & (0.057) \\ 
  & & & & \\ 
 30-49 & $-$0.143 & 0.069 & 0.007 & 0.013 \\ 
  & (0.193) & (0.191) & (0.117) & (0.093) \\ 
  & & & & \\ 
 50-87 & $-$0.374$^{*}$ & 0.310 & 0.055 & 0.036 \\ 
  & (0.199) & (0.197) & (0.121) & (0.096) \\ 
  & & & & \\ 
 Non voting & $-$0.019 & 0.155 & $-$0.083 & 0.036 \\ 
  & (0.124) & (0.123) & (0.075) & (0.060) \\ 
  & & & & \\ 
 Other & $-$0.285 & 0.146 & 0.129 & 0.065 \\ 
  & (0.181) & (0.179) & (0.110) & (0.087) \\ 
  & & & & \\ 
 Trump & $-$0.221$^{***}$ & 0.203$^{**}$ & 0.009 & 0.019 \\ 
  & (0.083) & (0.082) & (0.050) & (0.040) \\ 
  & & & & \\ 
 Climate treatment only & $-$0.105 & 0.047 & 0.072 & 0.022 \\ 
  & (0.112) & (0.111) & (0.068) & (0.054) \\ 
  & & & & \\ 
 No treatment & 0.066 & $-$0.056 & 0.028 & $-$0.031 \\ 
  & (0.110) & (0.110) & (0.067) & (0.053) \\ 
  & & & & \\ 
 Policy treatment only & 0.009 & $-$0.038 & 0.033 & $-$0.026 \\ 
  & (0.100) & (0.100) & (0.061) & (0.049) \\ 
  & & & & \\ 
 Constant & 0.576$^{**}$ & 0.291 & $-$0.012 & $-$0.034 \\ 
  & (0.242) & (0.240) & (0.147) & (0.117) \\ 
  & & & & \\ 
\hline \\[-1.8ex] 
Mean &  &  &  &  \\ 
Observations & 191 & 191 & 191 & 191 \\ 
\hline 
\hline \\[-1.8ex] 
\textit{Note:}  & \multicolumn{4}{r}{$^{*}$p$<$0.1; $^{**}$p$<$0.05; $^{***}$p$<$0.01} \\ 
\end{tabular} 
}
	\end{center}
	{\footnotesize Note: The dependent variables are indicator variables equal to one if the respondent indicates that this source of energy was her main way of heating at home. The \textit{Renewable} variable corresponds to the answer ``Wood, solar, geothermal, or heat pump."
	The \textit{race: White only} indicator variable equals one if the respondent's self reported race is only ``White". The regression includes controls for gender, having children and having completed a college degree. The three \textit{status} indicator variables indicate the difference in mean compared to a reference group of people not working (either unemployed or inactive). The \textit{status: Working} indicator variable includes respondents who self-reported being either ``Full-time employed", ``Part-time employed", or ``Self-employed". The three \textit{Income} indicator variables indicate difference in mean compared to a reference group of people in the first quartile of household's annual income in 2019 (income $<$ \textdollar 35,000). The two \textit{age} indicator variables indicate difference in mean compared to a reference group of people aged between 18 and 29. The two \textit{vote} indicator variables indicate difference in mean compared to a reference group of people who either did not vote in the 2020 Presidential election or voted for another candidate than Biden or Trump.
	\newline  *p$<$0.1; **p$<$0.05; ***p$<$0.01}
\end{table}	

\begin{table}[h!]
	\caption{Consumption and GHG}
	\begin{center}
		\scalebox{0.7}{
\begin{tabular}{@{\extracolsep{5pt}}lccc} 
\\[-1.8ex]\hline 
\hline \\[-1.8ex] 
 & \multicolumn{3}{c}{Household behavior} \\ 
\cline{2-4} 
\\[-1.8ex] & Km driven (2019) & Flights (2015-19) & Rarely eat beef \\ 
\\[-1.8ex] & (1) & (2) & (3)\\ 
\hline \\[-1.8ex] 
 White only & 6,043.559 & $-$5.477 & $-$0.052 \\ 
  & (11,677.840) & (4.502) & (0.084) \\ 
  & & & \\ 
 Male & 9,276.885 & 0.162 & 0.076 \\ 
  & (10,196.930) & (3.929) & (0.073) \\ 
  & & & \\ 
 Children & 4,575.636 & 1.755 & $-$0.143$^{*}$ \\ 
  & (10,344.300) & (3.985) & (0.074) \\ 
  & & & \\ 
 No college & $-$87.655 & 0.997 & $-$0.075 \\ 
  & (11,605.810) & (4.453) & (0.083) \\ 
  & & & \\ 
 Retired & $-$6,520.138 & 2.143 & 0.055 \\ 
  & (18,265.610) & (7.031) & (0.131) \\ 
  & & & \\ 
 Student & 3,368.230 & 7.547 & 0.534$^{*}$ \\ 
  & (43,495.630) & (16.765) & (0.312) \\ 
  & & & \\ 
 Working & $-$7,220.331 & 13.541$^{*}$ & 0.111 \\ 
  & (18,027.830) & (6.952) & (0.129) \\ 
  & & & \\ 
 Income Q2 & $-$1,734.426 & 0.012 & 0.003 \\ 
  & (15,499.660) & (5.929) & (0.110) \\ 
  & & & \\ 
 Income Q3 & 8,220.868 & 2.063 & $-$0.184$^{*}$ \\ 
  & (14,657.770) & (5.629) & (0.105) \\ 
  & & & \\ 
 Income Q4 & 18,635.790 & 9.768 & $-$0.075 \\ 
  & (15,531.520) & (5.974) & (0.111) \\ 
  & & & \\ 
 30-49 & 4,923.735 & 9.658 & $-$0.043 \\ 
  & (25,197.100) & (9.718) & (0.181) \\ 
  & & & \\ 
 50-87 & 423.609 & 14.741 & 0.172 \\ 
  & (25,990.010) & (10.018) & (0.186) \\ 
  & & & \\ 
 Non voting & $-$9,460.838 & 2.591 & $-$0.155 \\ 
  & (16,231.560) & (6.250) & (0.116) \\ 
  & & & \\ 
 Other & $-$3,621.577 & $-$7.767 & 0.108 \\ 
  & (23,618.680) & (9.109) & (0.170) \\ 
  & & & \\ 
 Trump & 3,057.661 & $-$1.974 & $-$0.087 \\ 
  & (10,835.880) & (4.160) & (0.077) \\ 
  & & & \\ 
 Both treatments & 31,043.940$^{**}$ & 10.500$^{*}$ & 0.080 \\ 
  & (14,548.890) & (5.565) & (0.104) \\ 
  & & & \\ 
 Climate treatment only & 7,662.152 & $-$3.859 & 0.066 \\ 
  & (13,608.610) & (5.248) & (0.098) \\ 
  & & & \\ 
 Policy treatment only & $-$2,664.481 & 1.998 & $-$0.098 \\ 
  & (12,476.710) & (4.811) & (0.090) \\ 
  & & & \\ 
 Constant & $-$5,100.263 & $-$11.909 & 0.337 \\ 
  & (31,154.870) & (12.018) & (0.224) \\ 
  & & & \\ 
\hline \\[-1.8ex] 
Mean & 18387.393 & 10.108 & 0.292 \\ 
Observations & 190 & 191 & 191 \\ 
\hline 
\hline \\[-1.8ex] 
\textit{Note:}  & \multicolumn{3}{r}{$^{*}$p$<$0.1; $^{**}$p$<$0.05; $^{***}$p$<$0.01} \\ 
\end{tabular} 
}
	\end{center}
	{\footnotesize Note: The variables \textit{Km drive (2019)} is a continous variable corresponding to the self-reported kilometers driven by the respondent's household in 2019. The \textit{Flights (2015-19)} corresponds to the self-reported number of round-trip flights taken between 2015 and 2019 included. The \textit{Rarely eat beef} is an indicator variable equal to one if the respondent indicates that she never eats beef or eats beef less than once a week.}
\end{table}	

\begin{landscape}
\begin{table}[h!]
	\caption{Main mode of transports used}
	\begin{center}
		\scalebox{0.6}{
\begin{tabular}{@{\extracolsep{5pt}}lccccccccc} 
\\[-1.8ex]\hline 
\hline \\[-1.8ex] 
 & \multicolumn{9}{c}{Transports used} \\ 
\cline{2-10} 
\\[-1.8ex] & Car/Bike (work) & Public (work) & Bicycle/Walk (work) & Car/Bike (shop) & Public (shop) & Bicycle/Walk (shop) & Car/Bike (leisure) & Public (leisure) & Bicycle/Walk (leisure) \\ 
\\[-1.8ex] & (1) & (2) & (3) & (4) & (5) & (6) & (7) & (8) & (9)\\ 
\hline \\[-1.8ex] 
 White only & 0.200$^{**}$ & $-$0.117 & $-$0.046 & 0.057 & $-$0.049 & 0.012 & 0.095 & 0.015 & $-$0.026 \\ 
  & (0.097) & (0.082) & (0.054) & (0.068) & (0.050) & (0.054) & (0.078) & (0.053) & (0.058) \\ 
  & & & & & & & & & \\ 
 Male & $-$0.060 & 0.038 & $-$0.026 & $-$0.102$^{*}$ & 0.047 & 0.027 & $-$0.228$^{***}$ & 0.073 & 0.110$^{**}$ \\ 
  & (0.089) & (0.074) & (0.049) & (0.059) & (0.044) & (0.047) & (0.067) & (0.046) & (0.050) \\ 
  & & & & & & & & & \\ 
 Children & 0.011 & $-$0.023 & 0.016 & $-$0.031 & 0.038 & $-$0.010 & $-$0.023 & 0.008 & 0.044 \\ 
  & (0.092) & (0.077) & (0.051) & (0.060) & (0.045) & (0.048) & (0.069) & (0.047) & (0.052) \\ 
  & & & & & & & & & \\ 
 No college & 0.072 & $-$0.026 & 0.025 & $-$0.019 & 0.049 & 0.010 & 0.038 & $-$0.004 & 0.008 \\ 
  & (0.107) & (0.090) & (0.059) & (0.066) & (0.049) & (0.052) & (0.076) & (0.052) & (0.056) \\ 
  & & & & & & & & & \\ 
 Retired & $-$0.023 & $-$0.040 & 0.032 & $-$0.033 & 0.061 & $-$0.035 & 0.105 & 0.012 & $-$0.101 \\ 
  & (0.197) & (0.165) & (0.109) & (0.111) & (0.082) & (0.088) & (0.125) & (0.085) & (0.093) \\ 
  & & & & & & & & & \\ 
 Student & $-$0.574$^{*}$ & $-$0.031 & 0.598$^{***}$ & $-$0.671$^{***}$ & 0.386$^{**}$ & 0.267 & $-$0.236 & 0.360$^{*}$ & $-$0.065 \\ 
  & (0.326) & (0.274) & (0.181) & (0.250) & (0.185) & (0.199) & (0.277) & (0.189) & (0.206) \\ 
  & & & & & & & & & \\ 
 Working & 0.034 & $-$0.006 & $-$0.036 & $-$0.091 & 0.032 & 0.038 & $-$0.029 & 0.090 & $-$0.051 \\ 
  & (0.176) & (0.147) & (0.097) & (0.107) & (0.079) & (0.085) & (0.120) & (0.082) & (0.089) \\ 
  & & & & & & & & & \\ 
 Income Q2 & 0.078 & 0.142 & $-$0.168$^{**}$ & 0.170$^{*}$ & $-$0.038 & $-$0.103 & 0.265$^{**}$ & $-$0.029 & $-$0.179$^{**}$ \\ 
  & (0.149) & (0.125) & (0.083) & (0.090) & (0.066) & (0.071) & (0.106) & (0.072) & (0.079) \\ 
  & & & & & & & & & \\ 
 Income Q3 & 0.154 & $-$0.065 & $-$0.071 & 0.216$^{**}$ & $-$0.082 & $-$0.118$^{*}$ & 0.260$^{***}$ & $-$0.026 & $-$0.183$^{**}$ \\ 
  & (0.139) & (0.117) & (0.077) & (0.084) & (0.062) & (0.067) & (0.098) & (0.067) & (0.073) \\ 
  & & & & & & & & & \\ 
 Income Q4 & 0.142 & 0.050 & $-$0.140$^{*}$ & 0.172$^{*}$ & $-$0.006 & $-$0.131$^{*}$ & 0.179$^{*}$ & $-$0.007 & $-$0.127$^{*}$ \\ 
  & (0.139) & (0.117) & (0.077) & (0.089) & (0.066) & (0.070) & (0.102) & (0.069) & (0.076) \\ 
  & & & & & & & & & \\ 
 30-49 & 0.070 & $-$0.283$^{*}$ & 0.191$^{*}$ & 0.037 & $-$0.218$^{**}$ & 0.161 & 0.114 & 0.016 & $-$0.229$^{*}$ \\ 
  & (0.187) & (0.157) & (0.103) & (0.145) & (0.107) & (0.115) & (0.160) & (0.109) & (0.119) \\ 
  & & & & & & & & & \\ 
 50-87 & 0.038 & $-$0.197 & 0.101 & 0.094 & $-$0.300$^{***}$ & 0.152 & 0.192 & $-$0.018 & $-$0.305$^{**}$ \\ 
  & (0.202) & (0.170) & (0.112) & (0.150) & (0.111) & (0.119) & (0.166) & (0.113) & (0.123) \\ 
  & & & & & & & & & \\ 
 Non voting & $-$0.043 & $-$0.225$^{*}$ & 0.172$^{**}$ & $-$0.282$^{***}$ & 0.045 & 0.173$^{**}$ & $-$0.145 & $-$0.090 & 0.152$^{*}$ \\ 
  & (0.138) & (0.115) & (0.076) & (0.095) & (0.070) & (0.075) & (0.110) & (0.075) & (0.081) \\ 
  & & & & & & & & & \\ 
 Other & 0.236 & $-$0.168 & $-$0.024 & 0.140 & $-$0.048 & $-$0.081 & 0.274$^{*}$ & $-$0.110 & $-$0.112 \\ 
  & (0.221) & (0.185) & (0.122) & (0.135) & (0.100) & (0.107) & (0.149) & (0.102) & (0.111) \\ 
  & & & & & & & & & \\ 
 Trump & $-$0.029 & $-$0.034 & 0.020 & $-$0.053 & 0.021 & 0.012 & $-$0.004 & $-$0.055 & 0.027 \\ 
  & (0.095) & (0.080) & (0.053) & (0.062) & (0.046) & (0.049) & (0.069) & (0.047) & (0.052) \\ 
  & & & & & & & & & \\ 
 transport\_available\_not & 0.049 & $-$0.126 & 0.072 & 0.093 & $-$0.056 & $-$0.035 & $-$0.019 & 0.007 & 0.014 \\ 
  & (0.097) & (0.081) & (0.053) & (0.059) & (0.044) & (0.047) & (0.067) & (0.046) & (0.050) \\ 
  & & & & & & & & & \\ 
 Constant & 0.473$^{**}$ & 0.507$^{**}$ & 0.029 & 0.737$^{***}$ & 0.303$^{**}$ & $-$0.010 & 0.503$^{***}$ & 0.006 & 0.455$^{***}$ \\ 
  & (0.234) & (0.196) & (0.129) & (0.176) & (0.130) & (0.140) & (0.193) & (0.132) & (0.143) \\ 
  & & & & & & & & & \\ 
\hline \\[-1.8ex] 
Mean & 0.779 & 0.139 & 0.066 & 0.819 & 0.08 & 0.09 & 0.773 & 0.074 & 0.102 \\ 
Observations & 118 & 118 & 118 & 184 & 184 & 184 & 174 & 174 & 174 \\ 
\hline 
\hline \\[-1.8ex] 
\textit{Note:}  & \multicolumn{9}{r}{$^{*}$p$<$0.1; $^{**}$p$<$0.05; $^{***}$p$<$0.01} \\ 
\end{tabular} 
}
	\end{center}
	{\footnotesize Note: The dependent variables are indicator variables equal to one if the respondent indicates she mainly uses this mode of transport for the activity in brackets. For instance, the \textit{Car/Bike (work)} variable equals one if the respondent mainly uses a car or a motorbike to go to work, school of university. \textit{Public} variables stand for ``Public Transports", \textit{Bicycle/Walk} stands for ``Walking or cycling", \textit{shop} for ``Grocery shopping" and \textit{leisure} for ``Leisure (excluding holidays)."
	See note under Table \ref{table heating} for a description of the covariates. \textit{PT not available} is an indicator variable equal to 1, if public transports are not available where the respondent lives.
	\newline *p$<$0.1; **p$<$0.05; ***p$<$0.01}	
\end{table}	
\end{landscape}

\clearpage
\subsection{Trust, perceptions of institution, inequality, and the future}

\begin{table}[h!]
	\caption{Trust in government and others}
	\begin{center}
		\scalebox{0.7}{
\begin{tabular}{@{\extracolsep{5pt}}lccc} 
\\[-1.8ex]\hline 
\hline \\[-1.8ex] 
 & \multicolumn{3}{c}{Do you trust…} \\ 
\cline{2-4} 
\\[-1.8ex] & most people & government to do what is right & government to spend revenue wisely \\ 
\\[-1.8ex] & (1) & (2) & (3)\\ 
\hline \\[-1.8ex] 
 White only & 0.037 & $-$0.120 & $-$0.003 \\ 
  & (0.096) & (0.078) & (0.058) \\ 
  & & & \\ 
 Male & 0.070 & 0.042 & 0.060 \\ 
  & (0.084) & (0.068) & (0.050) \\ 
  & & & \\ 
 Children & 0.088 & 0.131$^{*}$ & 0.160$^{***}$ \\ 
  & (0.086) & (0.070) & (0.051) \\ 
  & & & \\ 
 No college & $-$0.064 & 0.001 & $-$0.024 \\ 
  & (0.096) & (0.077) & (0.057) \\ 
  & & & \\ 
 Retired & 0.158 & $-$0.021 & 0.161$^{*}$ \\ 
  & (0.150) & (0.123) & (0.091) \\ 
  & & & \\ 
 Student & $-$0.498 & 0.031 & $-$0.247 \\ 
  & (0.402) & (0.295) & (0.217) \\ 
  & & & \\ 
 Working & 0.071 & 0.186 & 0.192$^{**}$ \\ 
  & (0.146) & (0.121) & (0.089) \\ 
  & & & \\ 
 Income Q2 & $-$0.052 & 0.015 & $-$0.117 \\ 
  & (0.126) & (0.104) & (0.076) \\ 
  & & & \\ 
 Income Q3 & 0.077 & $-$0.020 & $-$0.121$^{*}$ \\ 
  & (0.123) & (0.098) & (0.072) \\ 
  & & & \\ 
 Income Q4 & 0.111 & $-$0.018 & $-$0.020 \\ 
  & (0.127) & (0.104) & (0.077) \\ 
  & & & \\ 
 30-49 & $-$0.294 & $-$0.239 & $-$0.219$^{*}$ \\ 
  & (0.206) & (0.171) & (0.126) \\ 
  & & & \\ 
 50-87 & $-$0.552$^{**}$ & $-$0.561$^{***}$ & $-$0.489$^{***}$ \\ 
  & (0.212) & (0.176) & (0.130) \\ 
  & & & \\ 
 Non voting & 0.043 & $-$0.068 & 0.075 \\ 
  & (0.139) & (0.108) & (0.080) \\ 
  & & & \\ 
 Other & 0.332$^{*}$ & 0.051 & $-$0.088 \\ 
  & (0.198) & (0.160) & (0.118) \\ 
  & & & \\ 
 Trump & 0.067 & 0.038 & $-$0.022 \\ 
  & (0.087) & (0.073) & (0.054) \\ 
  & & & \\ 
 Constant & 0.638$^{**}$ & 0.672$^{***}$ & 0.309$^{**}$ \\ 
  & (0.246) & (0.203) & (0.150) \\ 
  & & & \\ 
\hline \\[-1.8ex] 
Mean & 0.489 & 0.338 & 0.154 \\ 
Observations & 176 & 191 & 191 \\ 
\hline 
\hline \\[-1.8ex] 
\textit{Note:}  & \multicolumn{3}{r}{$^{*}$p$<$0.1; $^{**}$p$<$0.05; $^{***}$p$<$0.01} \\ 
\end{tabular} 
}
	\end{center}
	{\footnotesize Note: The dependent variables are indicator variables. The \textit{most people} variable equals one if the respondent assigns a score greather than 5, on a scale from 0 to 10, to the question asking about trusting other people (0: ``One needs to be careful", 5: ``Most people can be trusted"). The \textit{government to do what is right} variable equals one if the respondent indicates trusting the U.S. government to do what is right ``Nearly all the time" or "Most of the time." The \textit{government to spend revenue wisely} variable equals one if the respondent indicates to ``fully agree" or ``somewhat agree" that authorities spend the revenue obtained from taxes and fees in a sensible way.
	See note under Table \ref{table heating} for a description of the covariates.
		\newline *p$<$0.1; **p$<$0.05; ***p$<$0.01}	
\end{table}	

\begin{table}[h!]
	\caption{Intervention, inequality and future}
	\begin{center}
		\scalebox{0.7}{
\begin{tabular}{@{\extracolsep{5pt}}lccc} 
\\[-1.8ex]\hline 
\hline \\[-1.8ex] 
\\[-1.8ex] & Active government & Inequality serious problem & World poorer or same \\ 
\\[-1.8ex] & (1) & (2) & (3)\\ 
\hline \\[-1.8ex] 
 White only & 0.045 & $-$0.022 & 0.146 \\ 
  & (0.096) & (0.087) & (0.095) \\ 
  & & & \\ 
 Male & 0.014 & 0.090 & $-$0.036 \\ 
  & (0.080) & (0.076) & (0.083) \\ 
  & & & \\ 
 Children & 0.068 & 0.118 & 0.027 \\ 
  & (0.082) & (0.077) & (0.084) \\ 
  & & & \\ 
 No college & $-$0.036 & 0.012 & $-$0.149 \\ 
  & (0.097) & (0.086) & (0.094) \\ 
  & & & \\ 
 Retired & 0.241 & $-$0.218 & $-$0.074 \\ 
  & (0.155) & (0.137) & (0.149) \\ 
  & & & \\ 
 Student & $-$0.433 & 0.130 & 0.554 \\ 
  & (0.349) & (0.326) & (0.355) \\ 
  & & & \\ 
 Working & 0.160 & $-$0.073 & $-$0.105 \\ 
  & (0.156) & (0.135) & (0.147) \\ 
  & & & \\ 
 Income Q2 & $-$0.122 & $-$0.011 & $-$0.039 \\ 
  & (0.125) & (0.115) & (0.126) \\ 
  & & & \\ 
 Income Q3 & $-$0.088 & 0.014 & $-$0.040 \\ 
  & (0.125) & (0.109) & (0.119) \\ 
  & & & \\ 
 Income Q4 & $-$0.055 & 0.075 & $-$0.110 \\ 
  & (0.129) & (0.116) & (0.127) \\ 
  & & & \\ 
 30-49 & $-$0.221 & 0.137 & $-$0.261 \\ 
  & (0.204) & (0.189) & (0.206) \\ 
  & & & \\ 
 50-87 & $-$0.477$^{**}$ & 0.222 & $-$0.096 \\ 
  & (0.213) & (0.195) & (0.212) \\ 
  & & & \\ 
 Non voting & $-$0.107 & $-$0.004 & $-$0.083 \\ 
  & (0.131) & (0.121) & (0.132) \\ 
  & & & \\ 
 Other & $-$0.172 & $-$0.043 & $-$0.070 \\ 
  & (0.195) & (0.177) & (0.193) \\ 
  & & & \\ 
 Trump & $-$0.231$^{***}$ & 0.328$^{***}$ & $-$0.163$^{*}$ \\ 
  & (0.086) & (0.081) & (0.088) \\ 
  & & & \\ 
 Climate treatment only & $-$0.141 & 0.081 & $-$0.035 \\ 
  & (0.121) & (0.110) & (0.120) \\ 
  & & & \\ 
 No treatment & $-$0.025 & $-$0.008 & 0.049 \\ 
  & (0.117) & (0.108) & (0.118) \\ 
  & & & \\ 
 Policy treatment only & $-$0.070 & 0.089 & 0.050 \\ 
  & (0.106) & (0.098) & (0.107) \\ 
  & & & \\ 
 Constant & 0.787$^{***}$ & $-$0.021 & 0.720$^{***}$ \\ 
  & (0.267) & (0.236) & (0.258) \\ 
  & & & \\ 
\hline \\[-1.8ex] 
Mean & 0.436 & 0.344 & 0.467 \\ 
Observations & 179 & 191 & 191 \\ 
\hline 
\hline \\[-1.8ex] 
\textit{Note:}  & \multicolumn{3}{r}{$^{*}$p$<$0.1; $^{**}$p$<$0.05; $^{***}$p$<$0.01} \\ 
\end{tabular} 
}
	\end{center}
	{\footnotesize Note: The dependent variables are indicator variables. The \textit{Active government} variable equals one if the respondent assigns a score greather than 3, on a scale from 1 to 5 asking about the purpose of government (1: ``Government should focus on most basic functions", 5: "Government should play an active role"). The \textit{Inequality serious problem} equals one if the respondent indicates that in the U.S. inequality is ``A serious problem" or ``A very serious problem." The \textit{World poorer or same} variable equals one if the respondent indicates that in 100y. the world will be ``About as rich as now on average" or ``Poorer."
	See note under Table \ref{table heating} for a description of the covariates.
	\newline *p$<$0.1; **p$<$0.05; ***p$<$0.01}
\end{table}	

\begin{table}[h!]
	\caption{Environmental views}
	\begin{center}
		\scalebox{0.7}{
\begin{tabular}{@{\extracolsep{5pt}}lcccc} 
\\[-1.8ex]\hline 
\hline \\[-1.8ex] 
 & \multicolumn{4}{c}{Environmental views} \\ 
\cline{2-5} 
\\[-1.8ex] & Collapse & Not a problem, progress & Need sustainable society & Other goals \\ 
\\[-1.8ex] & (1) & (2) & (3) & (4)\\ 
\hline \\[-1.8ex] 
 White only & $-$0.023 & $-$0.022 & 0.051 & 0.051 \\ 
  & (0.054) & (0.073) & (0.090) & (0.072) \\ 
  & & & & \\ 
 Male & $-$0.037 & 0.077 & 0.036 & $-$0.024 \\ 
  & (0.047) & (0.063) & (0.078) & (0.063) \\ 
  & & & & \\ 
 Children & $-$0.079 & 0.143$^{**}$ & 0.034 & 0.050 \\ 
  & (0.048) & (0.064) & (0.079) & (0.064) \\ 
  & & & & \\ 
 No college & $-$0.102$^{*}$ & 0.067 & $-$0.192$^{**}$ & 0.141$^{**}$ \\ 
  & (0.054) & (0.072) & (0.089) & (0.071) \\ 
  & & & & \\ 
 Retired & $-$0.044 & $-$0.091 & $-$0.085 & 0.126 \\ 
  & (0.085) & (0.114) & (0.140) & (0.112) \\ 
  & & & & \\ 
 Student & 0.153 & 0.146 & $-$0.157 & 0.122 \\ 
  & (0.202) & (0.271) & (0.334) & (0.268) \\ 
  & & & & \\ 
 Working & $-$0.020 & 0.033 & $-$0.044 & 0.053 \\ 
  & (0.084) & (0.112) & (0.138) & (0.111) \\ 
  & & & & \\ 
 Income Q2 & 0.094 & 0.031 & $-$0.112 & $-$0.034 \\ 
  & (0.071) & (0.096) & (0.118) & (0.095) \\ 
  & & & & \\ 
 Income Q3 & 0.028 & $-$0.018 & $-$0.119 & 0.102 \\ 
  & (0.068) & (0.091) & (0.112) & (0.090) \\ 
  & & & & \\ 
 Income Q4 & $-$0.026 & $-$0.002 & $-$0.095 & 0.114 \\ 
  & (0.072) & (0.097) & (0.119) & (0.096) \\ 
  & & & & \\ 
 30-49 & 0.121 & 0.096 & $-$0.071 & $-$0.283$^{*}$ \\ 
  & (0.117) & (0.157) & (0.193) & (0.155) \\ 
  & & & & \\ 
 50-87 & 0.105 & 0.105 & 0.196 & $-$0.426$^{***}$ \\ 
  & (0.121) & (0.162) & (0.199) & (0.160) \\ 
  & & & & \\ 
 Non voting & 0.018 & 0.135 & $-$0.270$^{**}$ & $-$0.007 \\ 
  & (0.075) & (0.101) & (0.124) & (0.100) \\ 
  & & & & \\ 
 Other & $-$0.048 & 0.102 & $-$0.027 & $-$0.115 \\ 
  & (0.110) & (0.147) & (0.181) & (0.146) \\ 
  & & & & \\ 
 Trump & $-$0.040 & 0.176$^{***}$ & $-$0.345$^{***}$ & 0.111$^{*}$ \\ 
  & (0.050) & (0.067) & (0.083) & (0.067) \\ 
  & & & & \\ 
 Climate treatment only & 0.034 & 0.009 & 0.028 & $-$0.094 \\ 
  & (0.068) & (0.091) & (0.112) & (0.090) \\ 
  & & & & \\ 
 No treatment & $-$0.030 & 0.097 & 0.067 & $-$0.096 \\ 
  & (0.067) & (0.090) & (0.111) & (0.089) \\ 
  & & & & \\ 
 Policy treatment only & 0.037 & 0.119 & 0.013 & $-$0.118 \\ 
  & (0.061) & (0.082) & (0.101) & (0.081) \\ 
  & & & & \\ 
 Constant & 0.116 & $-$0.191 & 0.547$^{**}$ & 0.360$^{*}$ \\ 
  & (0.147) & (0.197) & (0.242) & (0.195) \\ 
  & & & & \\ 
\hline \\[-1.8ex] 
Mean &  &  &  &  \\ 
Observations & 191 & 191 & 191 & 191 \\ 
\hline 
\hline \\[-1.8ex] 
\textit{Note:}  & \multicolumn{4}{r}{$^{*}$p$<$0.1; $^{**}$p$<$0.05; $^{***}$p$<$0.01} \\ 
\end{tabular} 
}
	\end{center}
	{\footnotesize Note: The variables are indicator variables equal to one if the respondent indicates that this statement is the closest to her view on environmental issues. The \textit{Collapse} variable corresponds to the statement ``Our civilization will eventually collapse, it is useless to try making society more sustainable", \textit{Not a problem, progress} to the statement ``Our civilization will develop so much that environmental issues will not be a problem in the distant future", \textit{Need, sustainable society} to the statement ``We should make our society as sustainable as possible to avoir irreversible damages," and \textit{Other goals} to the statement ``I believe we have more important goals than sustainability."}
\end{table}	

\clearpage
\subsection{Climate change (attitudes and risks)}


\begin{table}[h!]
	\caption{Climate change existence}
	\begin{center}
		\scalebox{0.7}{
\begin{tabular}{@{\extracolsep{5pt}}lccc} 
\\[-1.8ex]\hline 
\hline \\[-1.8ex] 
 & \multicolumn{3}{c}{Climate change is…} \\ 
\cline{2-4} 
\\[-1.8ex] & not a reality & mainly due to natural climate variability & mainly due to human activity \\ 
\\[-1.8ex] & (1) & (2) & (3)\\ 
\hline \\[-1.8ex] 
 White only & $-$0.102$^{*}$ & 0.086 & $-$0.018 \\ 
  & (0.056) & (0.085) & (0.090) \\ 
  & & & \\ 
 Male & $-$0.054 & 0.070 & $-$0.003 \\ 
  & (0.049) & (0.074) & (0.078) \\ 
  & & & \\ 
 Children & $-$0.059 & 0.116 & $-$0.021 \\ 
  & (0.050) & (0.075) & (0.079) \\ 
  & & & \\ 
 No college & 0.020 & 0.024 & 0.017 \\ 
  & (0.056) & (0.084) & (0.089) \\ 
  & & & \\ 
 Retired & 0.082 & 0.039 & $-$0.043 \\ 
  & (0.088) & (0.132) & (0.140) \\ 
  & & & \\ 
 Student & $-$0.030 & 0.083 & 0.091 \\ 
  & (0.210) & (0.315) & (0.334) \\ 
  & & & \\ 
 Working & 0.041 & 0.169 & $-$0.182 \\ 
  & (0.087) & (0.131) & (0.138) \\ 
  & & & \\ 
 Income Q2 & $-$0.023 & 0.050 & $-$0.059 \\ 
  & (0.074) & (0.112) & (0.118) \\ 
  & & & \\ 
 Income Q3 & $-$0.012 & 0.028 & 0.061 \\ 
  & (0.070) & (0.106) & (0.112) \\ 
  & & & \\ 
 Income Q4 & 0.089 & 0.017 & 0.048 \\ 
  & (0.075) & (0.112) & (0.119) \\ 
  & & & \\ 
 30-49 & 0.249$^{**}$ & $-$0.317$^{*}$ & $-$0.100 \\ 
  & (0.122) & (0.183) & (0.193) \\ 
  & & & \\ 
 50-87 & 0.074 & $-$0.155 & $-$0.103 \\ 
  & (0.125) & (0.188) & (0.199) \\ 
  & & & \\ 
 Non voting & 0.017 & 0.139 & $-$0.330$^{***}$ \\ 
  & (0.078) & (0.118) & (0.124) \\ 
  & & & \\ 
 Other & 0.034 & $-$0.141 & 0.049 \\ 
  & (0.114) & (0.171) & (0.181) \\ 
  & & & \\ 
 Trump & 0.131$^{**}$ & 0.269$^{***}$ & $-$0.418$^{***}$ \\ 
  & (0.052) & (0.078) & (0.083) \\ 
  & & & \\ 
 Both treatments & $-$0.070 & 0.041 & $-$0.009 \\ 
  & (0.070) & (0.105) & (0.111) \\ 
  & & & \\ 
 Climate treatment only & 0.076 & 0.085 & $-$0.084 \\ 
  & (0.066) & (0.099) & (0.104) \\ 
  & & & \\ 
 Policy treatment only & $-$0.099 & 0.098 & 0.003 \\ 
  & (0.060) & (0.091) & (0.096) \\ 
  & & & \\ 
 Constant & 0.041 & 0.018 & 0.912$^{***}$ \\ 
  & (0.150) & (0.226) & (0.239) \\ 
  & & & \\ 
\hline \\[-1.8ex] 
Mean & 0.103 & 0.303 & 0.503 \\ 
Observations & 191 & 191 & 191 \\ 
\hline 
\hline \\[-1.8ex] 
\textit{Note:}  & \multicolumn{3}{r}{$^{*}$p$<$0.1; $^{**}$p$<$0.05; $^{***}$p$<$0.01} \\ 
\end{tabular} 
}
	\end{center}
	{\footnotesize Note: The dependent variables are indicator variables equal to one if the statement corresponds to the respondent's belief about climate change. For instance, the variable \textit{not a reality} equals one if the respondent thinks that climate change is not a reality. See note under Table \ref{table heating} for a description of the covariates.
	\newline *p$<$0.1; **p$<$0.05; ***p$<$0.01}
\end{table}		

\begin{table}[h!]
	\caption{Halving GHG}
	\begin{center}
		\scalebox{0.7}{
\begin{tabular}{@{\extracolsep{5pt}}lcccc} 
\\[-1.8ex]\hline 
\hline \\[-1.8ex] 
 & \multicolumn{4}{c}{Halving global GHG emissions} \\ 
\cline{2-5} 
\\[-1.8ex] & has no impact on temperatures & will decrease temperatures & will stabilize temperatures & will increase temperatures, just more slowly \\ 
\\[-1.8ex] & (1) & (2) & (3) & (4)\\ 
\hline \\[-1.8ex] 
 White only & $-$0.039 & $-$0.024 & $-$0.066 & 0.137 \\ 
  & (0.052) & (0.052) & (0.067) & (0.087) \\ 
  & & & & \\ 
 Male & 0.040 & $-$0.107$^{**}$ & 0.061 & 0.160$^{**}$ \\ 
  & (0.045) & (0.045) & (0.059) & (0.076) \\ 
  & & & & \\ 
 Children & $-$0.007 & 0.057 & 0.076 & $-$0.134$^{*}$ \\ 
  & (0.046) & (0.046) & (0.060) & (0.078) \\ 
  & & & & \\ 
 No college & 0.095$^{*}$ & 0.004 & $-$0.013 & $-$0.086 \\ 
  & (0.051) & (0.051) & (0.067) & (0.086) \\ 
  & & & & \\ 
 Retired & $-$0.090 & $-$0.136$^{*}$ & 0.184$^{*}$ & 0.026 \\ 
  & (0.082) & (0.082) & (0.106) & (0.137) \\ 
  & & & & \\ 
 Student & 0.208 & $-$0.066 & 0.126 & $-$0.215 \\ 
  & (0.195) & (0.196) & (0.254) & (0.327) \\ 
  & & & & \\ 
 Working & 0.004 & $-$0.141$^{*}$ & 0.097 & $-$0.070 \\ 
  & (0.080) & (0.080) & (0.104) & (0.134) \\ 
  & & & & \\ 
 Income Q2 & 0.121$^{*}$ & 0.049 & $-$0.024 & $-$0.115 \\ 
  & (0.069) & (0.069) & (0.089) & (0.115) \\ 
  & & & & \\ 
 Income Q3 & 0.079 & 0.059 & $-$0.042 & $-$0.054 \\ 
  & (0.065) & (0.065) & (0.084) & (0.109) \\ 
  & & & & \\ 
 Income Q4 & 0.153$^{**}$ & 0.012 & $-$0.051 & $-$0.002 \\ 
  & (0.069) & (0.069) & (0.090) & (0.116) \\ 
  & & & & \\ 
 30-49 & $-$0.032 & $-$0.153 & 0.124 & 0.023 \\ 
  & (0.113) & (0.114) & (0.147) & (0.190) \\ 
  & & & & \\ 
 50-87 & 0.034 & $-$0.160 & $-$0.088 & 0.063 \\ 
  & (0.117) & (0.117) & (0.152) & (0.195) \\ 
  & & & & \\ 
 Non voting & $-$0.025 & 0.150$^{**}$ & 0.198$^{**}$ & $-$0.389$^{***}$ \\ 
  & (0.072) & (0.072) & (0.093) & (0.120) \\ 
  & & & & \\ 
 Other & $-$0.056 & 0.114 & 0.181 & $-$0.360$^{**}$ \\ 
  & (0.106) & (0.106) & (0.138) & (0.178) \\ 
  & & & & \\ 
 Trump & 0.191$^{***}$ & 0.029 & 0.044 & $-$0.346$^{***}$ \\ 
  & (0.048) & (0.048) & (0.063) & (0.081) \\ 
  & & & & \\ 
 Constant & $-$0.056 & 0.351$^{**}$ & 0.001 & 0.578$^{**}$ \\ 
  & (0.135) & (0.135) & (0.175) & (0.226) \\ 
  & & & & \\ 
\hline \\[-1.8ex] 
Mean & 0.097 & 0.092 & 0.154 & 0.467 \\ 
Observations & 191 & 191 & 191 & 191 \\ 
\hline 
\hline \\[-1.8ex] 
\textit{Note:}  & \multicolumn{4}{r}{$^{*}$p$<$0.1; $^{**}$p$<$0.05; $^{***}$p$<$0.01} \\ 
\end{tabular} 
}
	\end{center}
	{\footnotesize Note: The dependent variables are indicator variables equal to one if the statement corresponds to the respondent's belief about the effects of halving global GHG emissions. For instance, the variable \textit{has no impact on temperatures} equals one if the respondent thinks that chalving global GHG emissions has no impact on temperatures. See note under Table \ref{table heating} for a description of the covariates.
	\newline *p$<$0.1; **p$<$0.05; ***p$<$0.01}
\end{table}

\begin{table}[h!]
	\caption{Comparisons of GHG emissions}
	\begin{center}
		\scalebox{0.7}{
\begin{tabular}{@{\extracolsep{5pt}}lccc} 
\\[-1.8ex]\hline 
\hline \\[-1.8ex] 
 & \multicolumn{3}{c}{Does this activity emits fare more GHG than this other one?} \\ 
\cline{2-4} 
\\[-1.8ex] & eating beef vs. two servings of pasta & eletricity produced by nuclear power vs. wind turbines & commuting by car vs. food waste \\ 
\\[-1.8ex] & (1) & (2) & (3)\\ 
\hline \\[-1.8ex] 
 race: White only & 0.079 & 0.027 & 0.075 \\ 
  & (0.082) & (0.087) & (0.087) \\ 
  & & & \\ 
 Male & 0.099 & $-$0.151$^{**}$ & $-$0.008 \\ 
  & (0.071) & (0.075) & (0.075) \\ 
  & & & \\ 
 Children & 0.078 & 0.159$^{**}$ & $-$0.226$^{***}$ \\ 
  & (0.073) & (0.077) & (0.077) \\ 
  & & & \\ 
 No college & $-$0.042 & $-$0.099 & 0.097 \\ 
  & (0.081) & (0.086) & (0.086) \\ 
  & & & \\ 
 status: Retired & $-$0.049 & $-$0.071 & $-$0.039 \\ 
  & (0.130) & (0.137) & (0.137) \\ 
  & & & \\ 
 status: Student & $-$0.023 & 0.011 & $-$0.249 \\ 
  & (0.310) & (0.328) & (0.329) \\ 
  & & & \\ 
 status: Working & 0.055 & $-$0.192 & $-$0.108 \\ 
  & (0.127) & (0.135) & (0.135) \\ 
  & & & \\ 
 Income Q2 & 0.081 & $-$0.120 & $-$0.023 \\ 
  & (0.109) & (0.116) & (0.116) \\ 
  & & & \\ 
 Income Q3 & 0.018 & $-$0.105 & 0.150 \\ 
  & (0.103) & (0.109) & (0.109) \\ 
  & & & \\ 
 Income Q4 & 0.191$^{*}$ & $-$0.176 & $-$0.077 \\ 
  & (0.110) & (0.116) & (0.116) \\ 
  & & & \\ 
 age: 30-49 & $-$0.080 & $-$0.156 & 0.367$^{*}$ \\ 
  & (0.179) & (0.189) & (0.189) \\ 
  & & & \\ 
 age: 50-87 & $-$0.191 & $-$0.479$^{**}$ & 0.510$^{***}$ \\ 
  & (0.182) & (0.192) & (0.192) \\ 
  & & & \\ 
 vote: Biden & 0.084 & 0.113 & $-$0.043 \\ 
  & (0.100) & (0.105) & (0.105) \\ 
  & & & \\ 
 vote: Trump & $-$0.097 & 0.067 & $-$0.030 \\ 
  & (0.108) & (0.114) & (0.114) \\ 
  & & & \\ 
 Constant & 0.211 & 0.903$^{***}$ & 0.181 \\ 
  & (0.201) & (0.212) & (0.212) \\ 
  & & & \\ 
\hline \\[-1.8ex] 
Mean & 0.318 & 0.39 & 0.477 \\ 
Observations & 191 & 191 & 191 \\ 
\hline 
\hline \\[-1.8ex] 
\end{tabular} 
}
	\end{center}
	{\footnotesize Note: The variables are indicator variables equal to one if the respondent thinks the statement is true. For instance, the \textit{eating beef vs. two servings of pasta} variable means that the respondent thinks eating one beef steak emits far more GHG than eating two serving of pasta. }
\end{table}		

\begin{landscape}
	\begin{table}[h!]
		\caption{Responsible party for CC}
		\begin{center}
			\scalebox{0.6}{
\begin{tabular}{@{\extracolsep{5pt}}lcccccccc} 
\\[-1.8ex]\hline 
\hline \\[-1.8ex] 
 & \multicolumn{8}{c}{Which of the following is predominantly responsible for CC?} \\ 
\cline{2-9} 
\\[-1.8ex] & Each of us & The rich & Governments & Companies & Previous generations & Some foreign countries & Natural causes & Climate change is not a reality \\ 
\\[-1.8ex] & (1) & (2) & (3) & (4) & (5) & (6) & (7) & (8)\\ 
\hline \\[-1.8ex] 
 White only & $-$0.029 & 0.012 & $-$0.145$^{*}$ & $-$0.060 & 0.002 & 0.044 & 0.016 & $-$0.027 \\ 
  & (0.092) & (0.071) & (0.081) & (0.093) & (0.073) & (0.087) & (0.091) & (0.049) \\ 
  & & & & & & & & \\ 
 Male & 0.120 & 0.071 & 0.118$^{*}$ & 0.050 & $-$0.034 & 0.132$^{*}$ & 0.034 & $-$0.024 \\ 
  & (0.080) & (0.062) & (0.071) & (0.081) & (0.064) & (0.076) & (0.079) & (0.043) \\ 
  & & & & & & & & \\ 
 Children & $-$0.112 & 0.012 & 0.084 & 0.029 & $-$0.039 & 0.033 & 0.009 & $-$0.003 \\ 
  & (0.081) & (0.063) & (0.072) & (0.082) & (0.065) & (0.077) & (0.081) & (0.044) \\ 
  & & & & & & & & \\ 
 No college & $-$0.108 & $-$0.074 & $-$0.001 & $-$0.069 & $-$0.101 & $-$0.094 & $-$0.047 & 0.082$^{*}$ \\ 
  & (0.091) & (0.071) & (0.080) & (0.092) & (0.073) & (0.087) & (0.090) & (0.049) \\ 
  & & & & & & & & \\ 
 Retired & 0.167 & 0.019 & 0.131 & 0.094 & 0.050 & $-$0.034 & 0.013 & 0.004 \\ 
  & (0.144) & (0.112) & (0.127) & (0.145) & (0.115) & (0.137) & (0.142) & (0.077) \\ 
  & & & & & & & & \\ 
 Student & $-$0.284 & $-$0.229 & $-$0.184 & $-$0.146 & 0.651$^{**}$ & $-$0.336 & 0.181 & $-$0.229 \\ 
  & (0.342) & (0.266) & (0.303) & (0.345) & (0.273) & (0.326) & (0.339) & (0.183) \\ 
  & & & & & & & & \\ 
 Working & 0.170 & 0.013 & 0.103 & $-$0.035 & $-$0.013 & $-$0.112 & 0.040 & $-$0.016 \\ 
  & (0.142) & (0.110) & (0.126) & (0.143) & (0.113) & (0.135) & (0.140) & (0.076) \\ 
  & & & & & & & & \\ 
 Income Q2 & $-$0.005 & 0.022 & $-$0.112 & $-$0.086 & $-$0.216$^{**}$ & $-$0.010 & $-$0.124 & 0.018 \\ 
  & (0.121) & (0.094) & (0.107) & (0.122) & (0.097) & (0.115) & (0.120) & (0.065) \\ 
  & & & & & & & & \\ 
 Income Q3 & $-$0.035 & $-$0.087 & $-$0.169$^{*}$ & $-$0.033 & $-$0.215$^{**}$ & 0.045 & $-$0.139 & $-$0.035 \\ 
  & (0.115) & (0.089) & (0.102) & (0.116) & (0.092) & (0.109) & (0.114) & (0.062) \\ 
  & & & & & & & & \\ 
 Income Q4 & $-$0.069 & $-$0.037 & $-$0.108 & 0.040 & $-$0.203$^{**}$ & 0.138 & $-$0.075 & $-$0.001 \\ 
  & (0.122) & (0.095) & (0.108) & (0.123) & (0.097) & (0.116) & (0.121) & (0.065) \\ 
  & & & & & & & & \\ 
 30-49 & $-$0.006 & 0.160 & 0.191 & 0.215 & 0.118 & $-$0.214 & $-$0.199 & $-$0.012 \\ 
  & (0.198) & (0.154) & (0.175) & (0.200) & (0.158) & (0.189) & (0.196) & (0.106) \\ 
  & & & & & & & & \\ 
 50-87 & 0.120 & 0.031 & 0.174 & 0.219 & 0.023 & $-$0.167 & $-$0.107 & $-$0.049 \\ 
  & (0.204) & (0.159) & (0.181) & (0.206) & (0.163) & (0.195) & (0.202) & (0.109) \\ 
  & & & & & & & & \\ 
 Non voting & $-$0.084 & 0.043 & 0.003 & $-$0.161 & $-$0.111 & 0.050 & $-$0.179 & 0.110 \\ 
  & (0.128) & (0.099) & (0.113) & (0.129) & (0.102) & (0.121) & (0.126) & (0.068) \\ 
  & & & & & & & & \\ 
 Other & 0.274 & $-$0.196 & $-$0.399$^{**}$ & 0.326$^{*}$ & $-$0.068 & 0.286 & 0.142 & $-$0.012 \\ 
  & (0.186) & (0.145) & (0.164) & (0.188) & (0.148) & (0.177) & (0.184) & (0.100) \\ 
  & & & & & & & & \\ 
 Trump & $-$0.259$^{***}$ & $-$0.135$^{**}$ & $-$0.202$^{***}$ & $-$0.178$^{**}$ & $-$0.086 & 0.006 & 0.317$^{***}$ & 0.127$^{***}$ \\ 
  & (0.085) & (0.066) & (0.075) & (0.086) & (0.068) & (0.081) & (0.084) & (0.045) \\ 
  & & & & & & & & \\ 
 Both treatments & $-$0.110 & $-$0.116 & $-$0.084 & 0.028 & $-$0.176$^{*}$ & $-$0.069 & 0.058 & $-$0.006 \\ 
  & (0.114) & (0.088) & (0.100) & (0.115) & (0.091) & (0.108) & (0.112) & (0.061) \\ 
  & & & & & & & & \\ 
 Climate treatment only & $-$0.075 & $-$0.040 & $-$0.077 & $-$0.092 & $-$0.102 & $-$0.128 & $-$0.080 & 0.056 \\ 
  & (0.107) & (0.083) & (0.095) & (0.108) & (0.085) & (0.102) & (0.106) & (0.057) \\ 
  & & & & & & & & \\ 
 Policy treatment only & $-$0.137 & $-$0.082 & $-$0.107 & $-$0.081 & $-$0.163$^{**}$ & $-$0.048 & 0.063 & 0.034 \\ 
  & (0.098) & (0.076) & (0.087) & (0.099) & (0.078) & (0.094) & (0.097) & (0.053) \\ 
  & & & & & & & & \\ 
 Constant & 0.493$^{**}$ & 0.190 & 0.222 & 0.355 & 0.512$^{***}$ & 0.422$^{*}$ & 0.453$^{*}$ & 0.063 \\ 
  & (0.245) & (0.191) & (0.217) & (0.248) & (0.196) & (0.234) & (0.243) & (0.131) \\ 
  & & & & & & & & \\ 
\hline \\[-1.8ex] 
Mean & 0.472 & 0.169 & 0.267 & 0.441 & 0.185 & 0.308 & 0.395 & 0.072 \\ 
Observations & 191 & 191 & 191 & 191 & 191 & 191 & 191 & 191 \\ 
\hline 
\hline \\[-1.8ex] 
\textit{Note:}  & \multicolumn{8}{r}{$^{*}$p$<$0.1; $^{**}$p$<$0.05; $^{***}$p$<$0.01} \\ 
\end{tabular} 
}
		\end{center}
	{\footnotesize Note: The dependent variables are indicator variables equal to one if the respondent thinks this category is predominantly responsible for climate change. For instance, \textit{Each of us} means that the respondent thinks that each of us are predominantly responsible for climate change. See note under Table \ref{table heating} for a description of the covariates.
	\newline *p$<$0.1; **p$<$0.05; ***p$<$0.01}
	\end{table}		
\end{landscape}

\begin{landscape}
	\begin{table}[h!]
		\caption{Possible to halt CC}
		\begin{center}
			\scalebox{0.6}{
\begin{tabular}{@{\extracolsep{5pt}}lccccc} 
\\[-1.8ex]\hline 
\hline \\[-1.8ex] 
 & \multicolumn{5}{c}{Can humanity halt CC?} \\ 
\cline{2-6} 
\\[-1.8ex] & Human have no noticeable influence & Better live with CC than try to halt it & Should stop emmissions, but not going to happen & Ambitious policies and awareness will succeed & Technologies and habits will suffice \\ 
\\[-1.8ex] & (1) & (2) & (3) & (4) & (5)\\ 
\hline \\[-1.8ex] 
 White only & $-$0.024 & 0.065 & 0.010 & 0.034 & $-$0.085 \\ 
  & (0.070) & (0.074) & (0.088) & (0.095) & (0.071) \\ 
  & & & & & \\ 
 Male & 0.152$^{**}$ & $-$0.140$^{**}$ & $-$0.040 & 0.005 & 0.023 \\ 
  & (0.061) & (0.065) & (0.077) & (0.084) & (0.062) \\ 
  & & & & & \\ 
 Children & 0.012 & 0.043 & 0.019 & $-$0.141$^{*}$ & 0.068 \\ 
  & (0.062) & (0.066) & (0.079) & (0.085) & (0.063) \\ 
  & & & & & \\ 
 No college & 0.141$^{**}$ & 0.088 & $-$0.135 & $-$0.035 & $-$0.059 \\ 
  & (0.070) & (0.074) & (0.088) & (0.095) & (0.071) \\ 
  & & & & & \\ 
 Retired & $-$0.196$^{*}$ & 0.074 & $-$0.125 & 0.220 & 0.027 \\ 
  & (0.111) & (0.117) & (0.139) & (0.151) & (0.112) \\ 
  & & & & & \\ 
 Student & $-$0.455$^{*}$ & 0.001 & 0.232 & 0.278 & $-$0.057 \\ 
  & (0.247) & (0.262) & (0.311) & (0.336) & (0.251) \\ 
  & & & & & \\ 
 Working & $-$0.148 & 0.171 & $-$0.251$^{*}$ & 0.179 & 0.049 \\ 
  & (0.109) & (0.115) & (0.137) & (0.148) & (0.110) \\ 
  & & & & & \\ 
 Income Q2 & $-$0.017 & 0.053 & 0.039 & $-$0.080 & 0.004 \\ 
  & (0.098) & (0.104) & (0.124) & (0.134) & (0.100) \\ 
  & & & & & \\ 
 Income Q3 & 0.001 & $-$0.046 & 0.017 & 0.060 & $-$0.031 \\ 
  & (0.090) & (0.095) & (0.113) & (0.122) & (0.091) \\ 
  & & & & & \\ 
 Income Q4 & $-$0.082 & $-$0.058 & 0.059 & $-$0.070 & 0.151 \\ 
  & (0.093) & (0.099) & (0.117) & (0.127) & (0.095) \\ 
  & & & & & \\ 
 30-49 & $-$0.215 & 0.230 & $-$0.220 & 0.217 & $-$0.013 \\ 
  & (0.143) & (0.151) & (0.179) & (0.194) & (0.145) \\ 
  & & & & & \\ 
 50-87 & $-$0.222 & 0.102 & $-$0.047 & 0.101 & 0.066 \\ 
  & (0.149) & (0.158) & (0.187) & (0.203) & (0.151) \\ 
  & & & & & \\ 
 Non voting & 0.022 & $-$0.027 & 0.047 & $-$0.072 & 0.031 \\ 
  & (0.102) & (0.108) & (0.128) & (0.139) & (0.104) \\ 
  & & & & & \\ 
 Other & $-$0.133 & $-$0.075 & 0.187 & $-$0.004 & 0.024 \\ 
  & (0.150) & (0.159) & (0.188) & (0.204) & (0.152) \\ 
  & & & & & \\ 
 Trump & 0.263$^{***}$ & 0.064 & $-$0.091 & $-$0.238$^{***}$ & 0.002 \\ 
  & (0.062) & (0.066) & (0.078) & (0.085) & (0.063) \\ 
  & & & & & \\ 
 Constant & 0.331$^{*}$ & $-$0.093 & 0.548$^{**}$ & 0.177 & 0.037 \\ 
  & (0.173) & (0.183) & (0.218) & (0.236) & (0.175) \\ 
  & & & & & \\ 
\hline \\[-1.8ex] 
Mean & 0.156 & 0.156 & 0.246 & 0.311 & 0.132 \\ 
Observations & 165 & 165 & 165 & 165 & 165 \\ 
\hline 
\hline \\[-1.8ex] 
\textit{Note:}  & \multicolumn{5}{r}{$^{*}$p$<$0.1; $^{**}$p$<$0.05; $^{***}$p$<$0.01} \\ 
\end{tabular} 
}
		\end{center}
	{\footnotesize Note: The dependent variables are indicator variables equal to one if the respondent thinks the statement is true. For instance, the \textit{Human have no noticeable influence} variable equals one if the respondent thinks humans have no noticeable influence on the climate. See note under Table \ref{table heating} for a description of the covariates.
	\newline *p$<$0.1; **p$<$0.05; ***p$<$0.01}
	\end{table}		
\end{landscape}

\begin{table}[h!]
	\caption{Talks often about CC}
	\begin{center}
		\scalebox{0.8}{
\begin{tabular}{@{\extracolsep{5pt}}lccc} 
\\[-1.8ex]\hline 
\hline \\[-1.8ex] 
 & \multicolumn{3}{c}{How often do you talk about CC?} \\ 
\cline{2-4} 
\\[-1.8ex] & Never & Yearly & Monthly \\ 
\\[-1.8ex] & (1) & (2) & (3)\\ 
\hline \\[-1.8ex] 
 White only & 0.153$^{*}$ & $-$0.149$^{*}$ & 0.147$^{**}$ \\ 
  & (0.079) & (0.076) & (0.070) \\ 
  & & & \\ 
 Male & 0.018 & $-$0.018 & 0.028 \\ 
  & (0.069) & (0.066) & (0.061) \\ 
  & & & \\ 
 Children & $-$0.130$^{*}$ & 0.037 & 0.110$^{*}$ \\ 
  & (0.071) & (0.068) & (0.063) \\ 
  & & & \\ 
 No college & 0.049 & $-$0.134$^{*}$ & 0.009 \\ 
  & (0.078) & (0.075) & (0.069) \\ 
  & & & \\ 
 Retired & $-$0.109 & 0.046 & 0.040 \\ 
  & (0.125) & (0.119) & (0.110) \\ 
  & & & \\ 
 Student & 0.442 & $-$0.477$^{*}$ & $-$0.177 \\ 
  & (0.299) & (0.286) & (0.264) \\ 
  & & & \\ 
 Working & $-$0.153 & $-$0.044 & 0.159 \\ 
  & (0.123) & (0.117) & (0.108) \\ 
  & & & \\ 
 Income Q2 & $-$0.091 & 0.154 & $-$0.049 \\ 
  & (0.105) & (0.101) & (0.093) \\ 
  & & & \\ 
 Income Q3 & $-$0.020 & 0.149 & $-$0.068 \\ 
  & (0.099) & (0.095) & (0.087) \\ 
  & & & \\ 
 Income Q4 & $-$0.044 & 0.083 & 0.051 \\ 
  & (0.106) & (0.101) & (0.093) \\ 
  & & & \\ 
 30-49 & $-$0.114 & $-$0.151 & $-$0.025 \\ 
  & (0.173) & (0.165) & (0.153) \\ 
  & & & \\ 
 50-87 & 0.200 & $-$0.241 & $-$0.273$^{*}$ \\ 
  & (0.178) & (0.170) & (0.157) \\ 
  & & & \\ 
 Non voting & 0.074 & $-$0.102 & $-$0.028 \\ 
  & (0.110) & (0.105) & (0.097) \\ 
  & & & \\ 
 Other & 0.307$^{*}$ & 0.064 & $-$0.230 \\ 
  & (0.162) & (0.155) & (0.143) \\ 
  & & & \\ 
 Trump & 0.365$^{***}$ & $-$0.136$^{*}$ & $-$0.137$^{**}$ \\ 
  & (0.074) & (0.070) & (0.065) \\ 
  & & & \\ 
 Constant & 0.303 & 0.531$^{***}$ & 0.191 \\ 
  & (0.206) & (0.197) & (0.182) \\ 
  & & & \\ 
\hline \\[-1.8ex] 
Mean & 0.446 & 0.215 & 0.231 \\ 
Observations & 191 & 191 & 191 \\ 
\hline 
\hline \\[-1.8ex] 
\textit{Note:}  & \multicolumn{3}{r}{$^{*}$p$<$0.1; $^{**}$p$<$0.05; $^{***}$p$<$0.01} \\ 
\end{tabular} 
}
	\end{center}
	{\footnotesize Note: The variables are indicator variables. For instance, ``"Never" equals one if the respondent never talks about climate change.}
\end{table}		

\begin{table}[h!]
	\caption{Most affected generations}
	\begin{center}
		\scalebox{0.7}{
\begin{tabular}{@{\extracolsep{5pt}}lccccc} 
\\[-1.8ex]\hline 
\hline \\[-1.8ex] 
 & \multicolumn{5}{c}{Which generations will be seriously affected by CC?} \\ 
\cline{2-6} 
\\[-1.8ex] & Born in 1960s & Born in 1990s & Born in 2020s & Born in 2050s & None of them \\ 
\\[-1.8ex] & (1) & (2) & (3) & (4) & (5)\\ 
\hline \\[-1.8ex] 
 race: White only & 0.034 & $-$0.022 & 0.070 & $-$0.089 & $-$0.016 \\ 
  & (0.068) & (0.086) & (0.087) & (0.089) & (0.058) \\ 
  & & & & & \\ 
 Male & $-$0.025 & 0.100 & 0.192$^{**}$ & $-$0.018 & 0.055 \\ 
  & (0.058) & (0.074) & (0.075) & (0.077) & (0.050) \\ 
  & & & & & \\ 
 Children & 0.073 & $-$0.008 & 0.040 & $-$0.040 & $-$0.027 \\ 
  & (0.060) & (0.076) & (0.078) & (0.079) & (0.051) \\ 
  & & & & & \\ 
 No college & 0.019 & $-$0.033 & $-$0.081 & $-$0.255$^{***}$ & 0.154$^{***}$ \\ 
  & (0.067) & (0.085) & (0.086) & (0.087) & (0.057) \\ 
  & & & & & \\ 
 status: Retired & $-$0.006 & 0.209 & $-$0.144 & 0.102 & $-$0.040 \\ 
  & (0.106) & (0.136) & (0.138) & (0.140) & (0.091) \\ 
  & & & & & \\ 
 status: Student & $-$0.082 & 0.083 & $-$0.901$^{***}$ & $-$0.249 & 0.331 \\ 
  & (0.255) & (0.325) & (0.330) & (0.334) & (0.218) \\ 
  & & & & & \\ 
 status: Working & 0.066 & 0.154 & $-$0.296$^{**}$ & $-$0.079 & 0.045 \\ 
  & (0.105) & (0.133) & (0.135) & (0.137) & (0.090) \\ 
  & & & & & \\ 
 Income Q2 & $-$0.114 & $-$0.144 & 0.156 & 0.043 & 0.087 \\ 
  & (0.090) & (0.114) & (0.116) & (0.118) & (0.077) \\ 
  & & & & & \\ 
 Income Q3 & $-$0.120 & $-$0.050 & 0.014 & 0.009 & 0.110 \\ 
  & (0.085) & (0.108) & (0.109) & (0.111) & (0.072) \\ 
  & & & & & \\ 
 Income Q4 & $-$0.025 & $-$0.028 & $-$0.094 & $-$0.072 & 0.155$^{**}$ \\ 
  & (0.090) & (0.115) & (0.117) & (0.118) & (0.077) \\ 
  & & & & & \\ 
 age: 30-49 & 0.185 & $-$0.366$^{*}$ & $-$0.199 & $-$0.220 & 0.127 \\ 
  & (0.147) & (0.187) & (0.190) & (0.193) & (0.126) \\ 
  & & & & & \\ 
 age: 50-87 & $-$0.031 & $-$0.333$^{*}$ & $-$0.206 & $-$0.215 & 0.192 \\ 
  & (0.149) & (0.190) & (0.193) & (0.196) & (0.128) \\ 
  & & & & & \\ 
 vote: Biden & 0.046 & 0.214$^{**}$ & 0.129 & $-$0.021 & $-$0.051 \\ 
  & (0.082) & (0.104) & (0.106) & (0.107) & (0.070) \\ 
  & & & & & \\ 
 vote: Trump & $-$0.037 & $-$0.012 & $-$0.199$^{*}$ & $-$0.128 & 0.249$^{***}$ \\ 
  & (0.089) & (0.113) & (0.115) & (0.116) & (0.076) \\ 
  & & & & & \\ 
 Constant & 0.099 & 0.402$^{*}$ & 0.682$^{***}$ & 0.816$^{***}$ & $-$0.241$^{*}$ \\ 
  & (0.165) & (0.210) & (0.213) & (0.216) & (0.141) \\ 
  & & & & & \\ 
\hline \\[-1.8ex] 
Mean & 0.169 & 0.318 & 0.462 & 0.354 & 0.133 \\ 
Observations & 191 & 191 & 191 & 191 & 191 \\ 
\hline 
\hline \\[-1.8ex] 
\end{tabular} 
}
	\end{center}
	{\footnotesize Note: The variables are indicator variables. For instance, ``Born in 1960s" euqlas one if the respondent thinks the people currently between 50 and 60y. old will be seriously affected by climate change.}
\end{table}		

\begin{table}[h!]
	\caption{Scenario with worlwide consensus}
	\begin{center}
		\scalebox{0.7}{
\begin{tabular}{@{\extracolsep{5pt}}lc} 
\\[-1.8ex]\hline 
\hline \\[-1.8ex] 
 & \multicolumn{1}{c}{Scenario: world consensus to fight CC and wider green transports and energy available} \\ 
\cline{2-2} 
\\[-1.8ex] & Willing to change lifestyle \\ 
\hline \\[-1.8ex] 
 White only & 0.005 \\ 
  & (0.085) \\ 
  & \\ 
 Male & 0.080 \\ 
  & (0.074) \\ 
  & \\ 
 Children & 0.114 \\ 
  & (0.075) \\ 
  & \\ 
 No college & $-$0.082 \\ 
  & (0.084) \\ 
  & \\ 
 Retired & 0.161 \\ 
  & (0.133) \\ 
  & \\ 
 Student & $-$0.308 \\ 
  & (0.319) \\ 
  & \\ 
 Working & 0.137 \\ 
  & (0.131) \\ 
  & \\ 
 Income Q2 & $-$0.065 \\ 
  & (0.112) \\ 
  & \\ 
 Income Q3 & $-$0.070 \\ 
  & (0.106) \\ 
  & \\ 
 Income Q4 & $-$0.058 \\ 
  & (0.113) \\ 
  & \\ 
 30-49 & 0.074 \\ 
  & (0.184) \\ 
  & \\ 
 50-87 & $-$0.227 \\ 
  & (0.190) \\ 
  & \\ 
 Non voting & $-$0.333$^{***}$ \\ 
  & (0.117) \\ 
  & \\ 
 Other & $-$0.152 \\ 
  & (0.173) \\ 
  & \\ 
 Trump & $-$0.258$^{***}$ \\ 
  & (0.079) \\ 
  & \\ 
 Constant & 0.534$^{**}$ \\ 
  & (0.220) \\ 
  & \\ 
\hline \\[-1.8ex] 
Mean & 0.456 \\ 
Observations & 191 \\ 
\hline 
\hline \\[-1.8ex] 
\textit{Note:}  & \multicolumn{1}{r}{$^{*}$p$<$0.1; $^{**}$p$<$0.05; $^{***}$p$<$0.01} \\ 
\end{tabular} 
}
	\end{center}
	{\footnotesize Note: The variable is an indicator variable equal to one, if the respondent is willing to adopt a sustainaible lifestyle in scenario where all countries agree on wide-reaching measures to flight climate change (where non-polluting transports and renewable nergy are easily available).}
\end{table}		

\begin{landscape}
	\begin{table}[h!]
		\caption{Conditions to change lifestyle}
		\begin{center}
			\scalebox{0.5}{
\begin{tabular}{@{\extracolsep{5pt}}lcccccccc} 
\\[-1.8ex]\hline 
\hline \\[-1.8ex] 
 & \multicolumn{8}{c}{Would you be willing to change your lifestyle?} \\ 
\cline{2-9} 
\\[-1.8ex] & Yes, if policies in the good direction & Yes, if financial means & Yes, if everyone does the same & No, only rich should & No, would affect me more than living with CC & No, CC not a real problem & Lifestyle already sustainable & Trying, but trouble to change \\ 
\\[-1.8ex] & (1) & (2) & (3) & (4) & (5) & (6) & (7) & (8)\\ 
\hline \\[-1.8ex] 
 White only & 0.059 & $-$0.081 & $-$0.044 & $-$0.075$^{*}$ & 0.013 & $-$0.091 & 0.045 & 0.002 \\ 
  & (0.083) & (0.080) & (0.088) & (0.044) & (0.056) & (0.058) & (0.066) & (0.042) \\ 
  & & & & & & & & \\ 
 Male & 0.111 & $-$0.017 & 0.092 & 0.086$^{**}$ & $-$0.045 & 0.057 & 0.011 & $-$0.080$^{**}$ \\ 
  & (0.073) & (0.070) & (0.077) & (0.039) & (0.048) & (0.051) & (0.057) & (0.036) \\ 
  & & & & & & & & \\ 
 Children & 0.091 & $-$0.040 & 0.060 & 0.006 & 0.021 & 0.055 & $-$0.082 & 0.008 \\ 
  & (0.074) & (0.071) & (0.078) & (0.039) & (0.049) & (0.052) & (0.058) & (0.037) \\ 
  & & & & & & & & \\ 
 No college & 0.027 & $-$0.054 & 0.056 & 0.035 & 0.119$^{**}$ & 0.074 & $-$0.094 & $-$0.058 \\ 
  & (0.082) & (0.079) & (0.087) & (0.044) & (0.055) & (0.058) & (0.065) & (0.041) \\ 
  & & & & & & & & \\ 
 Retired & 0.058 & $-$0.020 & 0.230$^{*}$ & $-$0.005 & 0.075 & $-$0.140 & 0.082 & 0.091 \\ 
  & (0.130) & (0.125) & (0.137) & (0.069) & (0.087) & (0.091) & (0.103) & (0.065) \\ 
  & & & & & & & & \\ 
 Student & $-$0.104 & $-$0.067 & 0.204 & $-$0.270 & $-$0.057 & $-$0.147 & 0.446$^{*}$ & 0.034 \\ 
  & (0.310) & (0.298) & (0.328) & (0.165) & (0.207) & (0.218) & (0.245) & (0.155) \\ 
  & & & & & & & & \\ 
 Working & 0.081 & 0.013 & 0.165 & $-$0.0004 & 0.020 & 0.021 & 0.147 & 0.086 \\ 
  & (0.129) & (0.124) & (0.136) & (0.068) & (0.086) & (0.090) & (0.102) & (0.064) \\ 
  & & & & & & & & \\ 
 Income Q2 & $-$0.050 & $-$0.115 & 0.031 & $-$0.062 & 0.032 & 0.027 & $-$0.158$^{*}$ & 0.051 \\ 
  & (0.110) & (0.105) & (0.116) & (0.058) & (0.073) & (0.077) & (0.087) & (0.055) \\ 
  & & & & & & & & \\ 
 Income Q3 & $-$0.115 & $-$0.118 & $-$0.116 & $-$0.115$^{**}$ & 0.080 & 0.064 & $-$0.171$^{**}$ & $-$0.078 \\ 
  & (0.104) & (0.100) & (0.110) & (0.055) & (0.069) & (0.073) & (0.082) & (0.052) \\ 
  & & & & & & & & \\ 
 Income Q4 & 0.017 & $-$0.135 & $-$0.091 & $-$0.009 & 0.048 & 0.072 & $-$0.117 & $-$0.056 \\ 
  & (0.111) & (0.106) & (0.117) & (0.059) & (0.074) & (0.078) & (0.087) & (0.055) \\ 
  & & & & & & & & \\ 
 30-49 & 0.124 & $-$0.133 & $-$0.205 & $-$0.118 & 0.108 & $-$0.127 & $-$0.044 & 0.015 \\ 
  & (0.180) & (0.173) & (0.190) & (0.096) & (0.120) & (0.126) & (0.142) & (0.090) \\ 
  & & & & & & & & \\ 
 50-87 & $-$0.025 & $-$0.344$^{*}$ & $-$0.285 & $-$0.182$^{*}$ & 0.025 & 0.042 & 0.021 & 0.077 \\ 
  & (0.185) & (0.178) & (0.196) & (0.099) & (0.124) & (0.130) & (0.146) & (0.093) \\ 
  & & & & & & & & \\ 
 Non voting & $-$0.304$^{***}$ & $-$0.212$^{*}$ & $-$0.029 & 0.065 & 0.009 & 0.045 & $-$0.020 & $-$0.046 \\ 
  & (0.116) & (0.111) & (0.122) & (0.061) & (0.077) & (0.081) & (0.091) & (0.058) \\ 
  & & & & & & & & \\ 
 Other & $-$0.072 & 0.051 & 0.031 & $-$0.074 & $-$0.070 & $-$0.103 & $-$0.042 & 0.056 \\ 
  & (0.169) & (0.162) & (0.178) & (0.090) & (0.112) & (0.118) & (0.133) & (0.084) \\ 
  & & & & & & & & \\ 
 Trump & $-$0.255$^{***}$ & $-$0.075 & $-$0.116 & $-$0.014 & 0.012 & 0.231$^{***}$ & $-$0.030 & $-$0.076$^{*}$ \\ 
  & (0.077) & (0.074) & (0.081) & (0.041) & (0.051) & (0.054) & (0.061) & (0.039) \\ 
  & & & & & & & & \\ 
 Both treatments & $-$0.161 & $-$0.133 & $-$0.004 & $-$0.101$^{*}$ & $-$0.011 & $-$0.066 & 0.040 & $-$0.085 \\ 
  & (0.103) & (0.099) & (0.109) & (0.055) & (0.069) & (0.072) & (0.081) & (0.052) \\ 
  & & & & & & & & \\ 
 Climate treatment only & $-$0.055 & $-$0.111 & 0.072 & $-$0.056 & 0.014 & $-$0.054 & $-$0.069 & $-$0.092$^{*}$ \\ 
  & (0.097) & (0.093) & (0.103) & (0.052) & (0.065) & (0.068) & (0.077) & (0.049) \\ 
  & & & & & & & & \\ 
 Policy treatment only & $-$0.124 & $-$0.067 & 0.035 & $-$0.076 & $-$0.008 & $-$0.028 & $-$0.071 & $-$0.047 \\ 
  & (0.089) & (0.085) & (0.094) & (0.047) & (0.059) & (0.062) & (0.070) & (0.045) \\ 
  & & & & & & & & \\ 
 Constant & 0.293 & 0.843$^{***}$ & 0.352 & 0.315$^{***}$ & $-$0.072 & 0.063 & 0.225 & 0.092 \\ 
  & (0.222) & (0.213) & (0.235) & (0.118) & (0.148) & (0.156) & (0.176) & (0.111) \\ 
  & & & & & & & & \\ 
\hline \\[-1.8ex] 
Mean & 0.313 & 0.236 & 0.292 & 0.062 & 0.092 & 0.118 & 0.138 & 0.051 \\ 
Observations & 191 & 191 & 191 & 191 & 191 & 191 & 191 & 191 \\ 
\hline 
\hline \\[-1.8ex] 
\textit{Note:}  & \multicolumn{8}{r}{$^{*}$p$<$0.1; $^{**}$p$<$0.05; $^{***}$p$<$0.01} \\ 
\end{tabular} 
}
		\end{center}
	{\footnotesize Note: The dependent variables are indicator variables equal to one if the respondent selects this answer. For instance, \textit{Yes, if policies in the good direction} indicates that the respondent is willing to change her lifestyle to fight climate change if policies went in this direction. See note under Table \ref{table heating} for a description of the covariates.
	\newline *p$<$0.1; **p$<$0.05; ***p$<$0.01}
	\end{table}		
\end{landscape}

\begin{landscape}
	\begin{table}[h!]
		\caption{Effects of policies to halt CC}
		\begin{center}
			\scalebox{0.6}{
\begin{tabular}{@{\extracolsep{5pt}}lccc} 
\\[-1.8ex]\hline 
\hline \\[-1.8ex] 
 & \multicolumn{3}{c}{The policies aimed at halting CC would } \\ 
\cline{2-4} 
\\[-1.8ex] & be an opportunity for our economy and improve our lifestyle & be costly, but we would maintain our lifestyle & would require deep change in our lifestyle \\ 
\\[-1.8ex] & (1) & (2) & (3)\\ 
\hline \\[-1.8ex] 
 White only & $-$0.025 & 0.036 & 0.130 \\ 
  & (0.087) & (0.090) & (0.084) \\ 
  & & & \\ 
 Male & 0.134$^{*}$ & 0.183$^{**}$ & $-$0.082 \\ 
  & (0.076) & (0.078) & (0.073) \\ 
  & & & \\ 
 Children & 0.001 & $-$0.070 & 0.079 \\ 
  & (0.077) & (0.079) & (0.074) \\ 
  & & & \\ 
 No college & $-$0.036 & $-$0.102 & 0.030 \\ 
  & (0.086) & (0.089) & (0.083) \\ 
  & & & \\ 
 Retired & $-$0.004 & 0.186 & $-$0.086 \\ 
  & (0.136) & (0.140) & (0.131) \\ 
  & & & \\ 
 Student & $-$0.077 & 0.174 & 0.114 \\ 
  & (0.325) & (0.334) & (0.312) \\ 
  & & & \\ 
 Working & 0.057 & 0.078 & $-$0.128 \\ 
  & (0.135) & (0.139) & (0.130) \\ 
  & & & \\ 
 Income Q2 & $-$0.057 & $-$0.135 & 0.054 \\ 
  & (0.115) & (0.118) & (0.111) \\ 
  & & & \\ 
 Income Q3 & $-$0.002 & $-$0.044 & $-$0.159 \\ 
  & (0.109) & (0.112) & (0.105) \\ 
  & & & \\ 
 Income Q4 & 0.153 & $-$0.058 & $-$0.105 \\ 
  & (0.116) & (0.119) & (0.111) \\ 
  & & & \\ 
 30-49 & 0.237 & $-$0.308 & $-$0.030 \\ 
  & (0.188) & (0.194) & (0.181) \\ 
  & & & \\ 
 50-87 & 0.216 & $-$0.560$^{***}$ & 0.041 \\ 
  & (0.194) & (0.200) & (0.187) \\ 
  & & & \\ 
 Non voting & $-$0.151 & $-$0.103 & $-$0.299$^{**}$ \\ 
  & (0.121) & (0.125) & (0.116) \\ 
  & & & \\ 
 Other & $-$0.114 & 0.337$^{*}$ & 0.366$^{**}$ \\ 
  & (0.176) & (0.181) & (0.170) \\ 
  & & & \\ 
 Trump & $-$0.245$^{***}$ & $-$0.123 & $-$0.036 \\ 
  & (0.081) & (0.083) & (0.078) \\ 
  & & & \\ 
 Both treatments & $-$0.100 & $-$0.042 & $-$0.245$^{**}$ \\ 
  & (0.108) & (0.111) & (0.104) \\ 
  & & & \\ 
 Climate treatment only & $-$0.084 & 0.009 & $-$0.173$^{*}$ \\ 
  & (0.102) & (0.105) & (0.098) \\ 
  & & & \\ 
 Policy treatment only & $-$0.155$^{*}$ & $-$0.089 & $-$0.101 \\ 
  & (0.093) & (0.096) & (0.090) \\ 
  & & & \\ 
 Constant & 0.221 & 0.825$^{***}$ & 0.507$^{**}$ \\ 
  & (0.233) & (0.239) & (0.224) \\ 
  & & & \\ 
\hline \\[-1.8ex] 
Mean & 0.354 & 0.426 & 0.313 \\ 
Observations & 191 & 191 & 191 \\ 
\hline 
\hline \\[-1.8ex] 
\textit{Note:}  & \multicolumn{3}{r}{$^{*}$p$<$0.1; $^{**}$p$<$0.05; $^{***}$p$<$0.01} \\ 
\end{tabular} 
}
		\end{center}
	{\footnotesize Note: The dependent variables are indicator variables. For instance, the \textit{be an opportunity for our economy and improve our lifestyle} equals one, if the respondent thinks that policies aiming at halting climate change would have such effects. See note under Table \ref{table heating} for a description of the covariates.
	\newline *p$<$0.1; **p$<$0.05; ***p$<$0.01}
	\end{table}		
\end{landscape}

\begin{landscape}
	\begin{table}[h!]
		\caption{Issues to address to halt CC}
		\begin{center}
			\scalebox{0.6}{
\begin{tabular}{@{\extracolsep{5pt}}lcccccc} 
\\[-1.8ex]\hline 
\hline \\[-1.8ex] 
 & \multicolumn{6}{c}{Which issues need to be addressed to halt CC?} \\ 
\cline{2-7} 
\\[-1.8ex] & Use of technologies that emit GHG & Level of waste & High tax transfers of living & Overconsumption & Overpopulation & None of them \\ 
\\[-1.8ex] & (1) & (2) & (3) & (4) & (5) & (6)\\ 
\hline \\[-1.8ex] 
 White only & 0.203$^{**}$ & 0.001 & 0.027 & $-$0.080 & $-$0.074 & $-$0.026 \\ 
  & (0.085) & (0.092) & (0.069) & (0.081) & (0.084) & (0.060) \\ 
  & & & & & & \\ 
 Male & 0.119 & 0.035 & 0.061 & 0.061 & 0.119 & $-$0.023 \\ 
  & (0.074) & (0.080) & (0.060) & (0.071) & (0.073) & (0.052) \\ 
  & & & & & & \\ 
 Children & 0.121 & 0.063 & 0.091 & 0.019 & $-$0.014 & 0.003 \\ 
  & (0.076) & (0.082) & (0.061) & (0.073) & (0.075) & (0.053) \\ 
  & & & & & & \\ 
 No college & $-$0.177$^{**}$ & $-$0.091 & $-$0.026 & $-$0.133$^{*}$ & $-$0.112 & 0.127$^{**}$ \\ 
  & (0.084) & (0.091) & (0.068) & (0.080) & (0.083) & (0.059) \\ 
  & & & & & & \\ 
 Retired & 0.039 & 0.025 & 0.061 & $-$0.220$^{*}$ & 0.070 & $-$0.071 \\ 
  & (0.134) & (0.144) & (0.108) & (0.128) & (0.132) & (0.094) \\ 
  & & & & & & \\ 
 Student & $-$0.298 & 0.583$^{*}$ & $-$0.342 & $-$0.203 & $-$0.262 & 0.123 \\ 
  & (0.321) & (0.345) & (0.259) & (0.306) & (0.316) & (0.225) \\ 
  & & & & & & \\ 
 Working & $-$0.103 & $-$0.033 & 0.100 & $-$0.122 & 0.054 & $-$0.052 \\ 
  & (0.132) & (0.142) & (0.106) & (0.126) & (0.130) & (0.092) \\ 
  & & & & & & \\ 
 Income Q2 & $-$0.106 & $-$0.094 & $-$0.107 & $-$0.021 & 0.034 & $-$0.010 \\ 
  & (0.113) & (0.122) & (0.091) & (0.108) & (0.111) & (0.079) \\ 
  & & & & & & \\ 
 Income Q3 & $-$0.058 & 0.042 & $-$0.058 & $-$0.021 & $-$0.088 & 0.042 \\ 
  & (0.107) & (0.115) & (0.086) & (0.102) & (0.105) & (0.075) \\ 
  & & & & & & \\ 
 Income Q4 & $-$0.077 & $-$0.037 & $-$0.098 & $-$0.107 & $-$0.092 & 0.070 \\ 
  & (0.114) & (0.122) & (0.091) & (0.108) & (0.112) & (0.080) \\ 
  & & & & & & \\ 
 30-49 & $-$0.095 & 0.140 & $-$0.227 & 0.171 & 0.156 & $-$0.090 \\ 
  & (0.186) & (0.200) & (0.150) & (0.177) & (0.183) & (0.131) \\ 
  & & & & & & \\ 
 50-87 & $-$0.123 & 0.311 & $-$0.536$^{***}$ & 0.342$^{*}$ & 0.198 & $-$0.033 \\ 
  & (0.192) & (0.206) & (0.154) & (0.183) & (0.189) & (0.134) \\ 
  & & & & & & \\ 
 Non voting & $-$0.230$^{*}$ & $-$0.136 & $-$0.026 & $-$0.166 & 0.086 & 0.107 \\ 
  & (0.118) & (0.127) & (0.095) & (0.113) & (0.116) & (0.083) \\ 
  & & & & & & \\ 
 Other & 0.053 & 0.173 & 0.143 & $-$0.036 & 0.055 & $-$0.068 \\ 
  & (0.174) & (0.187) & (0.140) & (0.166) & (0.172) & (0.122) \\ 
  & & & & & & \\ 
 Trump & $-$0.374$^{***}$ & $-$0.184$^{**}$ & 0.011 & $-$0.200$^{***}$ & $-$0.077 & 0.215$^{***}$ \\ 
  & (0.079) & (0.085) & (0.064) & (0.076) & (0.078) & (0.056) \\ 
  & & & & & & \\ 
 Constant & 0.587$^{***}$ & 0.238 & 0.488$^{***}$ & 0.319 & 0.119 & 0.110 \\ 
  & (0.222) & (0.239) & (0.179) & (0.212) & (0.218) & (0.156) \\ 
  & & & & & & \\ 
\hline \\[-1.8ex] 
Mean & 0.487 & 0.436 & 0.195 & 0.267 & 0.272 & 0.123 \\ 
Observations & 191 & 191 & 191 & 191 & 191 & 191 \\ 
\hline 
\hline \\[-1.8ex] 
\textit{Note:}  & \multicolumn{6}{r}{$^{*}$p$<$0.1; $^{**}$p$<$0.05; $^{***}$p$<$0.01} \\ 
\end{tabular} 
}
		\end{center}
	{\footnotesize Note: The variables are indicator variables equal to one if the respondent thinks this issue should be addressed to halt climate change. For instance, \textit{Level of waste} equals one if the respondent thinks that we need to address the level of waste to halt climate change.}	
	\end{table}		
\end{landscape}

\clearpage
\subsection{International burden-sharing}

\begin{table}[h!]
	\caption{Best level to implement policies to tackle climate change}
	\begin{center}
		\scalebox{0.7}{
\begin{tabular}{@{\extracolsep{5pt}}lcccc} 
\\[-1.8ex]\hline 
\hline \\[-1.8ex] 
 & \multicolumn{4}{c}{The right level to implement policies to tackle CC is:} \\ 
\cline{2-5} 
\\[-1.8ex] & Local & State & Federal & Global \\ 
\\[-1.8ex] & (1) & (2) & (3) & (4)\\ 
\hline \\[-1.8ex] 
 race: White only & 0.050 & 0.003 & 0.038 & 0.090 \\ 
  & (0.087) & (0.091) & (0.089) & (0.088) \\ 
  & & & & \\ 
 Male & 0.031 & 0.061 & 0.165$^{**}$ & $-$0.097 \\ 
  & (0.075) & (0.079) & (0.077) & (0.076) \\ 
  & & & & \\ 
 Children & 0.054 & 0.195$^{**}$ & 0.064 & $-$0.016 \\ 
  & (0.078) & (0.081) & (0.079) & (0.078) \\ 
  & & & & \\ 
 No college & $-$0.093 & $-$0.042 & $-$0.049 & $-$0.355$^{***}$ \\ 
  & (0.086) & (0.090) & (0.088) & (0.086) \\ 
  & & & & \\ 
 status: Retired & 0.052 & 0.077 & $-$0.032 & $-$0.052 \\ 
  & (0.137) & (0.144) & (0.140) & (0.138) \\ 
  & & & & \\ 
 status: Student & $-$0.169 & 0.045 & $-$0.476 & $-$0.318 \\ 
  & (0.329) & (0.345) & (0.335) & (0.330) \\ 
  & & & & \\ 
 status: Working & 0.023 & $-$0.020 & $-$0.151 & $-$0.198 \\ 
  & (0.135) & (0.141) & (0.137) & (0.135) \\ 
  & & & & \\ 
 Income Q2 & 0.043 & $-$0.032 & 0.218$^{*}$ & 0.244$^{**}$ \\ 
  & (0.116) & (0.121) & (0.118) & (0.116) \\ 
  & & & & \\ 
 Income Q3 & $-$0.057 & $-$0.098 & 0.150 & 0.098 \\ 
  & (0.109) & (0.114) & (0.111) & (0.110) \\ 
  & & & & \\ 
 Income Q4 & 0.016 & $-$0.044 & 0.125 & 0.037 \\ 
  & (0.116) & (0.122) & (0.118) & (0.117) \\ 
  & & & & \\ 
 age: 30-49 & 0.141 & 0.180 & 0.192 & $-$0.229 \\ 
  & (0.190) & (0.198) & (0.193) & (0.190) \\ 
  & & & & \\ 
 age: 50-87 & 0.105 & 0.121 & 0.154 & $-$0.028 \\ 
  & (0.193) & (0.202) & (0.196) & (0.193) \\ 
  & & & & \\ 
 vote: Biden & 0.071 & 0.158 & 0.095 & 0.039 \\ 
  & (0.106) & (0.111) & (0.108) & (0.106) \\ 
  & & & & \\ 
 vote: Trump & $-$0.041 & $-$0.083 & $-$0.155 & $-$0.162 \\ 
  & (0.115) & (0.120) & (0.117) & (0.115) \\ 
  & & & & \\ 
 Constant & 0.078 & 0.130 & 0.064 & 0.778$^{***}$ \\ 
  & (0.213) & (0.223) & (0.217) & (0.214) \\ 
  & & & & \\ 
\hline \\[-1.8ex] 
Mean & 0.303 & 0.446 & 0.41 & 0.508 \\ 
Observations & 191 & 191 & 191 & 191 \\ 
\hline 
\hline \\[-1.8ex] 
\end{tabular} 
}
	\end{center}
	{\footnotesize Note: The variables are indicator variables equal to one if the respondent thinks public policies to tackle climate change need to be put in place at this level.}
\end{table}	

\begin{landscape}
	\begin{table}[h!]
	\caption{Countries that should bear the costs}
	\begin{center}
		\scalebox{0.6}{
\begin{tabular}{@{\extracolsep{5pt}}lccccc} 
\\[-1.8ex]\hline 
\hline \\[-1.8ex] 
 & \multicolumn{5}{c}{Which countries bear should bear the costs of fighting CC?} \\ 
\cline{2-6} 
\\[-1.8ex] & Pay in proportion to income & Pay in proportion to current emissions & Pay in proportion to past emissions (from 1990) & Richest pay alone & Richest pay, and even more to help vulnerable countries \\ 
\\[-1.8ex] & (1) & (2) & (3) & (4) & (5)\\ 
\hline \\[-1.8ex] 
 White only & $-$0.057 & 0.012 & 0.019 & 0.011 & 0.108 \\ 
  & (0.087) & (0.089) & (0.086) & (0.071) & (0.080) \\ 
  & & & & & \\ 
 Male & 0.041 & 0.163$^{**}$ & 0.156$^{**}$ & 0.067 & 0.069 \\ 
  & (0.076) & (0.078) & (0.075) & (0.062) & (0.070) \\ 
  & & & & & \\ 
 Children & 0.035 & 0.112 & 0.142$^{*}$ & 0.032 & 0.083 \\ 
  & (0.077) & (0.079) & (0.076) & (0.063) & (0.071) \\ 
  & & & & & \\ 
 No college & 0.046 & $-$0.087 & 0.032 & 0.055 & 0.085 \\ 
  & (0.086) & (0.088) & (0.085) & (0.070) & (0.079) \\ 
  & & & & & \\ 
 Retired & $-$0.060 & $-$0.101 & $-$0.047 & 0.051 & $-$0.028 \\ 
  & (0.135) & (0.139) & (0.134) & (0.111) & (0.125) \\ 
  & & & & & \\ 
 Student & $-$0.342 & $-$0.634$^{*}$ & $-$0.284 & 0.175 & $-$0.263 \\ 
  & (0.323) & (0.332) & (0.319) & (0.265) & (0.298) \\ 
  & & & & & \\ 
 Working & 0.095 & $-$0.021 & $-$0.038 & 0.147 & 0.028 \\ 
  & (0.134) & (0.138) & (0.132) & (0.110) & (0.124) \\ 
  & & & & & \\ 
 Income Q2 & $-$0.029 & $-$0.031 & $-$0.081 & $-$0.011 & $-$0.015 \\ 
  & (0.114) & (0.118) & (0.113) & (0.094) & (0.105) \\ 
  & & & & & \\ 
 Income Q3 & 0.024 & 0.005 & 0.042 & 0.036 & 0.039 \\ 
  & (0.108) & (0.112) & (0.107) & (0.089) & (0.100) \\ 
  & & & & & \\ 
 Income Q4 & 0.061 & 0.069 & 0.054 & 0.096 & 0.005 \\ 
  & (0.115) & (0.118) & (0.114) & (0.094) & (0.106) \\ 
  & & & & & \\ 
 30-49 & $-$0.315$^{*}$ & $-$0.464$^{**}$ & $-$0.274 & $-$0.119 & $-$0.297$^{*}$ \\ 
  & (0.187) & (0.193) & (0.185) & (0.153) & (0.173) \\ 
  & & & & & \\ 
 50-87 & $-$0.492$^{**}$ & $-$0.393$^{**}$ & $-$0.576$^{***}$ & $-$0.558$^{***}$ & $-$0.608$^{***}$ \\ 
  & (0.193) & (0.199) & (0.191) & (0.158) & (0.178) \\ 
  & & & & & \\ 
 Non voting & $-$0.331$^{***}$ & $-$0.272$^{**}$ & $-$0.325$^{***}$ & $-$0.167$^{*}$ & $-$0.253$^{**}$ \\ 
  & (0.120) & (0.124) & (0.119) & (0.099) & (0.111) \\ 
  & & & & & \\ 
 Other & 0.061 & 0.024 & $-$0.392$^{**}$ & $-$0.221 & $-$0.442$^{***}$ \\ 
  & (0.175) & (0.181) & (0.173) & (0.144) & (0.162) \\ 
  & & & & & \\ 
 Trump & $-$0.282$^{***}$ & $-$0.142$^{*}$ & $-$0.170$^{**}$ & $-$0.133$^{**}$ & $-$0.275$^{***}$ \\ 
  & (0.080) & (0.082) & (0.079) & (0.066) & (0.074) \\ 
  & & & & & \\ 
 Climate treatment only & $-$0.168 & $-$0.214$^{*}$ & 0.096 & 0.101 & 0.021 \\ 
  & (0.109) & (0.112) & (0.107) & (0.089) & (0.100) \\ 
  & & & & & \\ 
 No treatment & $-$0.088 & $-$0.014 & 0.008 & 0.069 & 0.195$^{**}$ \\ 
  & (0.107) & (0.110) & (0.106) & (0.088) & (0.099) \\ 
  & & & & & \\ 
 Policy treatment only & $-$0.063 & 0.020 & $-$0.024 & 0.070 & 0.056 \\ 
  & (0.097) & (0.100) & (0.096) & (0.080) & (0.090) \\ 
  & & & & & \\ 
 Constant & 1.058$^{***}$ & 0.971$^{***}$ & 0.817$^{***}$ & 0.482$^{**}$ & 0.704$^{***}$ \\ 
  & (0.234) & (0.241) & (0.232) & (0.192) & (0.216) \\ 
  & & & & & \\ 
\hline \\[-1.8ex] 
Mean & 0.477 & 0.569 & 0.451 & 0.292 & 0.364 \\ 
Observations & 191 & 191 & 191 & 191 & 191 \\ 
\hline 
\hline \\[-1.8ex] 
\textit{Note:}  & \multicolumn{5}{r}{$^{*}$p$<$0.1; $^{**}$p$<$0.05; $^{***}$p$<$0.01} \\ 
\end{tabular} 
}
	\end{center}
	{\footnotesize Note: The dependent variables are indicator variables equal to one if the respondents indicates to ``Strongly agree" or ``Somewhat agree" to this proposition regarding how countries should bear the costs of fighting climate change. For instance, \textit{Pay in proportion to income} equals one if the respondent thinks that all countries should pay in proportion to their income. See note under Table \ref{table heating} for a description of the covariates.
	\newline *p$<$0.1; **p$<$0.05; ***p$<$0.01}
\end{table}	
\end{landscape}


\begin{landscape}
	\begin{table}[h!]
	\caption{Right to pollute}
	\begin{center}
		\scalebox{0.6}{
\begin{tabular}{@{\extracolsep{5pt}}lccccc} 
\\[-1.8ex]\hline 
\hline \\[-1.8ex] 
 & \multicolumn{5}{c}{Are you in favor of a system of equal quota to emit GHG at individual levels, with monetary compensation and tax?} \\ 
\cline{2-6} 
\\[-1.8ex] & No, should compensate the poorest & Yes & No, if pollute more, more rights & No, not at individual level & No, no restrictions of emissions \\ 
\\[-1.8ex] & (1) & (2) & (3) & (4) & (5)\\ 
\hline \\[-1.8ex] 
 White only & 0.027 & 0.026 & 0.026 & $-$0.033 & 0.003 \\ 
  & (0.054) & (0.080) & (0.037) & (0.077) & (0.055) \\ 
  & & & & & \\ 
 Male & 0.026 & 0.102 & 0.009 & $-$0.030 & $-$0.045 \\ 
  & (0.047) & (0.070) & (0.033) & (0.067) & (0.048) \\ 
  & & & & & \\ 
 Children & $-$0.030 & 0.044 & 0.023 & 0.019 & $-$0.035 \\ 
  & (0.048) & (0.071) & (0.033) & (0.068) & (0.049) \\ 
  & & & & & \\ 
 No college & 0.039 & 0.036 & $-$0.027 & $-$0.079 & 0.086 \\ 
  & (0.054) & (0.079) & (0.037) & (0.076) & (0.054) \\ 
  & & & & & \\ 
 Retired & $-$0.158$^{*}$ & $-$0.074 & 0.021 & 0.069 & 0.028 \\ 
  & (0.085) & (0.125) & (0.058) & (0.120) & (0.086) \\ 
  & & & & & \\ 
 Student & 0.211 & 0.009 & 0.205 & $-$0.156 & $-$0.080 \\ 
  & (0.202) & (0.298) & (0.139) & (0.285) & (0.205) \\ 
  & & & & & \\ 
 Working & $-$0.032 & $-$0.070 & $-$0.010 & 0.041 & $-$0.016 \\ 
  & (0.084) & (0.124) & (0.058) & (0.118) & (0.085) \\ 
  & & & & & \\ 
 Income Q2 & 0.029 & $-$0.155 & 0.040 & 0.019 & 0.078 \\ 
  & (0.071) & (0.105) & (0.049) & (0.101) & (0.072) \\ 
  & & & & & \\ 
 Income Q3 & $-$0.061 & $-$0.050 & 0.019 & 0.017 & 0.150$^{**}$ \\ 
  & (0.068) & (0.100) & (0.047) & (0.096) & (0.069) \\ 
  & & & & & \\ 
 Income Q4 & 0.011 & 0.058 & $-$0.036 & $-$0.026 & 0.101 \\ 
  & (0.072) & (0.106) & (0.050) & (0.102) & (0.073) \\ 
  & & & & & \\ 
 30-49 & 0.009 & 0.092 & $-$0.188$^{**}$ & $-$0.145 & 0.106 \\ 
  & (0.117) & (0.173) & (0.081) & (0.165) & (0.119) \\ 
  & & & & & \\ 
 50-87 & 0.089 & $-$0.190 & $-$0.248$^{***}$ & $-$0.063 & 0.048 \\ 
  & (0.120) & (0.178) & (0.083) & (0.170) & (0.122) \\ 
  & & & & & \\ 
 Non voting & $-$0.072 & $-$0.175 & $-$0.092$^{*}$ & 0.031 & 0.055 \\ 
  & (0.075) & (0.111) & (0.052) & (0.106) & (0.076) \\ 
  & & & & & \\ 
 Other & $-$0.162 & $-$0.357$^{**}$ & $-$0.018 & $-$0.062 & 0.135 \\ 
  & (0.110) & (0.162) & (0.076) & (0.155) & (0.111) \\ 
  & & & & & \\ 
 Trump & $-$0.050 & $-$0.189$^{**}$ & 0.0004 & 0.059 & 0.179$^{***}$ \\ 
  & (0.050) & (0.074) & (0.034) & (0.071) & (0.051) \\ 
  & & & & & \\ 
 Both & $-$0.048 & $-$0.034 & $-$0.031 & 0.075 & 0.058 \\ 
  & (0.067) & (0.099) & (0.046) & (0.095) & (0.068) \\ 
  & & & & & \\ 
 Climate treatment only & $-$0.120$^{*}$ & $-$0.003 & 0.020 & $-$0.144 & 0.156$^{**}$ \\ 
  & (0.063) & (0.093) & (0.044) & (0.089) & (0.064) \\ 
  & & & & & \\ 
 Policy treatment only & $-$0.025 & 0.0001 & $-$0.059 & $-$0.017 & 0.029 \\ 
  & (0.058) & (0.086) & (0.040) & (0.082) & (0.059) \\ 
  & & & & & \\ 
 Constant & 0.163 & 0.461$^{**}$ & 0.250$^{**}$ & 0.291 & $-$0.158 \\ 
  & (0.144) & (0.214) & (0.100) & (0.204) & (0.147) \\ 
  & & & & & \\ 
\hline \\[-1.8ex] 
Mean & 0.087 & 0.287 & 0.041 & 0.2 & 0.097 \\ 
Observations & 191 & 191 & 191 & 191 & 191 \\ 
\hline 
\hline \\[-1.8ex] 
\textit{Note:}  & \multicolumn{5}{r}{$^{*}$p$<$0.1; $^{**}$p$<$0.05; $^{***}$p$<$0.01} \\ 
\end{tabular} 
}
	\end{center}
	{\footnotesize Note: The variables are indicator variables equal to one if the respondent is in favor of this proposition regarding the implementation of an equal allowance to emit GHG (where big polluters pay for their excess emissions and those who pollute less receive a monetary compensation). For instance, the \textit{No, should compensate the poorest} variable equals one if the respondent does not agree to this proposal because she thinks ``those who will be hurt more by climate change should be compensated more", \textit{Yes} if the respondent thinks ``this would be a fair solution", \textit{No, if pollute more more rights} if the respondent thinks ``those who currently pollute more should have more rights to pollute", \textit{No, not at individual levels} if the respondent thinks ``rights to pollute should not be defined at the individual level but at another level", and \textit{No, no restrictions of emissions} if the respondent thinks ``we should not restrict GHG emissions."}
\end{table}	
\end{landscape}

\begin{table}[h!]
	\caption{Should the U.S. act?} \label{table US should act}
	\begin{center}
		\scalebox{0.7}{
\begin{tabular}{@{\extracolsep{5pt}}lccc} 
\\[-1.8ex]\hline 
\hline \\[-1.8ex] 
 & \multicolumn{3}{c}{Should the U.S. take measures to fight CC?} \\ 
\cline{2-4} 
\\[-1.8ex] & Yes & Only if fair international agreement & No \\ 
\\[-1.8ex] & (1) & (2) & (3)\\ 
\hline \\[-1.8ex] 
 White only & 0.051 & 0.048 & $-$0.077 \\ 
  & (0.086) & (0.074) & (0.073) \\ 
  & & & \\ 
 Male & 0.017 & 0.130$^{**}$ & $-$0.014 \\ 
  & (0.075) & (0.065) & (0.064) \\ 
  & & & \\ 
 Children & 0.153$^{**}$ & $-$0.042 & $-$0.049 \\ 
  & (0.076) & (0.066) & (0.065) \\ 
  & & & \\ 
 No college & 0.007 & $-$0.025 & 0.093 \\ 
  & (0.085) & (0.073) & (0.073) \\ 
  & & & \\ 
 Retired & $-$0.052 & 0.027 & $-$0.002 \\ 
  & (0.135) & (0.116) & (0.115) \\ 
  & & & \\ 
 Student & $-$0.074 & $-$0.329 & 0.423 \\ 
  & (0.321) & (0.276) & (0.273) \\ 
  & & & \\ 
 Working & $-$0.039 & $-$0.131 & 0.085 \\ 
  & (0.133) & (0.114) & (0.113) \\ 
  & & & \\ 
 Income Q2 & $-$0.040 & 0.021 & 0.076 \\ 
  & (0.114) & (0.098) & (0.097) \\ 
  & & & \\ 
 Income Q3 & 0.041 & $-$0.083 & 0.074 \\ 
  & (0.108) & (0.093) & (0.092) \\ 
  & & & \\ 
 Income Q4 & 0.057 & $-$0.079 & 0.103 \\ 
  & (0.115) & (0.098) & (0.097) \\ 
  & & & \\ 
 30-49 & $-$0.041 & 0.012 & $-$0.009 \\ 
  & (0.186) & (0.160) & (0.158) \\ 
  & & & \\ 
 50-87 & $-$0.056 & $-$0.049 & $-$0.050 \\ 
  & (0.192) & (0.165) & (0.163) \\ 
  & & & \\ 
 Non voting & $-$0.305$^{**}$ & 0.172$^{*}$ & 0.001 \\ 
  & (0.120) & (0.103) & (0.102) \\ 
  & & & \\ 
 Other & $-$0.228 & 0.243 & $-$0.079 \\ 
  & (0.175) & (0.150) & (0.148) \\ 
  & & & \\ 
 Trump & $-$0.479$^{***}$ & 0.283$^{***}$ & 0.192$^{***}$ \\ 
  & (0.080) & (0.068) & (0.068) \\ 
  & & & \\ 
 Climate treatment only & $-$0.082 & 0.007 & 0.079 \\ 
  & (0.108) & (0.093) & (0.092) \\ 
  & & & \\ 
 No treatment & 0.047 & $-$0.042 & 0.007 \\ 
  & (0.107) & (0.092) & (0.091) \\ 
  & & & \\ 
 Policy treatment only & 0.003 & $-$0.018 & 0.011 \\ 
  & (0.097) & (0.083) & (0.082) \\ 
  & & & \\ 
 Constant & 0.623$^{***}$ & 0.142 & 0.100 \\ 
  & (0.233) & (0.200) & (0.198) \\ 
  & & & \\ 
\hline \\[-1.8ex] 
Mean & 0.492 & 0.205 & 0.19 \\ 
Observations & 191 & 191 & 191 \\ 
\hline 
\hline \\[-1.8ex] 
\textit{Note:}  & \multicolumn{3}{r}{$^{*}$p$<$0.1; $^{**}$p$<$0.05; $^{***}$p$<$0.01} \\ 
\end{tabular} 
}
	\end{center}
	{\footnotesize Note: The dependent variables are indicator variables. For instance, the \textit{Yes} variable equals one if the respondent thinks the U.S. should take measures to flight climate change. See note under Table \ref{table heating} for a description of the covariates.
	\newline *p$<$0.1; **p$<$0.05; ***p$<$0.01}
\end{table}	

\begin{table}[h!]
	\caption{Extent to which the U.S. should act}
	\begin{center}
		\scalebox{0.7}{
\begin{tabular}{@{\extracolsep{5pt}}lccc} 
\\[-1.8ex]\hline 
\hline \\[-1.8ex] 
 & \multicolumn{3}{c}{How what the U.S. should do depends on what other countries do?} \\ 
\cline{2-4} 
\\[-1.8ex] & U.S. more ambitious, if others less & U.S. more ambitious, if others as well & U.S. less ambitious, if others are \\ 
\\[-1.8ex] & (1) & (2) & (3)\\ 
\hline \\[-1.8ex] 
 White only & $-$0.010 & $-$0.028 & 0.039 \\ 
  & (0.104) & (0.109) & (0.059) \\ 
  & & & \\ 
 Male & 0.116 & $-$0.047 & $-$0.068 \\ 
  & (0.096) & (0.101) & (0.055) \\ 
  & & & \\ 
 Children & 0.045 & 0.015 & $-$0.061 \\ 
  & (0.097) & (0.101) & (0.055) \\ 
  & & & \\ 
 No college & 0.044 & $-$0.173 & 0.129$^{**}$ \\ 
  & (0.104) & (0.108) & (0.059) \\ 
  & & & \\ 
 Retired & $-$0.002 & $-$0.050 & 0.052 \\ 
  & (0.155) & (0.162) & (0.088) \\ 
  & & & \\ 
 Student & $-$0.451 & 0.555 & $-$0.104 \\ 
  & (0.546) & (0.569) & (0.311) \\ 
  & & & \\ 
 Working & 0.097 & $-$0.150 & 0.054 \\ 
  & (0.162) & (0.169) & (0.092) \\ 
  & & & \\ 
 Income Q2 & 0.039 & $-$0.006 & $-$0.032 \\ 
  & (0.139) & (0.145) & (0.079) \\ 
  & & & \\ 
 Income Q3 & 0.014 & 0.038 & $-$0.052 \\ 
  & (0.134) & (0.140) & (0.076) \\ 
  & & & \\ 
 Income Q4 & 0.126 & $-$0.071 & $-$0.055 \\ 
  & (0.139) & (0.145) & (0.079) \\ 
  & & & \\ 
 30-49 & 0.007 & $-$0.061 & 0.055 \\ 
  & (0.249) & (0.260) & (0.142) \\ 
  & & & \\ 
 50-87 & $-$0.280 & 0.254 & 0.026 \\ 
  & (0.261) & (0.272) & (0.148) \\ 
  & & & \\ 
 Non voting & $-$0.227 & 0.240 & $-$0.013 \\ 
  & (0.160) & (0.167) & (0.091) \\ 
  & & & \\ 
 Other & 0.021 & 0.010 & $-$0.032 \\ 
  & (0.188) & (0.196) & (0.107) \\ 
  & & & \\ 
 Trump & $-$0.317$^{***}$ & 0.163 & 0.154$^{***}$ \\ 
  & (0.097) & (0.101) & (0.055) \\ 
  & & & \\ 
 Constant & 0.596$^{*}$ & 0.384 & 0.020 \\ 
  & (0.315) & (0.328) & (0.179) \\ 
  & & & \\ 
\hline \\[-1.8ex] 
Mean & 0.522 & 0.403 & 0.075 \\ 
Observations & 133 & 133 & 133 \\ 
\hline 
\hline \\[-1.8ex] 
\textit{Note:}  & \multicolumn{3}{r}{$^{*}$p$<$0.1; $^{**}$p$<$0.05; $^{***}$p$<$0.01} \\ 
\end{tabular} 
}
	\end{center}
	{\footnotesize Note: The dependent variables are indicator variables equal to one if the respondent agrees with the proposition. For instance, \textit{U.S. more ambitious, if others less} equals one if the respondent thinks ``The U.S. should take even more ambitious measures if other countries are less amibitious." The sample includes respondents who answered \textit{Yes} or \textit{Only if fair international agreement at the question from Table \ref{table US should act}.   See note under Table \ref{table heating} for a description of the covariates.}
	\newline *p$<$0.1; **p$<$0.05; ***p$<$0.01}
\end{table}	

\begin{landscape}
	\begin{table}[h!]
	\caption{International measures}
	\begin{center}
		\scalebox{0.6}{
\begin{tabular}{@{\extracolsep{5pt}}lccc} 
\\[-1.8ex]\hline 
\hline \\[-1.8ex] 
 & \multicolumn{3}{c}{Approve those measures} \\ 
\cline{2-4} 
\\[-1.8ex] & Global democratic assembly to fight CC & Global tax on GHG emissions funding a global basic income (30 dollars per month per adult) & Global tax on top 1% to finance poorest countries \\ 
\\[-1.8ex] & (1) & (2) & (3)\\ 
\hline \\[-1.8ex] 
 White only & 0.117 & 0.122 & 0.201$^{**}$ \\ 
  & (0.087) & (0.078) & (0.081) \\ 
  & & & \\ 
 Male & 0.071 & 0.101 & 0.017 \\ 
  & (0.076) & (0.068) & (0.071) \\ 
  & & & \\ 
 Children & 0.117 & 0.038 & 0.145$^{**}$ \\ 
  & (0.077) & (0.069) & (0.072) \\ 
  & & & \\ 
 No college & 0.016 & 0.111 & 0.015 \\ 
  & (0.086) & (0.077) & (0.081) \\ 
  & & & \\ 
 Retired & $-$0.024 & $-$0.017 & $-$0.065 \\ 
  & (0.135) & (0.122) & (0.127) \\ 
  & & & \\ 
 Student & $-$0.405 & $-$0.434 & $-$0.130 \\ 
  & (0.322) & (0.291) & (0.303) \\ 
  & & & \\ 
 Working & $-$0.040 & 0.169 & 0.004 \\ 
  & (0.134) & (0.121) & (0.126) \\ 
  & & & \\ 
 Income Q2 & $-$0.030 & $-$0.049 & $-$0.053 \\ 
  & (0.114) & (0.103) & (0.107) \\ 
  & & & \\ 
 Income Q3 & $-$0.071 & $-$0.043 & $-$0.150 \\ 
  & (0.108) & (0.098) & (0.102) \\ 
  & & & \\ 
 Income Q4 & 0.001 & 0.038 & $-$0.100 \\ 
  & (0.115) & (0.104) & (0.108) \\ 
  & & & \\ 
 30-49 & $-$0.005 & $-$0.180 & $-$0.212 \\ 
  & (0.187) & (0.169) & (0.176) \\ 
  & & & \\ 
 50-87 & $-$0.159 & $-$0.438$^{**}$ & $-$0.403$^{**}$ \\ 
  & (0.193) & (0.174) & (0.181) \\ 
  & & & \\ 
 Non voting & $-$0.383$^{***}$ & $-$0.212$^{*}$ & $-$0.415$^{***}$ \\ 
  & (0.120) & (0.109) & (0.113) \\ 
  & & & \\ 
 Other & $-$0.428$^{**}$ & $-$0.469$^{***}$ & $-$0.371$^{**}$ \\ 
  & (0.175) & (0.158) & (0.165) \\ 
  & & & \\ 
 Trump & $-$0.392$^{***}$ & $-$0.281$^{***}$ & $-$0.448$^{***}$ \\ 
  & (0.080) & (0.072) & (0.075) \\ 
  & & & \\ 
 Both treatments & 0.072 & $-$0.014 & $-$0.222$^{**}$ \\ 
  & (0.107) & (0.097) & (0.101) \\ 
  & & & \\ 
 Climate treatment only & $-$0.028 & $-$0.081 & $-$0.143 \\ 
  & (0.101) & (0.091) & (0.095) \\ 
  & & & \\ 
 Policy treatment only & 0.008 & $-$0.081 & $-$0.161$^{*}$ \\ 
  & (0.093) & (0.084) & (0.087) \\ 
  & & & \\ 
 Constant & 0.598$^{**}$ & 0.611$^{***}$ & 0.949$^{***}$ \\ 
  & (0.231) & (0.209) & (0.217) \\ 
  & & & \\ 
\hline \\[-1.8ex] 
Mean & 0.462 & 0.359 & 0.431 \\ 
Observations & 191 & 191 & 191 \\ 
\hline 
\hline \\[-1.8ex] 
\textit{Note:}  & \multicolumn{3}{r}{$^{*}$p$<$0.1; $^{**}$p$<$0.05; $^{***}$p$<$0.01} \\ 
\end{tabular} 
}
	\end{center}
	{\footnotesize Note: The dependent variables are indicator variables equal to one if the respondent approves this proposition. For instance, ``Global democratic assembly to fight CC" equals one if the respondent approves of ``establishing a global democratic assembly which role would be to take action against climate change." See note under Table \ref{table heating} for a description of the covariates.
	\newline *p$<$0.1; **p$<$0.05; ***p$<$0.01}
\end{table}	
\end{landscape}


\clearpage
\section{Post-treatment}
\subsection{Preferences 1: Emission standards}

\begin{table}[h!]
	\caption{Opinion on emission standards} \label{table standard opinion}
	\begin{center}
		\scalebox{0.7}{
\begin{tabular}{@{\extracolsep{5pt}}lcccccc} 
\\[-1.8ex]\hline 
\hline \\[-1.8ex] 
 & \multicolumn{6}{c}{C02 emission limit for cars policy in the U.S.} \\ 
\cline{2-7} 
\\[-1.8ex] & Does exist & Trust federal gov. & Effective & Positive impact on jobs & Positive side effects & Support \\ 
\hline \\[-1.8ex] 
 Control group mean & 0.188 & 0.333 & 0.438 & 0.312 & 0.479 & 0.583  \\ \hline \\[-1.8ex] race: White only & $-$0.059 & 0.085 & 0.092 & 0.090 & 0.157$^{*}$ & 0.140 \\ 
  & (0.083) & (0.088) & (0.087) & (0.084) & (0.090) & (0.088) \\ 
  & & & & & & \\ 
 Male & 0.071 & 0.054 & 0.041 & $-$0.003 & 0.093 & 0.112 \\ 
  & (0.071) & (0.076) & (0.075) & (0.073) & (0.078) & (0.076) \\ 
  & & & & & & \\ 
 Children & 0.138$^{*}$ & 0.160$^{**}$ & 0.066 & 0.028 & 0.024 & 0.022 \\ 
  & (0.073) & (0.078) & (0.077) & (0.074) & (0.079) & (0.077) \\ 
  & & & & & & \\ 
 No college & $-$0.030 & 0.023 & $-$0.071 & $-$0.082 & $-$0.119 & $-$0.092 \\ 
  & (0.082) & (0.087) & (0.086) & (0.083) & (0.089) & (0.087) \\ 
  & & & & & & \\ 
 status: Retired & 0.063 & $-$0.029 & 0.143 & 0.070 & 0.217 & 0.161 \\ 
  & (0.129) & (0.138) & (0.136) & (0.132) & (0.140) & (0.137) \\ 
  & & & & & & \\ 
 status: Student & $-$0.387 & $-$0.152 & $-$0.312 & $-$0.048 & $-$0.371 & $-$0.351 \\ 
  & (0.308) & (0.328) & (0.324) & (0.314) & (0.335) & (0.328) \\ 
  & & & & & & \\ 
 staths: Working & 0.078 & 0.028 & 0.143 & 0.102 & 0.132 & 0.067 \\ 
  & (0.128) & (0.136) & (0.134) & (0.130) & (0.139) & (0.136) \\ 
  & & & & & & \\ 
 Income Q2 & $-$0.040 & 0.107 & $-$0.022 & 0.029 & $-$0.063 & 0.139 \\ 
  & (0.109) & (0.116) & (0.115) & (0.111) & (0.118) & (0.116) \\ 
  & & & & & & \\ 
 Income Q3 & $-$0.012 & 0.160 & $-$0.002 & 0.082 & 0.028 & 0.094 \\ 
  & (0.103) & (0.110) & (0.109) & (0.106) & (0.112) & (0.110) \\ 
  & & & & & & \\ 
 Income Q4 & 0.149 & 0.135 & 0.051 & 0.063 & 0.026 & 0.073 \\ 
  & (0.110) & (0.117) & (0.116) & (0.112) & (0.119) & (0.117) \\ 
  & & & & & & \\ 
 age: 30-49 & $-$0.123 & $-$0.207 & 0.189 & $-$0.093 & $-$0.218 & $-$0.338$^{*}$ \\ 
  & (0.178) & (0.189) & (0.187) & (0.181) & (0.193) & (0.189) \\ 
  & & & & & & \\ 
 age: 50-87 & $-$0.348$^{*}$ & $-$0.363$^{*}$ & 0.077 & $-$0.357$^{*}$ & $-$0.353$^{*}$ & $-$0.427$^{**}$ \\ 
  & (0.180) & (0.192) & (0.190) & (0.184) & (0.196) & (0.192) \\ 
  & & & & & & \\ 
 vote: Biden & 0.010 & 0.230$^{**}$ & 0.162 & 0.177$^{*}$ & 0.139 & 0.192$^{*}$ \\ 
  & (0.100) & (0.106) & (0.105) & (0.102) & (0.109) & (0.106) \\ 
  & & & & & & \\ 
 vote: Trump & 0.062 & 0.043 & $-$0.172 & $-$0.085 & $-$0.112 & $-$0.171 \\ 
  & (0.109) & (0.116) & (0.115) & (0.111) & (0.118) & (0.116) \\ 
  & & & & & & \\ 
 Both treatments & 0.231$^{**}$ & 0.269$^{**}$ & 0.249$^{**}$ & 0.136 & 0.093 & 0.080 \\ 
  & (0.102) & (0.109) & (0.108) & (0.104) & (0.111) & (0.109) \\ 
  & & & & & & \\ 
 Climate treatment only & 0.119 & 0.133 & 0.135 & 0.042 & 0.003 & 0.056 \\ 
  & (0.096) & (0.102) & (0.101) & (0.098) & (0.104) & (0.102) \\ 
  & & & & & & \\ 
 Policy treatment only & 0.151$^{*}$ & 0.132 & 0.223$^{**}$ & 0.040 & 0.117 & $-$0.006 \\ 
  & (0.088) & (0.094) & (0.093) & (0.090) & (0.096) & (0.094) \\ 
  & & & & & & \\ 
 Constant & 0.250 & 0.134 & 0.022 & 0.332 & 0.390$^{*}$ & 0.558$^{**}$ \\ 
  & (0.214) & (0.228) & (0.225) & (0.218) & (0.232) & (0.227) \\ 
  & & & & & & \\ 
\hline \\[-1.8ex] 

Observations & 191 & 191 & 191 & 191 & 191 & 191 \\ 
\hline 
\hline \\[-1.8ex] 
\end{tabular} }
	\end{center}
	{\footnotesize Note: See note under Table \ref{table heating} for a description of the covariates. The three \textit{treatment} indicator variables indicate difference in mean compared to the control group (people who did not see any video).
	\newline *p$<$0.1; **p$<$0.05; ***p$<$0.01}
\end{table}	

\begin{table}[h!]
	\caption{Perceived winners of an emission standards policy}
	\begin{center}
		\scalebox{0.7}{
\begin{tabular}{@{\extracolsep{5pt}}lcccccc} 
\\[-1.8ex]\hline 
\hline \\[-1.8ex] 
 & \multicolumn{6}{c}{Winners of emission limits for cars policy} \\ 
\cline{2-7} 
\\[-1.8ex] & Poorest & Middle class & Richest & Urban & Rural & Own household \\ 
\\[-1.8ex] & (1) & (2) & (3) & (4) & (5) & (6)\\ 
\hline \\[-1.8ex] 
 White only & 0.070 & 0.103 & 0.090 & 0.119 & 0.072 & 0.137 \\ 
  & (0.087) & (0.082) & (0.080) & (0.082) & (0.081) & (0.084) \\ 
  & & & & & & \\ 
 Male & 0.104 & 0.064 & 0.107 & 0.074 & $-$0.055 & 0.075 \\ 
  & (0.076) & (0.071) & (0.070) & (0.072) & (0.071) & (0.073) \\ 
  & & & & & & \\ 
 Children & 0.022 & 0.024 & 0.133$^{*}$ & 0.054 & 0.017 & 0.055 \\ 
  & (0.077) & (0.072) & (0.071) & (0.073) & (0.072) & (0.074) \\ 
  & & & & & & \\ 
 No college & $-$0.074 & $-$0.047 & $-$0.011 & $-$0.085 & $-$0.022 & 0.016 \\ 
  & (0.086) & (0.081) & (0.079) & (0.082) & (0.081) & (0.083) \\ 
  & & & & & & \\ 
 Retired & $-$0.060 & 0.093 & 0.092 & 0.321$^{**}$ & 0.222$^{*}$ & 0.035 \\ 
  & (0.136) & (0.127) & (0.125) & (0.129) & (0.127) & (0.131) \\ 
  & & & & & & \\ 
 Student & $-$0.741$^{**}$ & $-$0.019 & 0.096 & $-$0.121 & $-$0.205 & $-$0.471 \\ 
  & (0.323) & (0.304) & (0.298) & (0.307) & (0.304) & (0.312) \\ 
  & & & & & & \\ 
 Working & $-$0.113 & $-$0.049 & 0.104 & 0.188 & 0.155 & $-$0.053 \\ 
  & (0.134) & (0.126) & (0.123) & (0.127) & (0.126) & (0.130) \\ 
  & & & & & & \\ 
 Income Q2 & $-$0.001 & $-$0.092 & 0.023 & 0.056 & 0.042 & $-$0.064 \\ 
  & (0.114) & (0.107) & (0.105) & (0.109) & (0.107) & (0.110) \\ 
  & & & & & & \\ 
 Income Q3 & $-$0.062 & $-$0.035 & $-$0.028 & 0.074 & 0.122 & $-$0.022 \\ 
  & (0.109) & (0.102) & (0.100) & (0.103) & (0.102) & (0.105) \\ 
  & & & & & & \\ 
 Income Q4 & 0.014 & $-$0.043 & 0.006 & 0.093 & 0.073 & $-$0.014 \\ 
  & (0.115) & (0.108) & (0.106) & (0.109) & (0.108) & (0.111) \\ 
  & & & & & & \\ 
 30-49 & $-$0.293 & $-$0.159 & $-$0.179 & $-$0.211 & $-$0.064 & $-$0.233 \\ 
  & (0.187) & (0.176) & (0.173) & (0.178) & (0.176) & (0.181) \\ 
  & & & & & & \\ 
 50-87 & $-$0.472$^{**}$ & $-$0.476$^{***}$ & $-$0.432$^{**}$ & $-$0.342$^{*}$ & $-$0.277 & $-$0.478$^{**}$ \\ 
  & (0.193) & (0.182) & (0.178) & (0.183) & (0.181) & (0.187) \\ 
  & & & & & & \\ 
 Non voting & $-$0.0002 & $-$0.261$^{**}$ & $-$0.279$^{**}$ & $-$0.255$^{**}$ & $-$0.067 & $-$0.162 \\ 
  & (0.121) & (0.113) & (0.111) & (0.114) & (0.113) & (0.116) \\ 
  & & & & & & \\ 
 Other & $-$0.100 & $-$0.192 & $-$0.352$^{**}$ & $-$0.157 & $-$0.085 & $-$0.233 \\ 
  & (0.176) & (0.165) & (0.162) & (0.167) & (0.165) & (0.170) \\ 
  & & & & & & \\ 
 Trump & $-$0.049 & $-$0.180$^{**}$ & $-$0.151$^{**}$ & $-$0.237$^{***}$ & $-$0.142$^{*}$ & $-$0.177$^{**}$ \\ 
  & (0.080) & (0.075) & (0.074) & (0.076) & (0.075) & (0.078) \\ 
  & & & & & & \\ 
 Both & 0.013 & 0.049 & $-$0.047 & 0.018 & 0.053 & 0.062 \\ 
  & (0.107) & (0.101) & (0.099) & (0.102) & (0.101) & (0.104) \\ 
  & & & & & & \\ 
 Climate treatment only & $-$0.033 & 0.025 & $-$0.084 & $-$0.031 & 0.097 & 0.063 \\ 
  & (0.101) & (0.095) & (0.093) & (0.096) & (0.095) & (0.098) \\ 
  & & & & & & \\ 
 Policy treatment only & 0.206$^{**}$ & 0.041 & $-$0.222$^{**}$ & 0.041 & 0.050 & 0.113 \\ 
  & (0.093) & (0.087) & (0.085) & (0.088) & (0.087) & (0.090) \\ 
  & & & & & & \\ 
 Constant & 0.685$^{***}$ & 0.616$^{***}$ & 0.527$^{**}$ & 0.277 & 0.210 & 0.546$^{**}$ \\ 
  & (0.232) & (0.218) & (0.213) & (0.220) & (0.218) & (0.224) \\ 
  & & & & & & \\ 
\hline \\[-1.8ex] 
Mean & 0.338 & 0.282 & 0.297 & 0.313 & 0.251 & 0.292 \\ 
Observations & 191 & 191 & 191 & 191 & 191 & 191 \\ 
\hline 
\hline \\[-1.8ex] 
\textit{Note:}  & \multicolumn{6}{r}{$^{*}$p$<$0.1; $^{**}$p$<$0.05; $^{***}$p$<$0.01} \\ 
\end{tabular} 
}
	\end{center}
	{\footnotesize Note: See notes under Table \ref{table heating} and Table \ref{table standard opinion} for a description of the covariates.
	\newline *p$<$0.1; **p$<$0.05; ***p$<$0.01}
\end{table}	

\begin{table}[h!]
	\caption{Perceived losers of an emission standards policy}
	\begin{center}
		\scalebox{0.7}{
\begin{tabular}{@{\extracolsep{5pt}}lcccccc} 
\\[-1.8ex]\hline 
\hline \\[-1.8ex] 
 & \multicolumn{6}{c}{Losers of emission limits for cars policy} \\ 
\cline{2-7} 
\\[-1.8ex] & Poorest & Middle class & Richest & Urban & Rural & Own household \\ 
\\[-1.8ex] & (1) & (2) & (3) & (4) & (5) & (6)\\ 
\hline \\[-1.8ex] 
 White only & 0.108 & $-$0.008 & $-$0.016 & 0.030 & $-$0.019 & $-$0.015 \\ 
  & (0.086) & (0.090) & (0.084) & (0.084) & (0.085) & (0.074) \\ 
  & & & & & & \\ 
 Male & 0.066 & 0.030 & 0.061 & 0.078 & 0.116 & 0.048 \\ 
  & (0.075) & (0.079) & (0.073) & (0.073) & (0.074) & (0.065) \\ 
  & & & & & & \\ 
 Children & 0.076 & 0.092 & 0.009 & 0.024 & 0.078 & 0.211$^{***}$ \\ 
  & (0.076) & (0.080) & (0.074) & (0.074) & (0.075) & (0.066) \\ 
  & & & & & & \\ 
 No college & 0.075 & 0.097 & 0.092 & $-$0.006 & 0.068 & $-$0.019 \\ 
  & (0.085) & (0.089) & (0.083) & (0.083) & (0.084) & (0.073) \\ 
  & & & & & & \\ 
 Retired & $-$0.035 & $-$0.062 & $-$0.084 & 0.069 & $-$0.084 & $-$0.216$^{*}$ \\ 
  & (0.134) & (0.141) & (0.131) & (0.131) & (0.133) & (0.116) \\ 
  & & & & & & \\ 
 Student & 0.480 & $-$0.022 & 0.182 & 0.423 & 0.287 & 0.756$^{***}$ \\ 
  & (0.320) & (0.336) & (0.312) & (0.313) & (0.317) & (0.276) \\ 
  & & & & & & \\ 
 Working & 0.001 & $-$0.042 & $-$0.229$^{*}$ & 0.112 & $-$0.132 & $-$0.129 \\ 
  & (0.133) & (0.139) & (0.129) & (0.130) & (0.131) & (0.114) \\ 
  & & & & & & \\ 
 Income Q2 & 0.160 & 0.203$^{*}$ & $-$0.018 & 0.024 & 0.147 & 0.141 \\ 
  & (0.113) & (0.119) & (0.110) & (0.111) & (0.112) & (0.097) \\ 
  & & & & & & \\ 
 Income Q3 & 0.134 & 0.103 & 0.116 & 0.012 & 0.085 & 0.044 \\ 
  & (0.107) & (0.113) & (0.105) & (0.105) & (0.106) & (0.093) \\ 
  & & & & & & \\ 
 Income Q4 & 0.084 & 0.205$^{*}$ & 0.070 & 0.058 & 0.214$^{*}$ & 0.046 \\ 
  & (0.114) & (0.120) & (0.111) & (0.111) & (0.113) & (0.098) \\ 
  & & & & & & \\ 
 30-49 & 0.087 & 0.131 & 0.187 & $-$0.053 & $-$0.030 & 0.053 \\ 
  & (0.186) & (0.195) & (0.181) & (0.181) & (0.183) & (0.160) \\ 
  & & & & & & \\ 
 50-87 & $-$0.009 & 0.121 & 0.029 & $-$0.114 & $-$0.163 & 0.133 \\ 
  & (0.191) & (0.201) & (0.186) & (0.187) & (0.189) & (0.165) \\ 
  & & & & & & \\ 
 Non voting & $-$0.089 & 0.122 & 0.086 & 0.159 & $-$0.122 & 0.017 \\ 
  & (0.119) & (0.125) & (0.116) & (0.117) & (0.118) & (0.103) \\ 
  & & & & & & \\ 
 Other & $-$0.123 & $-$0.017 & $-$0.107 & $-$0.175 & $-$0.113 & $-$0.164 \\ 
  & (0.174) & (0.183) & (0.169) & (0.170) & (0.172) & (0.150) \\ 
  & & & & & & \\ 
 Trump & 0.207$^{***}$ & 0.252$^{***}$ & 0.171$^{**}$ & 0.187$^{**}$ & 0.221$^{***}$ & 0.320$^{***}$ \\ 
  & (0.079) & (0.083) & (0.077) & (0.078) & (0.079) & (0.068) \\ 
  & & & & & & \\ 
 Climate treatment only & 0.044 & $-$0.016 & 0.041 & 0.035 & $-$0.095 & 0.067 \\ 
  & (0.108) & (0.113) & (0.105) & (0.105) & (0.107) & (0.093) \\ 
  & & & & & & \\ 
 No treatment & $-$0.036 & $-$0.011 & $-$0.043 & $-$0.051 & $-$0.145 & $-$0.027 \\ 
  & (0.106) & (0.112) & (0.103) & (0.104) & (0.105) & (0.091) \\ 
  & & & & & & \\ 
 Policy treatment only & $-$0.085 & $-$0.125 & 0.246$^{***}$ & $-$0.062 & $-$0.168$^{*}$ & 0.0003 \\ 
  & (0.097) & (0.102) & (0.094) & (0.094) & (0.096) & (0.083) \\ 
  & & & & & & \\ 
 Constant & $-$0.020 & 0.0004 & 0.100 & 0.105 & 0.319 & $-$0.038 \\ 
  & (0.232) & (0.244) & (0.226) & (0.227) & (0.230) & (0.200) \\ 
  & & & & & & \\ 
\hline \\[-1.8ex] 
Mean &  &  &  &  &  &  \\ 
Observations & 191 & 191 & 191 & 191 & 191 & 191 \\ 
\hline 
\hline \\[-1.8ex] 
\textit{Note:}  & \multicolumn{6}{r}{$^{*}$p$<$0.1; $^{**}$p$<$0.05; $^{***}$p$<$0.01} \\ 
\end{tabular} 
}
	\end{center}
	{\footnotesize Note: See notes under Table \ref{table heating} and Table \ref{table standard opinion} for a description of the covariates.
	\newline *p$<$0.1; **p$<$0.05; ***p$<$0.01}
\end{table}	

\clearpage
\subsection{Preferences 2: Green investments}

\begin{table}[h!]
	\caption{Opinion on green investments}
	\begin{center}
		\scalebox{0.7}{
\begin{tabular}{@{\extracolsep{5pt}}lccccc} 
\\[-1.8ex]\hline 
\hline \\[-1.8ex] 
 & \multicolumn{5}{c}{C02 emission limit for cars policy in the U.S.} \\ 
\cline{2-6} 
\\[-1.8ex] & Trust federal gov. & Effective & Positive impact on jobs & Positive side effects & Support \\ 
\\[-1.8ex] & (1) & (2) & (3) & (4) & (5)\\ 
\hline \\[-1.8ex] 
 White only & $-$0.121 & 0.024 & $-$0.021 & $-$0.006 & 0.084 \\ 
  & (0.085) & (0.084) & (0.086) & (0.091) & (0.086) \\ 
  & & & & & \\ 
 Male & 0.050 & 0.085 & 0.138$^{*}$ & 0.021 & 0.097 \\ 
  & (0.074) & (0.073) & (0.075) & (0.079) & (0.075) \\ 
  & & & & & \\ 
 Children & 0.066 & 0.076 & 0.103 & 0.037 & 0.075 \\ 
  & (0.075) & (0.074) & (0.076) & (0.080) & (0.076) \\ 
  & & & & & \\ 
 No college & 0.036 & $-$0.122 & $-$0.094 & $-$0.101 & $-$0.056 \\ 
  & (0.084) & (0.083) & (0.085) & (0.090) & (0.085) \\ 
  & & & & & \\ 
 Retired & $-$0.061 & $-$0.098 & 0.084 & 0.092 & $-$0.063 \\ 
  & (0.132) & (0.131) & (0.135) & (0.142) & (0.134) \\ 
  & & & & & \\ 
 Student & $-$0.723$^{**}$ & $-$0.428 & $-$0.580$^{*}$ & $-$0.492 & $-$0.486 \\ 
  & (0.315) & (0.312) & (0.322) & (0.338) & (0.319) \\ 
  & & & & & \\ 
 Working & $-$0.034 & $-$0.118 & 0.113 & 0.053 & $-$0.087 \\ 
  & (0.131) & (0.129) & (0.133) & (0.140) & (0.132) \\ 
  & & & & & \\ 
 Income Q2 & 0.194$^{*}$ & 0.021 & 0.082 & 0.026 & 0.080 \\ 
  & (0.111) & (0.110) & (0.114) & (0.119) & (0.113) \\ 
  & & & & & \\ 
 Income Q3 & 0.179$^{*}$ & $-$0.047 & 0.027 & $-$0.107 & $-$0.026 \\ 
  & (0.106) & (0.105) & (0.108) & (0.113) & (0.107) \\ 
  & & & & & \\ 
 Income Q4 & 0.218$^{*}$ & 0.054 & 0.010 & 0.049 & 0.031 \\ 
  & (0.112) & (0.111) & (0.115) & (0.120) & (0.114) \\ 
  & & & & & \\ 
 30-49 & 0.121 & $-$0.402$^{**}$ & $-$0.154 & $-$0.071 & $-$0.357$^{*}$ \\ 
  & (0.183) & (0.181) & (0.187) & (0.196) & (0.185) \\ 
  & & & & & \\ 
 50-87 & $-$0.186 & $-$0.497$^{***}$ & $-$0.342$^{*}$ & $-$0.187 & $-$0.476$^{**}$ \\ 
  & (0.188) & (0.186) & (0.192) & (0.202) & (0.191) \\ 
  & & & & & \\ 
 Non voting & $-$0.066 & $-$0.458$^{***}$ & $-$0.250$^{**}$ & $-$0.316$^{**}$ & $-$0.365$^{***}$ \\ 
  & (0.118) & (0.116) & (0.120) & (0.126) & (0.119) \\ 
  & & & & & \\ 
 Other & $-$0.027 & 0.032 & $-$0.099 & $-$0.092 & $-$0.107 \\ 
  & (0.171) & (0.170) & (0.175) & (0.183) & (0.173) \\ 
  & & & & & \\ 
 Trump & $-$0.221$^{***}$ & $-$0.394$^{***}$ & $-$0.327$^{***}$ & $-$0.271$^{***}$ & $-$0.444$^{***}$ \\ 
  & (0.078) & (0.077) & (0.080) & (0.084) & (0.079) \\ 
  & & & & & \\ 
 Both treatments & 0.255$^{**}$ & 0.179$^{*}$ & 0.033 & 0.125 & 0.063 \\ 
  & (0.105) & (0.104) & (0.107) & (0.112) & (0.106) \\ 
  & & & & & \\ 
 Climate treatment only & 0.055 & 0.057 & $-$0.024 & 0.032 & 0.064 \\ 
  & (0.099) & (0.098) & (0.101) & (0.106) & (0.100) \\ 
  & & & & & \\ 
 Policy treatment only & 0.034 & 0.038 & $-$0.075 & 0.119 & $-$0.037 \\ 
  & (0.090) & (0.090) & (0.092) & (0.097) & (0.092) \\ 
  & & & & & \\ 
 Constant & 0.411$^{*}$ & 1.136$^{***}$ & 0.710$^{***}$ & 0.653$^{***}$ & 1.052$^{***}$ \\ 
  & (0.226) & (0.224) & (0.231) & (0.242) & (0.229) \\ 
  & & & & & \\ 
\hline \\[-1.8ex] 
Control group mean & 0.354 & 0.521 & 0.542 & 0.438 & 0.562 \\ 
Observations & 191 & 191 & 191 & 191 & 191 \\ 
\hline 
\hline \\[-1.8ex] 
\textit{Note:}  & \multicolumn{5}{r}{$^{*}$p$<$0.1; $^{**}$p$<$0.05; $^{***}$p$<$0.01} \\ 
\end{tabular} 
}
	\end{center}
	{\footnotesize Note: See notes under Table \ref{table heating} and Table \ref{table standard opinion} for a description of the covariates.
	\newline *p$<$0.1; **p$<$0.05; ***p$<$0.01}
\end{table}	

\begin{table}[h!]
	\caption{Perceived winners of a green investments policy}
	\begin{center}
		\scalebox{0.7}{
\begin{tabular}{@{\extracolsep{5pt}}lcccccc} 
\\[-1.8ex]\hline 
\hline \\[-1.8ex] 
 & \multicolumn{6}{c}{Winners of emission limits for cars policy} \\ 
\cline{2-7} 
\\[-1.8ex] & Poorest & Middle class & Richest & Urban & Rural & Own household \\ 
\\[-1.8ex] & (1) & (2) & (3) & (4) & (5) & (6)\\ 
\hline \\[-1.8ex] 
 White only & $-$0.067 & 0.037 & $-$0.014 & 0.060 & $-$0.025 & 0.137$^{*}$ \\ 
  & (0.085) & (0.083) & (0.083) & (0.082) & (0.084) & (0.077) \\ 
  & & & & & & \\ 
 Male & 0.135$^{*}$ & 0.148$^{**}$ & 0.077 & 0.042 & $-$0.026 & 0.084 \\ 
  & (0.074) & (0.072) & (0.072) & (0.072) & (0.074) & (0.067) \\ 
  & & & & & & \\ 
 Children & 0.066 & 0.060 & 0.129$^{*}$ & 0.072 & 0.058 & 0.074 \\ 
  & (0.075) & (0.073) & (0.074) & (0.073) & (0.075) & (0.068) \\ 
  & & & & & & \\ 
 No college & $-$0.055 & 0.007 & 0.002 & $-$0.057 & 0.026 & 0.081 \\ 
  & (0.084) & (0.082) & (0.082) & (0.082) & (0.083) & (0.076) \\ 
  & & & & & & \\ 
 Retired & $-$0.107 & 0.095 & 0.048 & 0.263$^{**}$ & 0.162 & 0.010 \\ 
  & (0.132) & (0.129) & (0.130) & (0.129) & (0.132) & (0.121) \\ 
  & & & & & & \\ 
 Student & $-$0.249 & $-$0.430 & $-$0.407 & $-$0.186 & $-$0.078 & $-$0.531$^{*}$ \\ 
  & (0.316) & (0.308) & (0.309) & (0.307) & (0.314) & (0.288) \\ 
  & & & & & & \\ 
 Working & $-$0.143 & 0.041 & 0.016 & 0.168 & 0.141 & 0.003 \\ 
  & (0.131) & (0.128) & (0.128) & (0.127) & (0.130) & (0.119) \\ 
  & & & & & & \\ 
 Income Q2 & $-$0.032 & $-$0.189$^{*}$ & 0.030 & 0.064 & $-$0.081 & 0.013 \\ 
  & (0.112) & (0.109) & (0.109) & (0.109) & (0.111) & (0.102) \\ 
  & & & & & & \\ 
 Income Q3 & $-$0.117 & $-$0.044 & $-$0.020 & 0.078 & $-$0.027 & 0.030 \\ 
  & (0.106) & (0.103) & (0.104) & (0.103) & (0.106) & (0.097) \\ 
  & & & & & & \\ 
 Income Q4 & 0.021 & $-$0.030 & 0.087 & 0.109 & 0.023 & 0.155 \\ 
  & (0.113) & (0.110) & (0.110) & (0.109) & (0.112) & (0.103) \\ 
  & & & & & & \\ 
 30-49 & 0.076 & $-$0.267 & $-$0.192 & $-$0.031 & 0.291 & $-$0.278$^{*}$ \\ 
  & (0.183) & (0.178) & (0.179) & (0.178) & (0.182) & (0.167) \\ 
  & & & & & & \\ 
 50-87 & $-$0.209 & $-$0.541$^{***}$ & $-$0.376$^{**}$ & $-$0.296 & 0.002 & $-$0.564$^{***}$ \\ 
  & (0.189) & (0.184) & (0.185) & (0.184) & (0.188) & (0.172) \\ 
  & & & & & & \\ 
 Non voting & $-$0.361$^{***}$ & $-$0.210$^{*}$ & $-$0.219$^{*}$ & $-$0.195$^{*}$ & $-$0.192 & $-$0.272$^{**}$ \\ 
  & (0.118) & (0.115) & (0.115) & (0.115) & (0.117) & (0.107) \\ 
  & & & & & & \\ 
 Other & $-$0.206 & $-$0.334$^{**}$ & $-$0.397$^{**}$ & $-$0.093 & $-$0.176 & $-$0.320$^{**}$ \\ 
  & (0.172) & (0.167) & (0.168) & (0.167) & (0.171) & (0.156) \\ 
  & & & & & & \\ 
 Trump & $-$0.157$^{**}$ & $-$0.258$^{***}$ & $-$0.214$^{***}$ & $-$0.252$^{***}$ & $-$0.156$^{**}$ & $-$0.342$^{***}$ \\ 
  & (0.078) & (0.076) & (0.077) & (0.076) & (0.078) & (0.071) \\ 
  & & & & & & \\ 
 Both treatments & $-$0.027 & 0.008 & 0.011 & 0.048 & 0.032 & 0.037 \\ 
  & (0.105) & (0.102) & (0.103) & (0.102) & (0.104) & (0.096) \\ 
  & & & & & & \\ 
 Climate treatment only & 0.018 & $-$0.052 & $-$0.039 & 0.055 & 0.047 & 0.027 \\ 
  & (0.099) & (0.096) & (0.097) & (0.096) & (0.098) & (0.090) \\ 
  & & & & & & \\ 
 Policy treatment only & 0.145 & $-$0.043 & $-$0.198$^{**}$ & $-$0.071 & 0.005 & 0.020 \\ 
  & (0.091) & (0.088) & (0.089) & (0.088) & (0.090) & (0.083) \\ 
  & & & & & & \\ 
 Constant & 0.651$^{***}$ & 0.771$^{***}$ & 0.636$^{***}$ & 0.278 & 0.152 & 0.635$^{***}$ \\ 
  & (0.226) & (0.220) & (0.222) & (0.220) & (0.225) & (0.206) \\ 
  & & & & & & \\ 
\hline \\[-1.8ex] 
Mean & 0.39 & 0.349 & 0.313 & 0.318 & 0.303 & 0.328 \\ 
Observations & 191 & 191 & 191 & 191 & 191 & 191 \\ 
\hline 
\hline \\[-1.8ex] 
\textit{Note:}  & \multicolumn{6}{r}{$^{*}$p$<$0.1; $^{**}$p$<$0.05; $^{***}$p$<$0.01} \\ 
\end{tabular} 
}
	\end{center}
	{\footnotesize Note: See notes under Table \ref{table heating} and Table \ref{table standard opinion} for a description of the covariates.
	\newline *p$<$0.1; **p$<$0.05; ***p$<$0.01}
\end{table}	

\begin{table}[h!]
	\caption{Perceived losers of a green investments policy}
	\begin{center}
		\scalebox{0.7}{
\begin{tabular}{@{\extracolsep{5pt}}lcccccc} 
\\[-1.8ex]\hline 
\hline \\[-1.8ex] 
 & \multicolumn{6}{c}{Losers of emission limits for cars policy} \\ 
\cline{2-7} 
\\[-1.8ex] & Poorest & Middle class & Richest & Urban & Rural & Own household \\ 
\\[-1.8ex] & (1) & (2) & (3) & (4) & (5) & (6)\\ 
\hline \\[-1.8ex] 
 White only & 0.070 & $-$0.022 & 0.025 & $-$0.034 & 0.049 & $-$0.013 \\ 
  & (0.082) & (0.078) & (0.079) & (0.080) & (0.079) & (0.068) \\ 
  & & & & & & \\ 
 Male & 0.037 & $-$0.003 & 0.084 & 0.026 & 0.037 & $-$0.037 \\ 
  & (0.072) & (0.068) & (0.069) & (0.070) & (0.069) & (0.060) \\ 
  & & & & & & \\ 
 Children & 0.118 & 0.048 & 0.039 & $-$0.007 & 0.063 & 0.060 \\ 
  & (0.073) & (0.069) & (0.070) & (0.071) & (0.070) & (0.060) \\ 
  & & & & & & \\ 
 No college & 0.047 & $-$0.013 & 0.158$^{**}$ & 0.009 & 0.106 & 0.019 \\ 
  & (0.081) & (0.077) & (0.078) & (0.079) & (0.079) & (0.068) \\ 
  & & & & & & \\ 
 Retired & $-$0.039 & $-$0.229$^{*}$ & $-$0.005 & $-$0.110 & $-$0.109 & $-$0.013 \\ 
  & (0.128) & (0.121) & (0.124) & (0.125) & (0.124) & (0.107) \\ 
  & & & & & & \\ 
 Student & 0.520$^{*}$ & 0.762$^{***}$ & $-$0.012 & 0.751$^{**}$ & 0.291 & 0.539$^{**}$ \\ 
  & (0.306) & (0.290) & (0.295) & (0.297) & (0.296) & (0.254) \\ 
  & & & & & & \\ 
 Working & 0.035 & $-$0.175 & 0.014 & $-$0.011 & $-$0.180 & $-$0.004 \\ 
  & (0.127) & (0.120) & (0.122) & (0.123) & (0.123) & (0.105) \\ 
  & & & & & & \\ 
 Income Q2 & 0.081 & 0.163 & 0.095 & $-$0.036 & 0.169 & 0.118 \\ 
  & (0.108) & (0.102) & (0.104) & (0.105) & (0.105) & (0.090) \\ 
  & & & & & & \\ 
 Income Q3 & 0.066 & 0.099 & 0.167$^{*}$ & $-$0.034 & 0.204$^{**}$ & 0.051 \\ 
  & (0.103) & (0.097) & (0.099) & (0.100) & (0.099) & (0.085) \\ 
  & & & & & & \\ 
 Income Q4 & $-$0.053 & 0.137 & 0.139 & 0.005 & 0.243$^{**}$ & 0.097 \\ 
  & (0.109) & (0.103) & (0.105) & (0.106) & (0.105) & (0.091) \\ 
  & & & & & & \\ 
 30-49 & 0.069 & 0.094 & 0.005 & $-$0.139 & $-$0.098 & 0.152 \\ 
  & (0.177) & (0.168) & (0.171) & (0.172) & (0.172) & (0.147) \\ 
  & & & & & & \\ 
 50-87 & 0.076 & 0.201 & $-$0.059 & $-$0.089 & $-$0.141 & 0.178 \\ 
  & (0.183) & (0.173) & (0.176) & (0.177) & (0.177) & (0.152) \\ 
  & & & & & & \\ 
 Non voting & 0.073 & $-$0.007 & 0.087 & $-$0.085 & 0.018 & 0.115 \\ 
  & (0.114) & (0.108) & (0.110) & (0.111) & (0.110) & (0.095) \\ 
  & & & & & & \\ 
 Other & $-$0.207 & $-$0.034 & $-$0.065 & $-$0.176 & 0.097 & $-$0.045 \\ 
  & (0.166) & (0.157) & (0.160) & (0.161) & (0.161) & (0.138) \\ 
  & & & & & & \\ 
 Trump & 0.233$^{***}$ & 0.308$^{***}$ & 0.192$^{***}$ & 0.258$^{***}$ & 0.220$^{***}$ & 0.344$^{***}$ \\ 
  & (0.076) & (0.072) & (0.073) & (0.074) & (0.073) & (0.063) \\ 
  & & & & & & \\ 
 Both & $-$0.041 & $-$0.046 & 0.035 & $-$0.022 & 0.067 & $-$0.089 \\ 
  & (0.102) & (0.096) & (0.098) & (0.099) & (0.098) & (0.084) \\ 
  & & & & & & \\ 
 Climate treatment only & 0.023 & 0.020 & 0.081 & 0.070 & 0.123 & 0.082 \\ 
  & (0.096) & (0.091) & (0.092) & (0.093) & (0.093) & (0.080) \\ 
  & & & & & & \\ 
 Policy treatment only & $-$0.101 & $-$0.082 & 0.223$^{***}$ & 0.022 & $-$0.005 & 0.060 \\ 
  & (0.088) & (0.083) & (0.085) & (0.085) & (0.085) & (0.073) \\ 
  & & & & & & \\ 
 Constant & $-$0.042 & 0.091 & $-$0.159 & 0.329 & 0.073 & $-$0.193 \\ 
  & (0.219) & (0.208) & (0.211) & (0.213) & (0.212) & (0.182) \\ 
  & & & & & & \\ 
\hline \\[-1.8ex] 
Mean & 0.267 & 0.256 & 0.236 & 0.246 & 0.236 & 0.195 \\ 
Observations & 191 & 191 & 191 & 191 & 191 & 191 \\ 
\hline 
\hline \\[-1.8ex] 
\textit{Note:}  & \multicolumn{6}{r}{$^{*}$p$<$0.1; $^{**}$p$<$0.05; $^{***}$p$<$0.01} \\ 
\end{tabular} 
}
	\end{center}
	{\footnotesize Note: See notes under Table \ref{table heating} and Table \ref{table standard opinion} for a description of the covariates.
	\newline *p$<$0.1; **p$<$0.05; ***p$<$0.01}
\end{table}	

\clearpage
\subsection{Preferences 3: Tax and dividend}

\begin{table}[h!]
	\caption{Opinion on carbon tax with cash transfers}
	\begin{center}
		\scalebox{0.7}{
\begin{tabular}{@{\extracolsep{5pt}}lccccc} 
\\[-1.8ex]\hline 
\hline \\[-1.8ex] 
 & \multicolumn{5}{c}{C02 emission limit for cars policy in the U.S.} \\ 
\cline{2-6} 
\\[-1.8ex] & Trust federal gov. & Effective & Positive impact on jobs & Positive side effects & Support \\ 
\\[-1.8ex] & (1) & (2) & (3) & (4) & (5)\\ 
\hline \\[-1.8ex] 
 White only & 0.077 & 0.077 & 0.089 & 0.142$^{*}$ & 0.107 \\ 
  & (0.078) & (0.084) & (0.084) & (0.085) & (0.085) \\ 
  & & & & & \\ 
 Male & 0.087 & 0.052 & 0.048 & $-$0.026 & 0.040 \\ 
  & (0.068) & (0.073) & (0.074) & (0.074) & (0.074) \\ 
  & & & & & \\ 
 Children & 0.090 & 0.010 & 0.027 & $-$0.037 & 0.071 \\ 
  & (0.069) & (0.074) & (0.075) & (0.075) & (0.075) \\ 
  & & & & & \\ 
 No college & 0.152$^{*}$ & 0.040 & 0.113 & 0.001 & $-$0.044 \\ 
  & (0.077) & (0.083) & (0.084) & (0.084) & (0.084) \\ 
  & & & & & \\ 
 Retired & $-$0.091 & $-$0.059 & $-$0.042 & $-$0.060 & $-$0.275$^{**}$ \\ 
  & (0.122) & (0.131) & (0.132) & (0.133) & (0.132) \\ 
  & & & & & \\ 
 Student & $-$0.442 & $-$0.727$^{**}$ & $-$0.253 & $-$0.315 & $-$0.769$^{**}$ \\ 
  & (0.291) & (0.313) & (0.314) & (0.316) & (0.316) \\ 
  & & & & & \\ 
 Working & $-$0.007 & $-$0.102 & 0.016 & $-$0.095 & $-$0.252$^{*}$ \\ 
  & (0.121) & (0.130) & (0.130) & (0.131) & (0.131) \\ 
  & & & & & \\ 
 Income Q2 & 0.068 & 0.193$^{*}$ & 0.040 & $-$0.016 & 0.127 \\ 
  & (0.103) & (0.111) & (0.111) & (0.112) & (0.112) \\ 
  & & & & & \\ 
 Income Q3 & 0.110 & 0.086 & 0.036 & 0.054 & 0.102 \\ 
  & (0.098) & (0.105) & (0.106) & (0.106) & (0.106) \\ 
  & & & & & \\ 
 Income Q4 & 0.076 & 0.044 & 0.020 & 0.090 & 0.120 \\ 
  & (0.104) & (0.111) & (0.112) & (0.113) & (0.113) \\ 
  & & & & & \\ 
 30-49 & $-$0.397$^{**}$ & 0.109 & $-$0.226 & $-$0.232 & $-$0.0004 \\ 
  & (0.169) & (0.181) & (0.182) & (0.183) & (0.183) \\ 
  & & & & & \\ 
 50-87 & $-$0.732$^{***}$ & $-$0.220 & $-$0.503$^{***}$ & $-$0.527$^{***}$ & $-$0.252 \\ 
  & (0.174) & (0.187) & (0.188) & (0.189) & (0.189) \\ 
  & & & & & \\ 
 Non voting & $-$0.415$^{***}$ & $-$0.265$^{**}$ & $-$0.372$^{***}$ & $-$0.285$^{**}$ & $-$0.141 \\ 
  & (0.109) & (0.117) & (0.117) & (0.118) & (0.118) \\ 
  & & & & & \\ 
 Other & $-$0.123 & $-$0.380$^{**}$ & $-$0.178 & $-$0.214 & $-$0.361$^{**}$ \\ 
  & (0.158) & (0.170) & (0.171) & (0.172) & (0.172) \\ 
  & & & & & \\ 
 Trump & $-$0.362$^{***}$ & $-$0.324$^{***}$ & $-$0.256$^{***}$ & $-$0.304$^{***}$ & $-$0.266$^{***}$ \\ 
  & (0.072) & (0.078) & (0.078) & (0.078) & (0.078) \\ 
  & & & & & \\ 
 Both treatments & 0.115 & 0.103 & 0.012 & 0.065 & 0.142 \\ 
  & (0.097) & (0.104) & (0.104) & (0.105) & (0.105) \\ 
  & & & & & \\ 
 Climate treatment only & 0.146 & 0.023 & 0.031 & $-$0.041 & $-$0.026 \\ 
  & (0.091) & (0.098) & (0.098) & (0.099) & (0.099) \\ 
  & & & & & \\ 
 Policy treatment only & $-$0.00003 & 0.140 & 0.088 & 0.119 & 0.074 \\ 
  & (0.084) & (0.090) & (0.090) & (0.091) & (0.091) \\ 
  & & & & & \\ 
 Constant & 0.879$^{***}$ & 0.508$^{**}$ & 0.685$^{***}$ & 0.895$^{***}$ & 0.674$^{***}$ \\ 
  & (0.209) & (0.224) & (0.225) & (0.227) & (0.226) \\ 
  & & & & & \\ 
\hline \\[-1.8ex] 
Control group mean & 0.396 & 0.375 & 0.354 & 0.396 & 0.396 \\ 
Observations & 191 & 191 & 191 & 191 & 191 \\ 
\hline 
\hline \\[-1.8ex] 
\textit{Note:}  & \multicolumn{5}{r}{$^{*}$p$<$0.1; $^{**}$p$<$0.05; $^{***}$p$<$0.01} \\ 
\end{tabular} 
}
	\end{center}
	{\footnotesize Note: See notes under Table \ref{table heating} and Table \ref{table standard opinion} for a description of the covariates.
	\newline *p$<$0.1; **p$<$0.05; ***p$<$0.01}
\end{table}	

\begin{table}[h!]
	\caption{Perceived winners of a carbon tax with cash transfers policy}
	\begin{center}
		\scalebox{0.7}{
\begin{tabular}{@{\extracolsep{5pt}}lcccccc} 
\\[-1.8ex]\hline 
\hline \\[-1.8ex] 
 & \multicolumn{6}{c}{Winners of emission limits for cars policy} \\ 
\cline{2-7} 
\\[-1.8ex] & Poorest & Middle class & Richest & Urban & Rural & Own household \\ 
\\[-1.8ex] & (1) & (2) & (3) & (4) & (5) & (6)\\ 
\hline \\[-1.8ex] 
 White only & 0.063 & 0.017 & 0.080 & 0.110 & 0.052 & 0.147$^{*}$ \\ 
  & (0.086) & (0.076) & (0.075) & (0.075) & (0.077) & (0.076) \\ 
  & & & & & & \\ 
 Male & 0.106 & 0.023 & 0.088 & 0.062 & $-$0.013 & 0.097 \\ 
  & (0.075) & (0.066) & (0.065) & (0.066) & (0.067) & (0.066) \\ 
  & & & & & & \\ 
 Children & $-$0.021 & $-$0.083 & 0.013 & 0.039 & $-$0.006 & 0.034 \\ 
  & (0.076) & (0.067) & (0.066) & (0.067) & (0.068) & (0.067) \\ 
  & & & & & & \\ 
 No college & 0.071 & 0.066 & 0.100 & 0.081 & 0.101 & 0.145$^{*}$ \\ 
  & (0.085) & (0.075) & (0.074) & (0.074) & (0.076) & (0.075) \\ 
  & & & & & & \\ 
 Retired & $-$0.272$^{**}$ & $-$0.115 & 0.165 & $-$0.060 & 0.026 & $-$0.198$^{*}$ \\ 
  & (0.134) & (0.119) & (0.117) & (0.117) & (0.120) & (0.119) \\ 
  & & & & & & \\ 
 Student & $-$0.569$^{*}$ & $-$0.162 & $-$0.336 & $-$0.347 & $-$0.350 & $-$0.709$^{**}$ \\ 
  & (0.320) & (0.283) & (0.278) & (0.280) & (0.285) & (0.283) \\ 
  & & & & & & \\ 
 Working & $-$0.280$^{**}$ & $-$0.198$^{*}$ & 0.107 & $-$0.035 & $-$0.009 & $-$0.215$^{*}$ \\ 
  & (0.133) & (0.117) & (0.115) & (0.116) & (0.118) & (0.118) \\ 
  & & & & & & \\ 
 Income Q2 & 0.069 & $-$0.002 & $-$0.001 & 0.048 & $-$0.040 & 0.010 \\ 
  & (0.113) & (0.100) & (0.098) & (0.099) & (0.101) & (0.100) \\ 
  & & & & & & \\ 
 Income Q3 & 0.003 & 0.034 & 0.035 & 0.104 & 0.142 & 0.054 \\ 
  & (0.108) & (0.095) & (0.093) & (0.094) & (0.096) & (0.095) \\ 
  & & & & & & \\ 
 Income Q4 & 0.076 & $-$0.030 & 0.173$^{*}$ & 0.155 & 0.056 & 0.089 \\ 
  & (0.114) & (0.101) & (0.099) & (0.100) & (0.102) & (0.101) \\ 
  & & & & & & \\ 
 30-49 & $-$0.152 & $-$0.423$^{**}$ & $-$0.111 & $-$0.147 & $-$0.039 & $-$0.108 \\ 
  & (0.186) & (0.164) & (0.161) & (0.162) & (0.165) & (0.164) \\ 
  & & & & & & \\ 
 50-87 & $-$0.419$^{**}$ & $-$0.833$^{***}$ & $-$0.522$^{***}$ & $-$0.391$^{**}$ & $-$0.388$^{**}$ & $-$0.515$^{***}$ \\ 
  & (0.191) & (0.169) & (0.166) & (0.167) & (0.171) & (0.169) \\ 
  & & & & & & \\ 
 Non voting & $-$0.110 & $-$0.340$^{***}$ & $-$0.164 & $-$0.300$^{***}$ & $-$0.182$^{*}$ & $-$0.167 \\ 
  & (0.119) & (0.106) & (0.104) & (0.104) & (0.106) & (0.106) \\ 
  & & & & & & \\ 
 Other & $-$0.119 & $-$0.093 & $-$0.259$^{*}$ & 0.055 & $-$0.093 & $-$0.083 \\ 
  & (0.174) & (0.154) & (0.151) & (0.152) & (0.155) & (0.154) \\ 
  & & & & & & \\ 
 Trump & $-$0.114 & $-$0.105 & $-$0.140$^{**}$ & $-$0.239$^{***}$ & $-$0.160$^{**}$ & $-$0.192$^{***}$ \\ 
  & (0.079) & (0.070) & (0.069) & (0.069) & (0.071) & (0.070) \\ 
  & & & & & & \\ 
 Climate treatment only & $-$0.031 & $-$0.002 & 0.048 & $-$0.051 & 0.033 & $-$0.064 \\ 
  & (0.108) & (0.095) & (0.094) & (0.094) & (0.096) & (0.095) \\ 
  & & & & & & \\ 
 No treatment & 0.002 & 0.023 & 0.077 & $-$0.025 & 0.042 & $-$0.088 \\ 
  & (0.106) & (0.094) & (0.092) & (0.093) & (0.095) & (0.094) \\ 
  & & & & & & \\ 
 Policy treatment only & 0.116 & 0.0001 & $-$0.093 & $-$0.043 & $-$0.006 & $-$0.006 \\ 
  & (0.097) & (0.086) & (0.084) & (0.085) & (0.086) & (0.086) \\ 
  & & & & & & \\ 
 Constant & 0.780$^{***}$ & 1.167$^{***}$ & 0.399$^{**}$ & 0.484$^{**}$ & 0.461$^{**}$ & 0.699$^{***}$ \\ 
  & (0.233) & (0.206) & (0.202) & (0.203) & (0.207) & (0.206) \\ 
  & & & & & & \\ 
\hline \\[-1.8ex] 
Mean &  &  &  &  &  &  \\ 
Observations & 191 & 191 & 191 & 191 & 191 & 191 \\ 
\hline 
\hline \\[-1.8ex] 
\textit{Note:}  & \multicolumn{6}{r}{$^{*}$p$<$0.1; $^{**}$p$<$0.05; $^{***}$p$<$0.01} \\ 
\end{tabular} 
}
	\end{center}
	{\footnotesize Note: See notes under Table \ref{table heating} and Table \ref{table standard opinion} for a description of the covariates.
	\newline *p$<$0.1; **p$<$0.05; ***p$<$0.01}
\end{table}	

\begin{table}[h!]
	\caption{Perceived losers of a carbon tax with cash transfers policy}
	\begin{center}
		\scalebox{0.7}{
\begin{tabular}{@{\extracolsep{5pt}}lcccccc} 
\\[-1.8ex]\hline 
\hline \\[-1.8ex] 
 & \multicolumn{6}{c}{Losers of emission limits for cars policy} \\ 
\cline{2-7} 
\\[-1.8ex] & Poorest & Middle class & Richest & Urban & Rural & Own household \\ 
\\[-1.8ex] & (1) & (2) & (3) & (4) & (5) & (6)\\ 
\hline \\[-1.8ex] 
 White only & 0.005 & 0.114 & $-$0.038 & 0.057 & 0.058 & $-$0.078 \\ 
  & (0.084) & (0.088) & (0.082) & (0.085) & (0.081) & (0.073) \\ 
  & & & & & & \\ 
 Male & 0.042 & 0.120 & 0.088 & 0.011 & 0.078 & 0.070 \\ 
  & (0.074) & (0.076) & (0.071) & (0.074) & (0.071) & (0.064) \\ 
  & & & & & & \\ 
 Children & 0.026 & 0.114 & 0.043 & 0.045 & 0.033 & 0.051 \\ 
  & (0.075) & (0.078) & (0.072) & (0.075) & (0.072) & (0.065) \\ 
  & & & & & & \\ 
 No college & $-$0.074 & 0.084 & 0.107 & 0.082 & 0.089 & $-$0.019 \\ 
  & (0.083) & (0.087) & (0.081) & (0.084) & (0.080) & (0.073) \\ 
  & & & & & & \\ 
 Retired & 0.047 & $-$0.072 & $-$0.398$^{***}$ & 0.044 & $-$0.068 & $-$0.064 \\ 
  & (0.132) & (0.137) & (0.128) & (0.133) & (0.127) & (0.114) \\ 
  & & & & & & \\ 
 Student & 0.013 & $-$0.152 & $-$0.280 & 0.155 & 0.148 & 0.852$^{***}$ \\ 
  & (0.314) & (0.326) & (0.304) & (0.317) & (0.303) & (0.273) \\ 
  & & & & & & \\ 
 Working & 0.166 & 0.079 & $-$0.317$^{**}$ & 0.035 & $-$0.042 & 0.012 \\ 
  & (0.130) & (0.135) & (0.126) & (0.131) & (0.126) & (0.113) \\ 
  & & & & & & \\ 
 Income Q2 & 0.052 & 0.108 & 0.061 & 0.039 & 0.121 & 0.168$^{*}$ \\ 
  & (0.111) & (0.115) & (0.108) & (0.112) & (0.107) & (0.097) \\ 
  & & & & & & \\ 
 Income Q3 & 0.101 & 0.067 & 0.060 & 0.090 & 0.085 & 0.060 \\ 
  & (0.105) & (0.110) & (0.102) & (0.106) & (0.102) & (0.092) \\ 
  & & & & & & \\ 
 Income Q4 & 0.065 & 0.172 & 0.045 & 0.058 & 0.265$^{**}$ & 0.078 \\ 
  & (0.112) & (0.116) & (0.108) & (0.113) & (0.108) & (0.097) \\ 
  & & & & & & \\ 
 30-49 & 0.028 & 0.069 & $-$0.106 & 0.028 & $-$0.068 & 0.053 \\ 
  & (0.182) & (0.189) & (0.176) & (0.184) & (0.176) & (0.158) \\ 
  & & & & & & \\ 
 50-87 & 0.088 & 0.263 & $-$0.015 & 0.002 & $-$0.005 & 0.166 \\ 
  & (0.188) & (0.195) & (0.182) & (0.189) & (0.181) & (0.163) \\ 
  & & & & & & \\ 
 Non voting & $-$0.102 & 0.146 & $-$0.143 & 0.027 & $-$0.041 & $-$0.039 \\ 
  & (0.117) & (0.122) & (0.113) & (0.118) & (0.113) & (0.102) \\ 
  & & & & & & \\ 
 Other & $-$0.106 & $-$0.274 & $-$0.175 & $-$0.212 & 0.024 & $-$0.152 \\ 
  & (0.171) & (0.177) & (0.165) & (0.172) & (0.165) & (0.148) \\ 
  & & & & & & \\ 
 Trump & 0.201$^{**}$ & 0.222$^{***}$ & 0.265$^{***}$ & 0.224$^{***}$ & 0.301$^{***}$ & 0.403$^{***}$ \\ 
  & (0.078) & (0.081) & (0.076) & (0.079) & (0.075) & (0.068) \\ 
  & & & & & & \\ 
 Climate treatment only & 0.081 & 0.003 & 0.036 & 0.153 & 0.025 & 0.043 \\ 
  & (0.106) & (0.110) & (0.102) & (0.107) & (0.102) & (0.092) \\ 
  & & & & & & \\ 
 No treatment & $-$0.084 & $-$0.097 & $-$0.098 & $-$0.045 & $-$0.160 & $-$0.054 \\ 
  & (0.104) & (0.108) & (0.101) & (0.105) & (0.101) & (0.091) \\ 
  & & & & & & \\ 
 Policy treatment only & $-$0.043 & 0.007 & 0.164$^{*}$ & $-$0.024 & $-$0.113 & $-$0.042 \\ 
  & (0.095) & (0.098) & (0.092) & (0.096) & (0.091) & (0.082) \\ 
  & & & & & & \\ 
 Constant & 0.007 & $-$0.255 & 0.417$^{*}$ & 0.0001 & 0.059 & $-$0.048 \\ 
  & (0.228) & (0.237) & (0.221) & (0.230) & (0.220) & (0.198) \\ 
  & & & & & & \\ 
\hline \\[-1.8ex] 
Mean & 0.287 & 0.354 & 0.282 & 0.282 & 0.277 & 0.262 \\ 
Observations & 191 & 191 & 191 & 191 & 191 & 191 \\ 
\hline 
\hline \\[-1.8ex] 
\textit{Note:}  & \multicolumn{6}{r}{$^{*}$p$<$0.1; $^{**}$p$<$0.05; $^{***}$p$<$0.01} \\ 
\end{tabular} 
}
	\end{center}
	{\footnotesize Note: See notes under Table \ref{table heating} and Table \ref{table standard opinion} for a description of the covariates.
	\newline *p$<$0.1; **p$<$0.05; ***p$<$0.01}
\end{table}	

\clearpage
\subsection{Preferences on climate policies}

\begin{table}[h!]
	\caption{Worried about climate change}
	\begin{center}
		\scalebox{0.7}{
\begin{tabular}{@{\extracolsep{5pt}}lc} 
\\[-1.8ex]\hline 
\hline \\[-1.8ex] 
\\[-1.8ex] & Worried about impacts of CC \\ 
\hline \\[-1.8ex] 
 White only & 0.034 \\ 
  & (0.086) \\ 
  & \\ 
 Male & $-$0.051 \\ 
  & (0.075) \\ 
  & \\ 
 Children & 0.097 \\ 
  & (0.076) \\ 
  & \\ 
 No college & $-$0.094 \\ 
  & (0.085) \\ 
  & \\ 
 Retired & 0.027 \\ 
  & (0.134) \\ 
  & \\ 
 Student & $-$0.435 \\ 
  & (0.320) \\ 
  & \\ 
 Working & $-$0.0003 \\ 
  & (0.133) \\ 
  & \\ 
 Income Q2 & 0.178 \\ 
  & (0.113) \\ 
  & \\ 
 Income Q3 & 0.021 \\ 
  & (0.107) \\ 
  & \\ 
 Income Q4 & 0.114 \\ 
  & (0.114) \\ 
  & \\ 
 30-49 & $-$0.047 \\ 
  & (0.185) \\ 
  & \\ 
 50-87 & $-$0.178 \\ 
  & (0.191) \\ 
  & \\ 
 Non voting & $-$0.082 \\ 
  & (0.119) \\ 
  & \\ 
 Other & $-$0.033 \\ 
  & (0.174) \\ 
  & \\ 
 Trump & $-$0.389$^{***}$ \\ 
  & (0.079) \\ 
  & \\ 
 Both treatments & $-$0.280$^{***}$ \\ 
  & (0.106) \\ 
  & \\ 
 Climate treatment only & $-$0.239$^{**}$ \\ 
  & (0.100) \\ 
  & \\ 
 Policy treatment only & $-$0.199$^{**}$ \\ 
  & (0.092) \\ 
  & \\ 
 Constant & 0.937$^{***}$ \\ 
  & (0.229) \\ 
  & \\ 
\hline \\[-1.8ex] 
Control group mean & 0.75 \\ 
Observations & 191 \\ 
\hline 
\hline \\[-1.8ex] 
\textit{Note:}  & \multicolumn{1}{r}{$^{*}$p$<$0.1; $^{**}$p$<$0.05; $^{***}$p$<$0.01} \\ 
\end{tabular} 
}
	\end{center}
	{\footnotesize Note: See notes under Table \ref{table heating} and Table \ref{table standard opinion} for a description of the covariates.
	\newline *p$<$0.1; **p$<$0.05; ***p$<$0.01}
\end{table}	

\begin{table}[h!]
	\caption{Support for climate policies}
	\begin{center}
		\scalebox{0.7}{
\begin{tabular}{@{\extracolsep{5pt}}lcccccc} 
\\[-1.8ex]\hline 
\hline \\[-1.8ex] 
 & \multicolumn{6}{c}{Support climate policies} \\ 
\cline{2-7} 
\\[-1.8ex] & Tax on flying & Tax on fossil fuels & Thermal renovation & Ban polluting vehicles in city centers & Subsidies & Global climate fund \\ 
\\[-1.8ex] & (1) & (2) & (3) & (4) & (5) & (6)\\ 
\hline \\[-1.8ex] 
 White only & 0.109 & $-$0.001 & 0.151$^{*}$ & 0.068 & 0.036 & 0.044 \\ 
  & (0.079) & (0.082) & (0.080) & (0.084) & (0.087) & (0.084) \\ 
  & & & & & & \\ 
 Male & 0.082 & 0.031 & 0.136$^{*}$ & 0.113 & 0.066 & 0.006 \\ 
  & (0.069) & (0.071) & (0.070) & (0.073) & (0.076) & (0.074) \\ 
  & & & & & & \\ 
 Children & $-$0.009 & 0.025 & 0.124$^{*}$ & 0.049 & 0.013 & 0.068 \\ 
  & (0.070) & (0.072) & (0.071) & (0.074) & (0.077) & (0.075) \\ 
  & & & & & & \\ 
 No college & $-$0.069 & $-$0.063 & $-$0.042 & $-$0.033 & $-$0.108 & $-$0.030 \\ 
  & (0.078) & (0.081) & (0.079) & (0.083) & (0.086) & (0.084) \\ 
  & & & & & & \\ 
 Retired & $-$0.092 & 0.035 & 0.011 & 0.019 & $-$0.062 & $-$0.093 \\ 
  & (0.124) & (0.128) & (0.125) & (0.131) & (0.135) & (0.132) \\ 
  & & & & & & \\ 
 Student & $-$0.221 & $-$0.103 & $-$0.450 & $-$0.145 & $-$0.651$^{**}$ & $-$0.436 \\ 
  & (0.295) & (0.305) & (0.297) & (0.312) & (0.323) & (0.315) \\ 
  & & & & & & \\ 
 Working & $-$0.049 & 0.053 & 0.006 & 0.091 & $-$0.023 & $-$0.064 \\ 
  & (0.122) & (0.126) & (0.123) & (0.129) & (0.134) & (0.131) \\ 
  & & & & & & \\ 
 Income Q2 & 0.159 & 0.089 & 0.115 & 0.109 & 0.080 & 0.213$^{*}$ \\ 
  & (0.104) & (0.108) & (0.105) & (0.110) & (0.114) & (0.111) \\ 
  & & & & & & \\ 
 Income Q3 & 0.116 & 0.087 & $-$0.079 & $-$0.063 & 0.003 & 0.142 \\ 
  & (0.099) & (0.102) & (0.100) & (0.105) & (0.108) & (0.106) \\ 
  & & & & & & \\ 
 Income Q4 & 0.173 & 0.167 & 0.049 & 0.111 & 0.066 & 0.224$^{**}$ \\ 
  & (0.105) & (0.109) & (0.106) & (0.111) & (0.115) & (0.112) \\ 
  & & & & & & \\ 
 30-49 & $-$0.162 & 0.124 & 0.060 & $-$0.146 & $-$0.147 & $-$0.157 \\ 
  & (0.171) & (0.177) & (0.172) & (0.181) & (0.187) & (0.182) \\ 
  & & & & & & \\ 
 50-87 & $-$0.420$^{**}$ & $-$0.166 & $-$0.116 & $-$0.322$^{*}$ & $-$0.252 & $-$0.363$^{*}$ \\ 
  & (0.176) & (0.182) & (0.178) & (0.186) & (0.193) & (0.188) \\ 
  & & & & & & \\ 
 Non voting & $-$0.426$^{***}$ & $-$0.361$^{***}$ & $-$0.412$^{***}$ & $-$0.482$^{***}$ & $-$0.369$^{***}$ & $-$0.244$^{**}$ \\ 
  & (0.110) & (0.114) & (0.111) & (0.116) & (0.120) & (0.117) \\ 
  & & & & & & \\ 
 Other & $-$0.380$^{**}$ & $-$0.265 & $-$0.123 & $-$0.381$^{**}$ & $-$0.321$^{*}$ & $-$0.295$^{*}$ \\ 
  & (0.160) & (0.166) & (0.162) & (0.169) & (0.175) & (0.171) \\ 
  & & & & & & \\ 
 Trump & $-$0.354$^{***}$ & $-$0.394$^{***}$ & $-$0.478$^{***}$ & $-$0.357$^{***}$ & $-$0.373$^{***}$ & $-$0.401$^{***}$ \\ 
  & (0.073) & (0.076) & (0.074) & (0.077) & (0.080) & (0.078) \\ 
  & & & & & & \\ 
 Climate treatment only & $-$0.074 & 0.121 & 0.106 & 0.074 & $-$0.099 & $-$0.015 \\ 
  & (0.099) & (0.103) & (0.100) & (0.105) & (0.109) & (0.106) \\ 
  & & & & & & \\ 
 No treatment & $-$0.010 & 0.017 & 0.096 & 0.046 & $-$0.099 & 0.010 \\ 
  & (0.098) & (0.101) & (0.099) & (0.103) & (0.107) & (0.104) \\ 
  & & & & & & \\ 
 Policy treatment only & $-$0.076 & $-$0.046 & $-$0.073 & $-$0.088 & $-$0.072 & $-$0.035 \\ 
  & (0.089) & (0.092) & (0.090) & (0.094) & (0.097) & (0.095) \\ 
  & & & & & & \\ 
 Constant & 0.802$^{***}$ & 0.550$^{**}$ & 0.498$^{**}$ & 0.703$^{***}$ & 0.971$^{***}$ & 0.793$^{***}$ \\ 
  & (0.214) & (0.221) & (0.216) & (0.226) & (0.234) & (0.228) \\ 
  & & & & & & \\ 
\hline \\[-1.8ex] 
Mean &  &  &  &  &  &  \\ 
Observations & 191 & 191 & 191 & 191 & 191 & 191 \\ 
\hline 
\hline \\[-1.8ex] 
\textit{Note:}  & \multicolumn{6}{r}{$^{*}$p$<$0.1; $^{**}$p$<$0.05; $^{***}$p$<$0.01} \\ 
\end{tabular} 
}
	\end{center}
	{\footnotesize Note: See notes under Table \ref{table heating} and Table \ref{table standard opinion} for a description of the covariates.
	\newline *p$<$0.1; **p$<$0.05; ***p$<$0.01}
\end{table}	

\begin{landscape}
	\begin{table}[h!]
	\caption{Support carbon tax, depending on the use of revenues}
	\begin{center}
		\scalebox{0.6}{
\begin{tabular}{@{\extracolsep{5pt}}lcccccccccc} 
\\[-1.8ex]\hline 
\hline \\[-1.8ex] 
 & \multicolumn{10}{c}{Support carbon tax if revenues allocated to…} \\ 
\cline{2-11} 
\\[-1.8ex] & Transfer to constrained HH & Transfers to poorest & Equal transfers & Tax rebates for affected firms & Infrastructure projects & Technology subsidies & Reduce deficit & Reduce CIT & Reduce PIT & Other \\ 
\\[-1.8ex] & (1) & (2) & (3) & (4) & (5) & (6) & (7) & (8) & (9) & (10)\\ 
\hline \\[-1.8ex] 
 White only & 0.054 & 0.099 & 0.136$^{*}$ & 0.003 & 0.083 & 0.040 & 0.162$^{*}$ & 0.036 & 0.092 & 0.034 \\ 
  & (0.089) & (0.084) & (0.080) & (0.084) & (0.085) & (0.087) & (0.091) & (0.081) & (0.090) & (0.074) \\ 
  & & & & & & & & & & \\ 
 Male & 0.103 & $-$0.038 & $-$0.008 & $-$0.033 & 0.023 & 0.101 & 0.167$^{**}$ & 0.065 & 0.088 & 0.023 \\ 
  & (0.077) & (0.073) & (0.070) & (0.073) & (0.075) & (0.076) & (0.079) & (0.070) & (0.078) & (0.065) \\ 
  & & & & & & & & & & \\ 
 Children & 0.047 & 0.052 & 0.022 & 0.021 & $-$0.084 & $-$0.032 & 0.022 & 0.147$^{**}$ & 0.104 & $-$0.004 \\ 
  & (0.078) & (0.074) & (0.071) & (0.074) & (0.076) & (0.077) & (0.080) & (0.071) & (0.079) & (0.066) \\ 
  & & & & & & & & & & \\ 
 No college & $-$0.029 & 0.013 & 0.075 & 0.040 & $-$0.146$^{*}$ & $-$0.095 & $-$0.115 & $-$0.008 & $-$0.025 & 0.048 \\ 
  & (0.088) & (0.083) & (0.079) & (0.083) & (0.085) & (0.086) & (0.090) & (0.080) & (0.089) & (0.074) \\ 
  & & & & & & & & & & \\ 
 Retired & $-$0.024 & $-$0.141 & $-$0.388$^{***}$ & $-$0.296$^{**}$ & 0.031 & $-$0.036 & $-$0.048 & $-$0.064 & $-$0.130 & 0.048 \\ 
  & (0.138) & (0.131) & (0.125) & (0.130) & (0.133) & (0.135) & (0.142) & (0.126) & (0.140) & (0.116) \\ 
  & & & & & & & & & & \\ 
 Student & $-$0.541 & $-$0.431 & $-$0.783$^{***}$ & $-$0.226 & $-$0.336 & $-$0.453 & $-$0.542 & $-$0.049 & $-$0.303 & $-$0.248 \\ 
  & (0.330) & (0.313) & (0.297) & (0.311) & (0.318) & (0.323) & (0.338) & (0.300) & (0.334) & (0.277) \\ 
  & & & & & & & & & & \\ 
 Working & $-$0.054 & $-$0.187 & $-$0.307$^{**}$ & $-$0.215$^{*}$ & $-$0.039 & $-$0.012 & $-$0.090 & $-$0.002 & $-$0.001 & 0.125 \\ 
  & (0.137) & (0.130) & (0.123) & (0.129) & (0.132) & (0.134) & (0.140) & (0.125) & (0.139) & (0.115) \\ 
  & & & & & & & & & & \\ 
 Income Q2 & 0.009 & 0.127 & $-$0.069 & 0.155 & 0.206$^{*}$ & 0.050 & 0.101 & 0.022 & $-$0.010 & 0.074 \\ 
  & (0.117) & (0.111) & (0.105) & (0.110) & (0.113) & (0.114) & (0.119) & (0.106) & (0.118) & (0.098) \\ 
  & & & & & & & & & & \\ 
 Income Q3 & $-$0.047 & 0.047 & $-$0.053 & 0.021 & 0.110 & 0.073 & 0.057 & 0.009 & $-$0.048 & 0.043 \\ 
  & (0.111) & (0.105) & (0.100) & (0.104) & (0.107) & (0.108) & (0.113) & (0.101) & (0.112) & (0.093) \\ 
  & & & & & & & & & & \\ 
 Income Q4 & $-$0.037 & $-$0.035 & $-$0.038 & 0.129 & 0.096 & 0.113 & 0.143 & $-$0.019 & $-$0.167 & 0.118 \\ 
  & (0.118) & (0.111) & (0.106) & (0.111) & (0.113) & (0.115) & (0.120) & (0.107) & (0.119) & (0.099) \\ 
  & & & & & & & & & & \\ 
 30-49 & 0.052 & $-$0.175 & $-$0.040 & $-$0.098 & 0.023 & 0.142 & $-$0.099 & 0.002 & $-$0.067 & 0.025 \\ 
  & (0.191) & (0.181) & (0.172) & (0.180) & (0.185) & (0.187) & (0.196) & (0.174) & (0.194) & (0.161) \\ 
  & & & & & & & & & & \\ 
 50-87 & $-$0.274 & $-$0.516$^{***}$ & $-$0.454$^{**}$ & $-$0.459$^{**}$ & $-$0.253 & $-$0.075 & $-$0.256 & $-$0.327$^{*}$ & $-$0.180 & $-$0.029 \\ 
  & (0.197) & (0.187) & (0.178) & (0.186) & (0.190) & (0.193) & (0.202) & (0.179) & (0.200) & (0.166) \\ 
  & & & & & & & & & & \\ 
 Non voting & $-$0.213$^{*}$ & $-$0.535$^{***}$ & $-$0.355$^{***}$ & $-$0.326$^{***}$ & $-$0.347$^{***}$ & $-$0.354$^{***}$ & $-$0.134 & $-$0.251$^{**}$ & $-$0.364$^{***}$ & 0.039 \\ 
  & (0.123) & (0.117) & (0.111) & (0.116) & (0.119) & (0.120) & (0.126) & (0.112) & (0.125) & (0.103) \\ 
  & & & & & & & & & & \\ 
 Other & $-$0.180 & $-$0.361$^{**}$ & $-$0.147 & $-$0.232 & $-$0.216 & $-$0.109 & $-$0.303 & $-$0.256 & $-$0.201 & $-$0.204 \\ 
  & (0.179) & (0.170) & (0.162) & (0.169) & (0.173) & (0.175) & (0.183) & (0.163) & (0.182) & (0.151) \\ 
  & & & & & & & & & & \\ 
 Trump & $-$0.161$^{*}$ & $-$0.336$^{***}$ & $-$0.189$^{**}$ & $-$0.152$^{*}$ & $-$0.355$^{***}$ & $-$0.304$^{***}$ & $-$0.148$^{*}$ & 0.016 & 0.076 & $-$0.108 \\ 
  & (0.082) & (0.078) & (0.074) & (0.077) & (0.079) & (0.080) & (0.084) & (0.075) & (0.083) & (0.069) \\ 
  & & & & & & & & & & \\ 
 Both treatments & $-$0.083 & $-$0.004 & $-$0.056 & $-$0.094 & 0.069 & $-$0.016 & 0.027 & 0.083 & 0.167 & $-$0.004 \\ 
  & (0.110) & (0.104) & (0.099) & (0.103) & (0.106) & (0.107) & (0.112) & (0.100) & (0.111) & (0.092) \\ 
  & & & & & & & & & & \\ 
 Climate treatment only & $-$0.059 & 0.137 & 0.003 & $-$0.062 & 0.044 & 0.048 & $-$0.032 & 0.078 & $-$0.031 & 0.134 \\ 
  & (0.103) & (0.098) & (0.093) & (0.097) & (0.100) & (0.101) & (0.106) & (0.094) & (0.105) & (0.087) \\ 
  & & & & & & & & & & \\ 
 Policy treatment only & $-$0.010 & $-$0.023 & 0.031 & $-$0.058 & $-$0.003 & $-$0.015 & $-$0.005 & 0.104 & 0.111 & 0.089 \\ 
  & (0.095) & (0.090) & (0.085) & (0.089) & (0.091) & (0.093) & (0.097) & (0.086) & (0.096) & (0.080) \\ 
  & & & & & & & & & & \\ 
 Constant & 0.618$^{***}$ & 1.022$^{***}$ & 1.028$^{***}$ & 1.035$^{***}$ & 0.784$^{***}$ & 0.592$^{**}$ & 0.535$^{**}$ & 0.345 & 0.437$^{*}$ & $-$0.010 \\ 
  & (0.237) & (0.224) & (0.213) & (0.223) & (0.228) & (0.231) & (0.242) & (0.215) & (0.239) & (0.199) \\ 
  & & & & & & & & & & \\ 
\hline \\[-1.8ex] 
Control group mean & 0.458 & 0.438 & 0.417 & 0.458 & 0.542 & 0.562 & 0.521 & 0.271 & 0.396 & 0.125 \\ 
Observations & 191 & 191 & 191 & 191 & 191 & 191 & 191 & 191 & 191 & 191 \\ 
\hline 
\hline \\[-1.8ex] 
\textit{Note:}  & \multicolumn{10}{r}{$^{*}$p$<$0.1; $^{**}$p$<$0.05; $^{***}$p$<$0.01} \\ 
\end{tabular} 
}
	\end{center}
	{\footnotesize Note: See notes under Table \ref{table heating} and Table \ref{table standard opinion} for a description of the covariates.
	\newline *p$<$0.1; **p$<$0.05; ***p$<$0.01}
\end{table}	
\end{landscape}

\clearpage
\subsection{Preferences for bans vs. incentives}

\begin{table}[h!]
	\caption{Renovation enforcement}
	\begin{center}
		\scalebox{0.7}{
\begin{tabular}{@{\extracolsep{5pt}}lcc} 
\\[-1.8ex]\hline 
\hline \\[-1.8ex] 
 & \multicolumn{2}{c}{Thermal renovation should be (if subsidized)} \\ 
\cline{2-3} 
\\[-1.8ex] & made mandatory & on a voluntary basis \\ 
\\[-1.8ex] & (1) & (2)\\ 
\hline \\[-1.8ex] 
 White only & 0.007 & 0.169$^{*}$ \\ 
  & (0.085) & (0.092) \\ 
  & & \\ 
 Male & 0.046 & $-$0.108 \\ 
  & (0.074) & (0.080) \\ 
  & & \\ 
 Children & 0.069 & 0.017 \\ 
  & (0.075) & (0.082) \\ 
  & & \\ 
 No college & $-$0.080 & $-$0.023 \\ 
  & (0.084) & (0.091) \\ 
  & & \\ 
 Retired & $-$0.264$^{**}$ & 0.078 \\ 
  & (0.133) & (0.144) \\ 
  & & \\ 
 Student & $-$0.483 & 0.207 \\ 
  & (0.317) & (0.343) \\ 
  & & \\ 
 Working & $-$0.087 & $-$0.065 \\ 
  & (0.131) & (0.142) \\ 
  & & \\ 
 Income Q2 & 0.035 & 0.088 \\ 
  & (0.112) & (0.121) \\ 
  & & \\ 
 Income Q3 & $-$0.119 & 0.140 \\ 
  & (0.106) & (0.115) \\ 
  & & \\ 
 Income Q4 & 0.030 & 0.055 \\ 
  & (0.113) & (0.122) \\ 
  & & \\ 
 30-49 & $-$0.063 & $-$0.121 \\ 
  & (0.183) & (0.199) \\ 
  & & \\ 
 50-87 & $-$0.118 & $-$0.086 \\ 
  & (0.189) & (0.205) \\ 
  & & \\ 
 Non voting & $-$0.333$^{***}$ & 0.176 \\ 
  & (0.118) & (0.128) \\ 
  & & \\ 
 Other & $-$0.228 & 0.022 \\ 
  & (0.172) & (0.187) \\ 
  & & \\ 
 Trump & $-$0.241$^{***}$ & 0.199$^{**}$ \\ 
  & (0.079) & (0.085) \\ 
  & & \\ 
 Climate treatment only & $-$0.126 & 0.229$^{**}$ \\ 
  & (0.107) & (0.116) \\ 
  & & \\ 
 No treatment & $-$0.036 & 0.029 \\ 
  & (0.105) & (0.114) \\ 
  & & \\ 
 Policy treatment only & $-$0.021 & 0.039 \\ 
  & (0.096) & (0.104) \\ 
  & & \\ 
 Constant & 0.747$^{***}$ & 0.273 \\ 
  & (0.230) & (0.249) \\ 
  & & \\ 
\hline \\[-1.8ex] 
Mean &  &  \\ 
Observations & 191 & 191 \\ 
\hline 
\hline \\[-1.8ex] 
\textit{Note:}  & \multicolumn{2}{r}{$^{*}$p$<$0.1; $^{**}$p$<$0.05; $^{***}$p$<$0.01} \\ 
\end{tabular} 
}
	\end{center}
	{\footnotesize Note: See notes under Table \ref{table heating} and Table \ref{table standard opinion} for a description of the covariates.
	\newline *p$<$0.1; **p$<$0.05; ***p$<$0.01}
\end{table}	

\begin{landscape}
	\begin{table}[h!]
	\caption{Flight restrictions enforcement}
	\begin{center}
		\scalebox{0.6}{
\begin{tabular}{@{\extracolsep{5pt}}lcccccc} 
\\[-1.8ex]\hline 
\hline \\[-1.8ex] 
 & \multicolumn{6}{c}{Government limit flight trips} \\ 
\cline{2-7} 
\\[-1.8ex] & Rationing (1000km) & Tradable (1000km) & Rationing (3000km) & Tradable (3000km) & Rationing (0.5 round-trip/year) & Tradable (0.5 round-trip/year) \\ 
\\[-1.8ex] & (1) & (2) & (3) & (4) & (5) & (6)\\ 
\hline \\[-1.8ex] 
 White only & 0.130 & $-$0.059 & 0.301$^{*}$ & $-$0.144 & 0.138 & 0.064 \\ 
  & (0.192) & (0.148) & (0.160) & (0.128) & (0.185) & (0.131) \\ 
  & & & & & & \\ 
 Male & $-$0.147 & 0.165 & $-$0.160 & 0.168 & $-$0.113 & 0.137 \\ 
  & (0.182) & (0.140) & (0.144) & (0.115) & (0.165) & (0.118) \\ 
  & & & & & & \\ 
 Children & $-$0.254 & 0.156 & $-$0.104 & 0.114 & $-$0.240 & 0.139 \\ 
  & (0.158) & (0.121) & (0.167) & (0.134) & (0.146) & (0.104) \\ 
  & & & & & & \\ 
 No college & $-$0.063 & 0.170 & 0.062 & 0.032 & $-$0.013 & 0.043 \\ 
  & (0.218) & (0.167) & (0.177) & (0.141) & (0.171) & (0.122) \\ 
  & & & & & & \\ 
 Retired & $-$0.536 & 0.386 & $-$0.249 & 0.249 & 0.116 & $-$0.190 \\ 
  & (0.383) & (0.294) & (0.281) & (0.225) & (0.278) & (0.198) \\ 
  & & & & & & \\ 
 Student & 0.281 & $-$0.274 & 0.535 & 0.673 &  &  \\ 
  & (0.575) & (0.441) & (0.864) & (0.690) &  &  \\ 
  & & & & & & \\ 
 Working & $-$0.626 & 0.556$^{*}$ & $-$0.012 & 0.142 & 0.116 & $-$0.199 \\ 
  & (0.398) & (0.306) & (0.286) & (0.229) & (0.273) & (0.194) \\ 
  & & & & & & \\ 
 Income Q2 & 0.060 & 0.048 & $-$0.483$^{**}$ & 0.234 & 0.263 & $-$0.218 \\ 
  & (0.219) & (0.168) & (0.230) & (0.184) & (0.271) & (0.193) \\ 
  & & & & & & \\ 
 Income Q3 & 0.190 & $-$0.057 & $-$0.274 & 0.085 & 0.011 & 0.026 \\ 
  & (0.222) & (0.170) & (0.232) & (0.185) & (0.241) & (0.171) \\ 
  & & & & & & \\ 
 Income Q4 & 0.016 & 0.038 & $-$0.266 & 0.138 & $-$0.048 & 0.045 \\ 
  & (0.262) & (0.201) & (0.238) & (0.190) & (0.247) & (0.176) \\ 
  & & & & & & \\ 
 30-49 & 0.279 & $-$0.217 & 0.437 & 0.185 & $-$0.367 & 0.076 \\ 
  & (0.322) & (0.247) & (0.588) & (0.469) & (0.381) & (0.271) \\ 
  & & & & & & \\ 
 50-87 & 0.006 & $-$0.419 & 0.677 & $-$0.050 & $-$0.519 & $-$0.271 \\ 
  & (0.341) & (0.262) & (0.616) & (0.492) & (0.378) & (0.269) \\ 
  & & & & & & \\ 
 Non voting & $-$0.404 & 0.359$^{*}$ & $-$0.197 & $-$0.207 & $-$0.239 & 0.178 \\ 
  & (0.252) & (0.194) & (0.256) & (0.204) & (0.286) & (0.204) \\ 
  & & & & & & \\ 
 Other & $-$0.213 & $-$0.251 & $-$0.809$^{*}$ & $-$0.297 & $-$0.439 & $-$0.011 \\ 
  & (0.312) & (0.239) & (0.453) & (0.362) & (0.321) & (0.228) \\ 
  & & & & & & \\ 
 Trump & $-$0.355$^{*}$ & $-$0.065 & $-$0.036 & $-$0.343$^{**}$ & $-$0.207 & 0.144 \\ 
  & (0.185) & (0.142) & (0.166) & (0.133) & (0.165) & (0.118) \\ 
  & & & & & & \\ 
 Climate treatment only & $-$0.218 & 0.184 & $-$0.199 & 0.059 & 0.184 & $-$0.380$^{*}$ \\ 
  & (0.214) & (0.165) & (0.212) & (0.169) & (0.271) & (0.193) \\ 
  & & & & & & \\ 
 No treatment & 0.235 & $-$0.082 & $-$0.146 & 0.150 & 0.270 & $-$0.410$^{**}$ \\ 
  & (0.206) & (0.158) & (0.217) & (0.173) & (0.271) & (0.193) \\ 
  & & & & & & \\ 
 Policy treatment only & 0.116 & $-$0.206 & $-$0.148 & 0.187 & 0.301 & $-$0.346$^{*}$ \\ 
  & (0.193) & (0.148) & (0.201) & (0.160) & (0.247) & (0.175) \\ 
  & & & & & & \\ 
 Constant & 1.071$^{*}$ & $-$0.149 & 0.307 & $-$0.135 & 0.710 & 0.526 \\ 
  & (0.558) & (0.428) & (0.613) & (0.489) & (0.520) & (0.370) \\ 
  & & & & & & \\ 
\hline \\[-1.8ex] 
Mean &  &  &  &  &  &  \\ 
Observations & 61 & 61 & 67 & 67 & 63 & 63 \\ 
\hline 
\hline \\[-1.8ex] 
\textit{Note:}  & \multicolumn{6}{r}{$^{*}$p$<$0.1; $^{**}$p$<$0.05; $^{***}$p$<$0.01} \\ 
\end{tabular} 
}
	\end{center}
	{\footnotesize Note: See notes under Table \ref{table heating} and Table \ref{table standard opinion} for a description of the covariates.
	\newline *p$<$0.1; **p$<$0.05; ***p$<$0.01}
\end{table}	
\end{landscape}

\begin{table}[h!]
	\caption{Cattle consumption restrictions enforcement}
	\begin{center}
		\scalebox{0.7}{
\begin{tabular}{@{\extracolsep{5pt}}lcccc} 
\\[-1.8ex]\hline 
\hline \\[-1.8ex] 
 & \multicolumn{4}{c}{Government limit cattle products, would approve…} \\ 
\cline{2-5} 
\\[-1.8ex] & Tax on cattle products (beefx2) & Sub Vegetables & No sub cattle & Ban intensive cattle \\ 
\\[-1.8ex] & (1) & (2) & (3) & (4)\\ 
\hline \\[-1.8ex] 
 White only & 0.064 & 0.106 & 0.044 & 0.082 \\ 
  & (0.074) & (0.081) & (0.087) & (0.059) \\ 
  & & & & \\ 
 Male & 0.114$^{*}$ & 0.046 & 0.079 & 0.013 \\ 
  & (0.064) & (0.071) & (0.076) & (0.052) \\ 
  & & & & \\ 
 Children & 0.074 & 0.095 & 0.162$^{**}$ & 0.024 \\ 
  & (0.065) & (0.072) & (0.077) & (0.052) \\ 
  & & & & \\ 
 No college & $-$0.039 & 0.063 & $-$0.124 & 0.056 \\ 
  & (0.073) & (0.081) & (0.086) & (0.058) \\ 
  & & & & \\ 
 Retired & $-$0.150 & $-$0.042 & $-$0.101 & 0.033 \\ 
  & (0.115) & (0.127) & (0.135) & (0.092) \\ 
  & & & & \\ 
 Student & 0.058 & $-$0.257 & $-$0.042 & $-$0.168 \\ 
  & (0.275) & (0.303) & (0.323) & (0.220) \\ 
  & & & & \\ 
 Working & $-$0.068 & 0.014 & $-$0.063 & 0.061 \\ 
  & (0.114) & (0.126) & (0.134) & (0.091) \\ 
  & & & & \\ 
 Income Q2 & 0.027 & $-$0.078 & 0.063 & $-$0.043 \\ 
  & (0.097) & (0.107) & (0.114) & (0.078) \\ 
  & & & & \\ 
 Income Q3 & $-$0.045 & $-$0.033 & $-$0.059 & $-$0.106 \\ 
  & (0.092) & (0.102) & (0.108) & (0.074) \\ 
  & & & & \\ 
 Income Q4 & $-$0.094 & $-$0.015 & $-$0.025 & $-$0.046 \\ 
  & (0.098) & (0.108) & (0.115) & (0.078) \\ 
  & & & & \\ 
 30-49 & $-$0.378$^{**}$ & 0.026 & $-$0.106 & $-$0.151 \\ 
  & (0.160) & (0.176) & (0.187) & (0.128) \\ 
  & & & & \\ 
 50-87 & $-$0.643$^{***}$ & $-$0.080 & $-$0.134 & $-$0.194 \\ 
  & (0.164) & (0.181) & (0.193) & (0.132) \\ 
  & & & & \\ 
 Non voting & $-$0.280$^{***}$ & $-$0.123 & $-$0.044 & $-$0.088 \\ 
  & (0.103) & (0.113) & (0.120) & (0.082) \\ 
  & & & & \\ 
 Other & $-$0.146 & $-$0.259 & 0.061 & $-$0.194 \\ 
  & (0.150) & (0.165) & (0.175) & (0.120) \\ 
  & & & & \\ 
 Trump & $-$0.150$^{**}$ & $-$0.246$^{***}$ & $-$0.071 & $-$0.169$^{***}$ \\ 
  & (0.068) & (0.075) & (0.080) & (0.055) \\ 
  & & & & \\ 
 Both & $-$0.097 & 0.084 & $-$0.164 & $-$0.096 \\ 
  & (0.091) & (0.101) & (0.107) & (0.073) \\ 
  & & & & \\ 
 Climate treatment only & $-$0.082 & 0.122 & $-$0.106 & $-$0.031 \\ 
  & (0.086) & (0.095) & (0.101) & (0.069) \\ 
  & & & & \\ 
 Policy treatment only & $-$0.095 & 0.134 & $-$0.086 & $-$0.044 \\ 
  & (0.079) & (0.087) & (0.093) & (0.063) \\ 
  & & & & \\ 
 Constant & 0.910$^{***}$ & 0.174 & 0.446$^{*}$ & 0.306$^{*}$ \\ 
  & (0.197) & (0.217) & (0.231) & (0.158) \\ 
  & & & & \\ 
\hline \\[-1.8ex] 
Mean & 0.251 & 0.262 & 0.287 & 0.113 \\ 
Observations & 191 & 191 & 191 & 191 \\ 
\hline 
\hline \\[-1.8ex] 
\textit{Note:}  & \multicolumn{4}{r}{$^{*}$p$<$0.1; $^{**}$p$<$0.05; $^{***}$p$<$0.01} \\ 
\end{tabular} 
}
	\end{center}
	{\footnotesize Note: See notes under Table \ref{table heating} and Table \ref{table standard opinion} for a description of the covariates.
	\newline *p$<$0.1; **p$<$0.05; ***p$<$0.01}
\end{table}	

\begin{table}[h!]
	\caption{Environment protection enforcement}
	\begin{center}
		\scalebox{0.7}{
\begin{tabular}{@{\extracolsep{5pt}}lcc} 
\\[-1.8ex]\hline 
\hline \\[-1.8ex] 
 & \multicolumn{2}{c}{Government protect environment} \\ 
\cline{2-3} 
\\[-1.8ex] & Force people & Encourage people \\ 
\\[-1.8ex] & (1) & (2)\\ 
\hline \\[-1.8ex] 
 White only & $-$0.007 & 0.116 \\ 
  & (0.077) & (0.088) \\ 
  & & \\ 
 Male & 0.066 & 0.015 \\ 
  & (0.067) & (0.077) \\ 
  & & \\ 
 Children & 0.129$^{*}$ & 0.021 \\ 
  & (0.068) & (0.078) \\ 
  & & \\ 
 No college & $-$0.058 & $-$0.017 \\ 
  & (0.076) & (0.087) \\ 
  & & \\ 
 Retired & $-$0.143 & $-$0.043 \\ 
  & (0.120) & (0.138) \\ 
  & & \\ 
 Student & 0.004 & 0.158 \\ 
  & (0.287) & (0.329) \\ 
  & & \\ 
 Working & 0.066 & $-$0.249$^{*}$ \\ 
  & (0.119) & (0.137) \\ 
  & & \\ 
 Income Q2 & $-$0.178$^{*}$ & 0.227$^{*}$ \\ 
  & (0.101) & (0.116) \\ 
  & & \\ 
 Income Q3 & $-$0.159 & 0.129 \\ 
  & (0.096) & (0.111) \\ 
  & & \\ 
 Income Q4 & 0.018 & 0.039 \\ 
  & (0.102) & (0.117) \\ 
  & & \\ 
 30-49 & $-$0.186 & 0.159 \\ 
  & (0.166) & (0.191) \\ 
  & & \\ 
 50-87 & $-$0.345$^{**}$ & 0.230 \\ 
  & (0.171) & (0.197) \\ 
  & & \\ 
 Non voting & $-$0.374$^{***}$ & 0.234$^{*}$ \\ 
  & (0.107) & (0.123) \\ 
  & & \\ 
 Other & $-$0.342$^{**}$ & 0.275 \\ 
  & (0.156) & (0.179) \\ 
  & & \\ 
 Trump & $-$0.269$^{***}$ & 0.220$^{***}$ \\ 
  & (0.071) & (0.082) \\ 
  & & \\ 
 Climate treatment only & $-$0.096 & 0.003 \\ 
  & (0.097) & (0.111) \\ 
  & & \\ 
 No treatment & $-$0.075 & $-$0.063 \\ 
  & (0.095) & (0.109) \\ 
  & & \\ 
 Policy treatment only & $-$0.054 & 0.041 \\ 
  & (0.087) & (0.099) \\ 
  & & \\ 
 Constant & 0.838$^{***}$ & 0.040 \\ 
  & (0.208) & (0.239) \\ 
  & & \\ 
\hline \\[-1.8ex] 
Mean & 0.354 & 0.41 \\ 
Observations & 191 & 191 \\ 
\hline 
\hline \\[-1.8ex] 
\textit{Note:}  & \multicolumn{2}{r}{$^{*}$p$<$0.1; $^{**}$p$<$0.05; $^{***}$p$<$0.01} \\ 
\end{tabular} 
}
	\end{center}
	{\footnotesize Note: See notes under Table \ref{table heating} and Table \ref{table standard opinion} for a description of the covariates.
	\newline *p$<$0.1; **p$<$0.05; ***p$<$0.01}
\end{table}	

\begin{table}[h!]
	\caption{Willingness to Pay}
	\begin{center}
		\scalebox{0.7}{
\begin{tabular}{@{\extracolsep{5pt}}lc} 
\\[-1.8ex]\hline 
\hline \\[-1.8ex] 
 & \multicolumn{1}{c}{WTP to limit global warming to safe levels} \\ 
\cline{2-2} 
\\[-1.8ex] & WTP (dollar a year) \\ 
\hline \\[-1.8ex] 
 White only & 3.703 \\ 
  & (69.510) \\ 
  & \\ 
 Male & $-$35.051 \\ 
  & (60.660) \\ 
  & \\ 
 Children & $-$26.705 \\ 
  & (61.530) \\ 
  & \\ 
 No college & $-$133.357$^{*}$ \\ 
  & (68.760) \\ 
  & \\ 
 Retired & $-$51.112 \\ 
  & (108.567) \\ 
  & \\ 
 Student & $-$227.498 \\ 
  & (258.864) \\ 
  & \\ 
 Working & 70.980 \\ 
  & (107.354) \\ 
  & \\ 
 Income Q2 & 111.338 \\ 
  & (91.542) \\ 
  & \\ 
 Income Q3 & 6.619 \\ 
  & (86.915) \\ 
  & \\ 
 Income Q4 & 82.615 \\ 
  & (92.239) \\ 
  & \\ 
 30-49 & $-$189.076 \\ 
  & (150.060) \\ 
  & \\ 
 50-87 & $-$124.849 \\ 
  & (154.682) \\ 
  & \\ 
 Non voting & $-$108.094 \\ 
  & (96.502) \\ 
  & \\ 
 Other & $-$171.629 \\ 
  & (140.655) \\ 
  & \\ 
 Trump & $-$202.555$^{***}$ \\ 
  & (64.230) \\ 
  & \\ 
 Both & $-$220.207$^{**}$ \\ 
  & (85.926) \\ 
  & \\ 
 Climate treatment only & $-$44.446 \\ 
  & (81.033) \\ 
  & \\ 
 Policy treatment only & $-$129.701$^{*}$ \\ 
  & (74.291) \\ 
  & \\ 
 Constant & 500.705$^{***}$ \\ 
  & (185.563) \\ 
  & \\ 
\hline \\[-1.8ex] 
Mean & 157.428 \\ 
Observations & 191 \\ 
\hline 
\hline \\[-1.8ex] 
\textit{Note:}  & \multicolumn{1}{r}{$^{*}$p$<$0.1; $^{**}$p$<$0.05; $^{***}$p$<$0.01} \\ 
\end{tabular} 
}
	\end{center}
	{\footnotesize Note: See notes under Table \ref{table heating} and Table \ref{table standard opinion} for a description of the covariates.
	\newline *p$<$0.1; **p$<$0.05; ***p$<$0.01}
\end{table}	

\clearpage
\subsection{Political views and media consumption}

\begin{table}[h!]
	\caption{Political views}
	\begin{center}
		\scalebox{0.7}{
\begin{tabular}{@{\extracolsep{5pt}}lccc} 
\\[-1.8ex]\hline 
\hline \\[-1.8ex] 
 & \multicolumn{3}{c}{Political views} \\ 
\cline{2-4} 
\\[-1.8ex] & Interest politics & Member environ org & Relative environ \\ 
\hline \\[-1.8ex] 
 Control group mean & 0.896 & 0.229 & 0.229  \\ \hline \\[-1.8ex] race: White only & 0.064 & 0.015 & $-$0.040 \\ 
  & (0.071) & (0.071) & (0.072) \\ 
  & & & \\ 
 Male & $-$0.049 & 0.034 & 0.045 \\ 
  & (0.061) & (0.062) & (0.062) \\ 
  & & & \\ 
 Children & $-$0.029 & 0.107$^{*}$ & 0.197$^{***}$ \\ 
  & (0.062) & (0.063) & (0.064) \\ 
  & & & \\ 
 No college & $-$0.054 & $-$0.044 & $-$0.021 \\ 
  & (0.070) & (0.071) & (0.071) \\ 
  & & & \\ 
 status: Retired & 0.137 & $-$0.034 & 0.002 \\ 
  & (0.110) & (0.112) & (0.113) \\ 
  & & & \\ 
 status: Student & 0.470$^{*}$ & 0.260 & 0.086 \\ 
  & (0.263) & (0.266) & (0.269) \\ 
  & & & \\ 
 staths: Working & 0.151 & $-$0.028 & 0.061 \\ 
  & (0.109) & (0.110) & (0.112) \\ 
  & & & \\ 
 Income Q2 & 0.085 & $-$0.094 & $-$0.081 \\ 
  & (0.093) & (0.094) & (0.095) \\ 
  & & & \\ 
 Income Q3 & 0.034 & $-$0.117 & $-$0.111 \\ 
  & (0.088) & (0.089) & (0.090) \\ 
  & & & \\ 
 Income Q4 & 0.081 & $-$0.053 & 0.042 \\ 
  & (0.094) & (0.095) & (0.096) \\ 
  & & & \\ 
 age: 30-49 & $-$0.116 & $-$0.236 & $-$0.018 \\ 
  & (0.152) & (0.154) & (0.155) \\ 
  & & & \\ 
 age: 50-87 & $-$0.065 & $-$0.450$^{***}$ & $-$0.212 \\ 
  & (0.154) & (0.156) & (0.157) \\ 
  & & & \\ 
 vote: Biden & 0.359$^{***}$ & 0.122 & 0.065 \\ 
  & (0.085) & (0.086) & (0.087) \\ 
  & & & \\ 
 vote: Trump & 0.288$^{***}$ & $-$0.009 & $-$0.038 \\ 
  & (0.093) & (0.094) & (0.095) \\ 
  & & & \\ 
 Both treatments & $-$0.124 & $-$0.096 & $-$0.065 \\ 
  & (0.087) & (0.088) & (0.089) \\ 
  & & & \\ 
 Climate treatment only & $-$0.147$^{*}$ & $-$0.094 & $-$0.087 \\ 
  & (0.082) & (0.083) & (0.084) \\ 
  & & & \\ 
 Policy treatment only & $-$0.021 & $-$0.023 & 0.015 \\ 
  & (0.076) & (0.076) & (0.077) \\ 
  & & & \\ 
 Constant & 0.505$^{***}$ & 0.558$^{***}$ & 0.250 \\ 
  & (0.183) & (0.185) & (0.187) \\ 
  & & & \\ 
\hline \\[-1.8ex] 

Observations & 191 & 191 & 191 \\ 
\hline 
\hline \\[-1.8ex] 
\end{tabular} }
	\end{center}
	{\footnotesize Note: See notes under Table \ref{table heating} and Table \ref{table standard opinion} for a description of the covariates.
	\newline *p$<$0.1; **p$<$0.05; ***p$<$0.01}
\end{table}	

\begin{landscape}
	\begin{table}[h!]
	\caption{Position on political spectrum}
	\begin{center}
		\scalebox{0.6}{
\begin{tabular}{@{\extracolsep{5pt}}lcccccccccccc} 
\\[-1.8ex]\hline 
\hline \\[-1.8ex] 
 & \multicolumn{12}{c}{Political positions} \\ 
\cline{2-13} 
\\[-1.8ex] & Far Left & Left & Center & Right & Far Right & Liberal & Conservative & Humanist & Patriot & Apolitical & Environmentalist & Feminist \\ 
\\[-1.8ex] & (1) & (2) & (3) & (4) & (5) & (6) & (7) & (8) & (9) & (10) & (11) & (12)\\ 
\hline \\[-1.8ex] 
 White only & 0.057 & 0.054 & $-$0.114 & 0.048 & 0.051 & $-$0.039 & $-$0.026 & 0.002 & 0.071 & $-$0.009 & 0.062 & 0.025 \\ 
  & (0.054) & (0.070) & (0.087) & (0.060) & (0.047) & (0.064) & (0.081) & (0.050) & (0.064) & (0.038) & (0.043) & (0.034) \\ 
  & & & & & & & & & & & & \\ 
 Male & 0.031 & $-$0.069 & $-$0.008 & 0.087 & 0.076$^{*}$ & 0.041 & $-$0.012 & $-$0.013 & 0.100$^{*}$ & 0.010 & 0.011 & $-$0.0003 \\ 
  & (0.047) & (0.061) & (0.076) & (0.052) & (0.041) & (0.056) & (0.071) & (0.043) & (0.056) & (0.033) & (0.037) & (0.029) \\ 
  & & & & & & & & & & & & \\ 
 Children & 0.032 & 0.024 & 0.140$^{*}$ & 0.031 & 0.004 & 0.024 & 0.040 & $-$0.001 & 0.051 & $-$0.010 & 0.010 & 0.009 \\ 
  & (0.048) & (0.062) & (0.077) & (0.053) & (0.041) & (0.057) & (0.072) & (0.044) & (0.056) & (0.034) & (0.038) & (0.030) \\ 
  & & & & & & & & & & & & \\ 
 No college & $-$0.004 & $-$0.055 & $-$0.011 & 0.006 & $-$0.056 & $-$0.041 & 0.087 & 0.015 & $-$0.118$^{*}$ & $-$0.017 & $-$0.046 & $-$0.044 \\ 
  & (0.054) & (0.069) & (0.086) & (0.059) & (0.046) & (0.063) & (0.080) & (0.049) & (0.063) & (0.038) & (0.042) & (0.033) \\ 
  & & & & & & & & & & & & \\ 
 Retired & 0.015 & $-$0.011 & 0.197 & $-$0.134 & $-$0.038 & $-$0.053 & 0.195 & 0.024 & $-$0.104 & $-$0.192$^{***}$ & $-$0.077 & $-$0.005 \\ 
  & (0.084) & (0.109) & (0.136) & (0.094) & (0.073) & (0.100) & (0.127) & (0.078) & (0.100) & (0.060) & (0.067) & (0.053) \\ 
  & & & & & & & & & & & & \\ 
 Student & $-$0.203 & $-$0.198 & 0.517 & $-$0.137 & 0.185 & $-$0.140 & $-$0.109 & 0.035 & $-$0.035 & $-$0.153 & $-$0.012 & $-$0.006 \\ 
  & (0.201) & (0.260) & (0.325) & (0.224) & (0.174) & (0.238) & (0.302) & (0.185) & (0.237) & (0.142) & (0.159) & (0.125) \\ 
  & & & & & & & & & & & & \\ 
 Working & 0.021 & $-$0.005 & 0.208 & $-$0.007 & 0.112 & $-$0.011 & 0.053 & 0.105 & 0.023 & $-$0.135$^{**}$ & $-$0.019 & 0.0002 \\ 
  & (0.084) & (0.108) & (0.135) & (0.093) & (0.072) & (0.099) & (0.125) & (0.077) & (0.098) & (0.059) & (0.066) & (0.052) \\ 
  & & & & & & & & & & & & \\ 
 Income Q2 & 0.005 & 0.078 & $-$0.008 & 0.042 & $-$0.121$^{*}$ & $-$0.061 & $-$0.088 & $-$0.087 & $-$0.051 & $-$0.067 & $-$0.081 & $-$0.043 \\ 
  & (0.071) & (0.092) & (0.115) & (0.079) & (0.062) & (0.084) & (0.107) & (0.066) & (0.084) & (0.050) & (0.056) & (0.044) \\ 
  & & & & & & & & & & & & \\ 
 Income Q3 & 0.017 & $-$0.008 & $-$0.064 & 0.054 & $-$0.111$^{*}$ & $-$0.083 & 0.012 & $-$0.051 & $-$0.066 & $-$0.008 & $-$0.092$^{*}$ & $-$0.105$^{**}$ \\ 
  & (0.068) & (0.087) & (0.109) & (0.075) & (0.058) & (0.080) & (0.101) & (0.062) & (0.080) & (0.048) & (0.053) & (0.042) \\ 
  & & & & & & & & & & & & \\ 
 Income Q4 & 0.076 & 0.099 & $-$0.029 & 0.061 & $-$0.079 & $-$0.088 & $-$0.036 & $-$0.052 & $-$0.095 & $-$0.066 & $-$0.128$^{**}$ & $-$0.076$^{*}$ \\ 
  & (0.072) & (0.093) & (0.116) & (0.080) & (0.062) & (0.085) & (0.108) & (0.066) & (0.085) & (0.051) & (0.057) & (0.045) \\ 
  & & & & & & & & & & & & \\ 
 30-49 & $-$0.101 & $-$0.140 & $-$0.217 & $-$0.077 & 0.025 & 0.0004 & $-$0.011 & 0.103 & $-$0.153 & 0.150$^{*}$ & 0.042 & 0.052 \\ 
  & (0.117) & (0.151) & (0.189) & (0.130) & (0.101) & (0.138) & (0.175) & (0.107) & (0.138) & (0.083) & (0.092) & (0.073) \\ 
  & & & & & & & & & & & & \\ 
 50-87 & $-$0.271$^{**}$ & $-$0.068 & $-$0.222 & 0.013 & 0.068 & $-$0.096 & $-$0.081 & 0.103 & 0.040 & 0.135 & 0.041 & 0.033 \\ 
  & (0.120) & (0.155) & (0.194) & (0.134) & (0.104) & (0.142) & (0.180) & (0.111) & (0.142) & (0.085) & (0.095) & (0.075) \\ 
  & & & & & & & & & & & & \\ 
 Non voting & $-$0.069 & $-$0.156 & 0.157 & 0.082 & $-$0.079 & $-$0.212$^{**}$ & $-$0.074 & 0.026 & $-$0.006 & 0.146$^{***}$ & $-$0.019 & 0.013 \\ 
  & (0.075) & (0.097) & (0.121) & (0.083) & (0.065) & (0.089) & (0.113) & (0.069) & (0.089) & (0.053) & (0.059) & (0.047) \\ 
  & & & & & & & & & & & & \\ 
 Other & $-$0.093 & $-$0.264$^{*}$ & $-$0.001 & $-$0.012 & $-$0.118 & $-$0.197 & 0.186 & $-$0.087 & $-$0.097 & 0.081 & $-$0.084 & $-$0.038 \\ 
  & (0.109) & (0.141) & (0.177) & (0.122) & (0.095) & (0.129) & (0.164) & (0.101) & (0.129) & (0.077) & (0.086) & (0.068) \\ 
  & & & & & & & & & & & & \\ 
 Trump & $-$0.085$^{*}$ & $-$0.259$^{***}$ & $-$0.219$^{***}$ & 0.067 & 0.023 & $-$0.134$^{**}$ & 0.465$^{***}$ & $-$0.013 & $-$0.017 & $-$0.014 & $-$0.039 & $-$0.026 \\ 
  & (0.050) & (0.065) & (0.081) & (0.055) & (0.043) & (0.059) & (0.075) & (0.046) & (0.059) & (0.035) & (0.039) & (0.031) \\ 
  & & & & & & & & & & & & \\ 
 Climate treatment only & 0.024 & 0.152$^{*}$ & $-$0.065 & $-$0.100 & $-$0.134$^{**}$ & 0.033 & 0.166 & 0.022 & $-$0.043 & $-$0.065 & 0.039 & 0.049 \\ 
  & (0.068) & (0.088) & (0.110) & (0.075) & (0.059) & (0.080) & (0.102) & (0.062) & (0.080) & (0.048) & (0.054) & (0.042) \\ 
  & & & & & & & & & & & & \\ 
 No treatment & $-$0.025 & 0.203$^{**}$ & 0.032 & $-$0.008 & $-$0.089 & 0.195$^{**}$ & 0.009 & 0.097 & 0.012 & 0.026 & 0.101$^{*}$ & 0.093$^{**}$ \\ 
  & (0.067) & (0.086) & (0.108) & (0.074) & (0.058) & (0.079) & (0.100) & (0.062) & (0.079) & (0.047) & (0.053) & (0.042) \\ 
  & & & & & & & & & & & & \\ 
 Policy treatment only & 0.055 & 0.088 & $-$0.029 & 0.026 & $-$0.081 & 0.017 & 0.079 & 0.157$^{***}$ & 0.058 & 0.011 & 0.086$^{*}$ & 0.039 \\ 
  & (0.061) & (0.079) & (0.098) & (0.068) & (0.053) & (0.072) & (0.091) & (0.056) & (0.072) & (0.043) & (0.048) & (0.038) \\ 
  & & & & & & & & & & & & \\ 
 Constant & 0.206 & 0.236 & 0.466$^{*}$ & 0.010 & 0.065 & 0.312$^{*}$ & 0.078 & $-$0.113 & 0.111 & 0.094 & 0.041 & 0.008 \\ 
  & (0.146) & (0.189) & (0.236) & (0.162) & (0.126) & (0.173) & (0.219) & (0.135) & (0.172) & (0.103) & (0.115) & (0.091) \\ 
  & & & & & & & & & & & & \\ 
\hline \\[-1.8ex] 
Mean & 0.097 & 0.169 & 0.344 & 0.108 & 0.067 & 0.144 & 0.338 & 0.072 & 0.128 & 0.051 & 0.051 & 0.031 \\ 
Observations & 191 & 191 & 191 & 191 & 191 & 191 & 191 & 191 & 191 & 191 & 191 & 191 \\ 
\hline 
\hline \\[-1.8ex] 
\textit{Note:}  & \multicolumn{12}{r}{$^{*}$p$<$0.1; $^{**}$p$<$0.05; $^{***}$p$<$0.01} \\ 
\end{tabular} 
}
	\end{center}
	{\footnotesize Note: See notes under Table \ref{table heating} and Table \ref{table standard opinion} for a description of the covariates.
	\newline *p$<$0.1; **p$<$0.05; ***p$<$0.01}
\end{table}	
\end{landscape}


\begin{table}[h!]
	\caption{Use of media}
	\begin{center}
		\scalebox{0.7}{
\begin{tabular}{@{\extracolsep{5pt}}lccccccc} 
\\[-1.8ex]\hline 
\hline \\[-1.8ex] 
 & \multicolumn{7}{c}{Media mainly used} \\ 
\cline{2-8} 
\\[-1.8ex] & TV (private) & TV (public) & Radio & Social media & Print & News websites & Other \\ 
\\[-1.8ex] & (1) & (2) & (3) & (4) & (5) & (6) & (7)\\ 
\hline \\[-1.8ex] 
 White only & $-$0.011 & 0.160$^{*}$ & $-$0.005 & $-$0.002 & $-$0.045 & $-$0.130$^{*}$ & 0.033 \\ 
  & (0.065) & (0.089) & (0.046) & (0.056) & (0.045) & (0.074) & (0.051) \\ 
  & & & & & & & \\ 
 Male & 0.035 & $-$0.083 & 0.029 & 0.022 & 0.018 & $-$0.016 & $-$0.005 \\ 
  & (0.057) & (0.078) & (0.040) & (0.049) & (0.039) & (0.064) & (0.045) \\ 
  & & & & & & & \\ 
 Children & 0.0003 & 0.019 & 0.075$^{*}$ & 0.025 & 0.042 & $-$0.042 & $-$0.120$^{***}$ \\ 
  & (0.057) & (0.079) & (0.041) & (0.050) & (0.040) & (0.065) & (0.045) \\ 
  & & & & & & & \\ 
 No college & 0.091 & 0.134 & $-$0.066 & $-$0.011 & $-$0.061 & $-$0.087 & 0.001 \\ 
  & (0.064) & (0.088) & (0.046) & (0.055) & (0.045) & (0.073) & (0.051) \\ 
  & & & & & & & \\ 
 Retired & 0.111 & $-$0.340$^{**}$ & 0.026 & 0.139 & 0.041 & 0.040 & $-$0.017 \\ 
  & (0.101) & (0.139) & (0.072) & (0.087) & (0.070) & (0.115) & (0.080) \\ 
  & & & & & & & \\ 
 Student & $-$0.218 & $-$0.240 & 0.575$^{***}$ & 0.049 & $-$0.102 & 0.370 & $-$0.434$^{**}$ \\ 
  & (0.241) & (0.332) & (0.172) & (0.208) & (0.168) & (0.275) & (0.191) \\ 
  & & & & & & & \\ 
 Working & 0.063 & $-$0.171 & 0.015 & 0.190$^{**}$ & $-$0.118$^{*}$ & 0.091 & $-$0.069 \\ 
  & (0.100) & (0.138) & (0.071) & (0.086) & (0.070) & (0.114) & (0.079) \\ 
  & & & & & & & \\ 
 Income Q2 & 0.206$^{**}$ & $-$0.032 & $-$0.041 & $-$0.220$^{***}$ & 0.062 & 0.021 & 0.004 \\ 
  & (0.085) & (0.118) & (0.061) & (0.074) & (0.059) & (0.097) & (0.068) \\ 
  & & & & & & & \\ 
 Income Q3 & 0.051 & 0.063 & $-$0.027 & $-$0.096 & 0.050 & 0.069 & $-$0.109$^{*}$ \\ 
  & (0.081) & (0.112) & (0.058) & (0.070) & (0.056) & (0.092) & (0.064) \\ 
  & & & & & & & \\ 
 Income Q4 & 0.095 & 0.185 & $-$0.053 & $-$0.205$^{***}$ & 0.052 & 0.039 & $-$0.112 \\ 
  & (0.086) & (0.118) & (0.061) & (0.074) & (0.060) & (0.098) & (0.068) \\ 
  & & & & & & & \\ 
 30-49 & 0.119 & $-$0.030 & $-$0.275$^{***}$ & $-$0.099 & 0.047 & 0.242 & $-$0.003 \\ 
  & (0.140) & (0.193) & (0.100) & (0.121) & (0.097) & (0.159) & (0.111) \\ 
  & & & & & & & \\ 
 50-87 & 0.111 & 0.096 & $-$0.216$^{**}$ & $-$0.265$^{**}$ & 0.003 & 0.371$^{**}$ & $-$0.100 \\ 
  & (0.144) & (0.199) & (0.103) & (0.124) & (0.100) & (0.164) & (0.114) \\ 
  & & & & & & & \\ 
 Non voting & 0.047 & $-$0.138 & 0.012 & $-$0.113 & 0.046 & $-$0.082 & 0.227$^{***}$ \\ 
  & (0.090) & (0.124) & (0.064) & (0.078) & (0.063) & (0.102) & (0.071) \\ 
  & & & & & & & \\ 
 Other & $-$0.025 & 0.097 & $-$0.027 & $-$0.116 & 0.217$^{**}$ & $-$0.117 & $-$0.029 \\ 
  & (0.131) & (0.181) & (0.094) & (0.113) & (0.091) & (0.149) & (0.104) \\ 
  & & & & & & & \\ 
 Trump & $-$0.017 & $-$0.019 & 0.147$^{***}$ & $-$0.063 & 0.023 & $-$0.123$^{*}$ & 0.052 \\ 
  & (0.060) & (0.082) & (0.043) & (0.052) & (0.042) & (0.068) & (0.047) \\ 
  & & & & & & & \\ 
 Both treatments & $-$0.057 & 0.025 & $-$0.024 & 0.033 & 0.077 & $-$0.048 & $-$0.006 \\ 
  & (0.080) & (0.110) & (0.057) & (0.069) & (0.056) & (0.091) & (0.063) \\ 
  & & & & & & & \\ 
 Climate treatment only & 0.007 & 0.044 & 0.024 & $-$0.068 & 0.002 & $-$0.065 & 0.057 \\ 
  & (0.075) & (0.104) & (0.054) & (0.065) & (0.053) & (0.086) & (0.060) \\ 
  & & & & & & & \\ 
 Policy treatment only & 0.050 & $-$0.051 & 0.007 & $-$0.082 & $-$0.043 & 0.033 & 0.085 \\ 
  & (0.069) & (0.095) & (0.049) & (0.060) & (0.048) & (0.079) & (0.055) \\ 
  & & & & & & & \\ 
 Constant & $-$0.179 & 0.353 & 0.203 & 0.341$^{**}$ & 0.037 & $-$0.002 & 0.247$^{*}$ \\ 
  & (0.173) & (0.238) & (0.124) & (0.149) & (0.120) & (0.197) & (0.137) \\ 
  & & & & & & & \\ 
\hline \\[-1.8ex] 
Mean & 0.133 & 0.349 & 0.072 & 0.108 & 0.062 & 0.185 & 0.092 \\ 
Observations & 191 & 191 & 191 & 191 & 191 & 191 & 191 \\ 
\hline 
\hline \\[-1.8ex] 
\textit{Note:}  & \multicolumn{7}{r}{$^{*}$p$<$0.1; $^{**}$p$<$0.05; $^{***}$p$<$0.01} \\ 
\end{tabular} 
}
	\end{center}
	{\footnotesize Note: See notes under Table \ref{table heating} and Table \ref{table standard opinion} for a description of the covariates.
	\newline *p$<$0.1; **p$<$0.05; ***p$<$0.01}
\end{table}	

\begin{table}[h!]
	\caption{Survey biased}
	\begin{center}
		\scalebox{0.7}{
\begin{tabular}{@{\extracolsep{5pt}}lccc} 
\\[-1.8ex]\hline 
\hline \\[-1.8ex] 
 & \multicolumn{3}{c}{Survey was biased} \\ 
\cline{2-4} 
\\[-1.8ex] & No & Yes, anti environment & Yes, pro environment \\ 
\\[-1.8ex] & (1) & (2) & (3)\\ 
\hline \\[-1.8ex] 
 White only & $-$0.006 & $-$0.028 & 0.035 \\ 
  & (0.093) & (0.045) & (0.092) \\ 
  & & & \\ 
 Male & $-$0.099 & $-$0.011 & 0.110 \\ 
  & (0.081) & (0.040) & (0.080) \\ 
  & & & \\ 
 Children & $-$0.072 & 0.051 & 0.021 \\ 
  & (0.082) & (0.040) & (0.081) \\ 
  & & & \\ 
 No college & $-$0.095 & 0.050 & 0.045 \\ 
  & (0.092) & (0.045) & (0.091) \\ 
  & & & \\ 
 Retired & 0.158 & $-$0.019 & $-$0.138 \\ 
  & (0.145) & (0.071) & (0.143) \\ 
  & & & \\ 
 Student & 0.089 & 0.148 & $-$0.237 \\ 
  & (0.346) & (0.169) & (0.341) \\ 
  & & & \\ 
 Working & 0.201 & 0.008 & $-$0.208 \\ 
  & (0.144) & (0.070) & (0.141) \\ 
  & & & \\ 
 Income Q2 & 0.095 & $-$0.058 & $-$0.037 \\ 
  & (0.122) & (0.060) & (0.121) \\ 
  & & & \\ 
 Income Q3 & $-$0.061 & $-$0.098$^{*}$ & 0.158 \\ 
  & (0.116) & (0.057) & (0.114) \\ 
  & & & \\ 
 Income Q4 & $-$0.118 & $-$0.110$^{*}$ & 0.227$^{*}$ \\ 
  & (0.123) & (0.060) & (0.121) \\ 
  & & & \\ 
 30-49 & 0.251 & $-$0.136 & $-$0.115 \\ 
  & (0.201) & (0.098) & (0.198) \\ 
  & & & \\ 
 50-87 & 0.392$^{*}$ & $-$0.169$^{*}$ & $-$0.223 \\ 
  & (0.207) & (0.101) & (0.204) \\ 
  & & & \\ 
 Non voting & 0.141 & $-$0.063 & $-$0.078 \\ 
  & (0.129) & (0.063) & (0.127) \\ 
  & & & \\ 
 Other & 0.216 & $-$0.087 & $-$0.129 \\ 
  & (0.188) & (0.092) & (0.185) \\ 
  & & & \\ 
 Trump & $-$0.175$^{**}$ & $-$0.032 & 0.207$^{**}$ \\ 
  & (0.086) & (0.042) & (0.085) \\ 
  & & & \\ 
 Both treatments & $-$0.007 & $-$0.105$^{*}$ & 0.112 \\ 
  & (0.115) & (0.056) & (0.113) \\ 
  & & & \\ 
 Climate treatment only & $-$0.170 & $-$0.115$^{**}$ & 0.285$^{***}$ \\ 
  & (0.108) & (0.053) & (0.107) \\ 
  & & & \\ 
 Policy treatment only & $-$0.236$^{**}$ & $-$0.026 & 0.262$^{***}$ \\ 
  & (0.099) & (0.049) & (0.098) \\ 
  & & & \\ 
 Constant & 0.339 & 0.344$^{***}$ & 0.317 \\ 
  & (0.248) & (0.121) & (0.244) \\ 
  & & & \\ 
\hline \\[-1.8ex] 
Control group mean & 0.583 & 0.104 & 0.312 \\ 
Observations & 191 & 191 & 191 \\ 
\hline 
\hline \\[-1.8ex] 
\textit{Note:}  & \multicolumn{3}{r}{$^{*}$p$<$0.1; $^{**}$p$<$0.05; $^{***}$p$<$0.01} \\ 
\end{tabular} 
}
	\end{center}
	{\footnotesize Note: See notes under Table \ref{table heating} and Table \ref{table standard opinion} for a description of the covariates.
	\newline *p$<$0.1; **p$<$0.05; ***p$<$0.01}
\end{table}	


\end{document}